\documentclass[spanish, 12pt]{exam}

%These tell TeX which packages to use.
\usepackage{array,epsfig}
\usepackage{amsmath}
\usepackage{amsfonts}
\usepackage{amssymb}
\usepackage{amsxtra}
\usepackage{amsthm}
\usepackage{mathrsfs}
\usepackage{color}
\usepackage{multicol}
\usepackage{verbatim}

\usepackage[utf8]{inputenc}
\usepackage[spanish]{babel}
\usepackage{eurosym}

\usepackage{graphicx}
\graphicspath{{../img/}}

\printanswers
\nopointsinmargin
\pointformat{}

%Pagination stuff.
%\setlength{\topmargin}{-.3 in}
%\setlength{\oddsidemargin}{0in}
%\setlength{\evensidemargin}{0in}
%\setlength{\textheight}{9.in}
%\setlength{\textwidth}{6.5in}
%\pagestyle{empty}

\renewcommand{\solutiontitle}{\noindent\textbf{Sol:}\enspace}

\newcommand{\class}{4º Académicas}
\newcommand{\examdate}{\today}
\newcommand{\examnum}{Ecuaciones y sistemas}
\newcommand{\tipo}{A}


\newcommand{\timelimit}{50 minutos}



\pagestyle{head}
\firstpageheader{\includegraphics[width=0.2\columnwidth]{header_left}}{\textbf{Departamento de Matemáticas\linebreak \class}\linebreak \examnum}{\includegraphics[width=0.1\columnwidth]{header_right}}
\runningheader{\class}{\examnum}{Página \thepage\ of \numpages}
\runningheadrule

\begin{document}

\includegraphics[width=0.9\columnwidth]{ec2grado}

\begin{questions}


\question Dadas las siguientes ecuaciones se pide:
\begin{itemize}
\item Resolverlas mediante la fórmula general de la ecuación de segundo grado
\item Comprobar las soluciones obtenidas
\item Factorizar el polinomio del primer miembro de cada ecuación
\item Comprobar las relaciones de Cardano-Vieta
\end{itemize}

\begin{multicols}{3}
\begin{parts}
\part[] $x^2-4x+3=0$ 

\begin{solution} \end{solution}

\part[] $x^2-5x+6=0$ 
\begin{solution} \end{solution}

\part[] $x^2+2x+5=0$ 
\begin{solution} \end{solution}

\part[] $2x^2-5x+2=0$ 
\begin{solution} \end{solution}

\part[] $6x^2-13x+6=0$ 
\begin{solution} \end{solution}

\part[] $x^2-4x+3=0$ 
\begin{solution} \end{solution}


\end{parts}
\end{multicols}

\question Escribir \emph{una} ecuación de 2º grado que tenga por soluciones.
\begin{multicols}{3}
\begin{parts}
\part[] $x_1=4 $, $x_2=-6$ 
\begin{solution} $x^2+2x-24=0$ \end{solution}

\part[] $x_1=-3 $, $x_2=-5$ 
\begin{solution} $x^2+8x+15=0$ \end{solution}

% \part[] $x_1=2 $, $x_2=-7$ 
% \begin{solution} $x^2+5x-14=0$ \end{solution}

% \part[] $x_1=-2/7 $, $x_2=7$ 
% \begin{solution} $7x^2-47x-14=0$ \end{solution}

% \part[] $x_1=-16 $, $x_2=9$ 
% \begin{solution} $x^2+7x-144=0$ \end{solution}

\part[] $x_1=-4 $, $x_2=-1/8$ 
\begin{solution} $8x^2+33x+4=0$ \end{solution}

% \part[] $x_1=2 $, $x_2=-2$ 
% \begin{solution} $x^2-4=0$ \end{solution}

\part[] $x_1=\sqrt{2} $, $x_2=-\sqrt{2}$ 
\begin{solution} $x^2-2=0$ \end{solution}

% \part[] $x_1=2/5 $, $x_2=2/5$ 
% \begin{solution} $25x^2-20x+4=0$ \end{solution}

\part[] $x_1=2+\sqrt{3} $, $x_2=2-\sqrt{3}$ 
\begin{solution} $x^2-4x+1=0$ \end{solution}

\end{parts}
\end{multicols}

\question ¿Para qué valores de \emph{a} la ecuación $x^2-6x+3+a=0$ tiene solución única? 
%\begin{multicols}{3}
\begin{solution} $a=-6 $\end{solution}
%\end{multicols}

\question Resolver las siguientes ecuaciones de \textbf{2º grado incompletas}:
\begin{multicols}{4}
\begin{parts}
\part[] $x^2-5x=0$  
\begin{solution} $x_1=0$, $x_2=5$ 
\end{solution}

% \part[] $2x^2-6x=0$  
% \begin{solution} $x_1=0$, $x_2=3$ 
% \end{solution}

\part[] $2x^2-18=0$  
\begin{solution} $x_1=3$, $x_2=-3$ 
\end{solution}

% \part[] $5x^2+x=0$  
% \begin{solution} $x_1=0$, $x_2=-1/5$ 
% \end{solution}

% \part[] $x^2=x$  
% \begin{solution} $x_1=0$, $x_2=2$ 
% \end{solution}



% \part[] $x^2+x=0$  
% \begin{solution} $x_1=0$, $x_2=-1$ 
% \end{solution}

% \part[] $4x^2-1=0$  
% \begin{solution} $x_1=1/2$, $x_2=-1/2$ 
% \end{solution}

% \part[] $-x^2+12x=0$  
% \begin{solution} $x_1=0$, $x_2=12$ 
% \end{solution}

% \part[] $x^2-10x=0$  
% \begin{solution} $x_1=0$, $x_2=10$ 
% \end{solution}

% \part[] $9x^2-4=0$  
% \begin{solution} $x_1=2/3$, $x_2=-2/3$ 
% \end{solution}

\end{parts}
\end{multicols}


\question Resolver las siguientes ecuaciones de \textbf{2º grado completas}:
\begin{multicols}{3}
\begin{parts}
\part[] $x^2-2x-8=0$  
\begin{solution} $x_1=4$, $x_2=-2$ 
\end{solution}

\part[] $2x^2-\sqrt{2} x-2=0$  
\begin{solution} $x_1=\sqrt{2}$, $x_2=-\sqrt{2}/2$ 
\end{solution}

\part[] $x^2+x+1=0$  
\begin{solution} $x_1=$, $x_2=$ 
\end{solution}

% % \part[] $x^2+x+1=0$  
% % \begin{solution} $x_1=$, $x_2=$ 
% % \end{solution}

% \part[] $x^2+x+1=0$  
% \begin{solution} $x_1=$, $x_2=$ 
% \end{solution}

% \part[] $x^2+x+1=0$  
% \begin{solution} $x_1=$, $x_2=$ 
% \end{solution}

% \part[] $x^2+x+1=0$  
% \begin{solution} $x_1=$, $x_2=$ 
% \end{solution}

% \part[] $x^2+x+1=0$  
% \begin{solution} $x_1=$, $x_2=$ 
% \end{solution}

% \part[] $x^2+x+1=0$  
% \begin{solution} $x_1=$, $x_2=$ 
% \end{solution}

% \part[] $x^2+x+1=0$  
% \begin{solution} $x_1=$, $x_2=$ 
% \end{solution}

% \part[] $x^2+x+1=0$  
% \begin{solution} $x_1=$, $x_2=$ 
% \end{solution}

% \part[] $x^2+x+1=0$  
% \begin{solution} $x_1=$, $x_2=$ 
% \end{solution}

% \part[] $x^2+x+1=0$  
% \begin{solution} $x_1=$, $x_2=$ 
% \end{solution}

% \part[] $x^2+x+1=0$  
% \begin{solution} $x_1=$, $x_2=$ 
% \end{solution}


\end{parts}
\end{multicols}


\question Resuelve las siguientes ecuaciones:
% \begin{multicols}{0}
\begin{parts}
\part[] $18 x^{4} + 3 x^{3} = 3 x^{2}$  
\begin{solution}$18 x^{4} + 3 x^{3} = 3 x^{2} \to 3 x^{3} \left(6 x + 1\right) = 3 x^{2} \to x=- \frac{1}{2}, x=0, x=\frac{1}{3}$ \end{solution}

\part $5 x^{4} - 20 x^{3} + 10 x^{2} + 20 x - 15=0$
\begin{solution}
$5 x^{4} - 20 x^{3} + 10 x^{2} + 20 x - 15 \to 5 \left(x - 3\right) \left(x - 1\right)^{2} \left(x + 1\right) \to x=-1, x=1, x=3$
\end{solution}

\part[]$\sqrt{x+4}-7=0$
\begin{solution}
$\sqrt{x+4}-7=0 \to \sqrt{x + 4} - 7 = 0 \to x=45$
\end{solution}

\part $\sqrt{169-x^2}+17=x$
\begin{solution}
$\sqrt{169-x^2}+17=x \to \sqrt{- x^{2} + 169} + 17 = x \to \nexists$
\end{solution}

\part[]$\sqrt{3x+1}=1+\sqrt{2x-1}$
\begin{solution}
 $\sqrt{3x+1}=1+\sqrt{2x-1} \to \sqrt{3 x + 1} = \sqrt{2 x - 1} + 1 \to x=1, x=5$
\end{solution}
\end{parts}
% \end{multicols}


\begin{comment}
\question 
\begin{multicols}{3}
\begin{parts}
\part[]  
\begin{solution} \end{solution}

\end{parts}
\end{multicols}
\end{comment}

\end{questions}
\end{document}


