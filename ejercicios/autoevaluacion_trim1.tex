
        \documentclass[spanish, 11pt]{exam}

        %These tell TeX which packages to use.
        \usepackage{array,epsfig}
        \usepackage{amsmath, textcomp}
        \usepackage{amsfonts}
        \usepackage{amssymb}
        \usepackage{amsxtra}
        \usepackage{amsthm}
        \usepackage{mathrsfs}
        \usepackage{color}
        \usepackage{multicol, xparse}
        \usepackage{verbatim}


        \usepackage[utf8]{inputenc}
        \usepackage[spanish]{babel}
        \usepackage{eurosym}

        \usepackage{graphicx}
        \graphicspath{{../img/}}



        \printanswers
        \nopointsinmargin
        \pointformat{}

        %Pagination stuff.
        %\setlength{\topmargin}{-.3 in}
        %\setlength{\oddsidemargin}{0in}
        %\setlength{\evensidemargin}{0in}
        %\setlength{\textheight}{9.in}
        %\setlength{\textwidth}{6.5in}
        %\pagestyle{empty}

        \let\multicolmulticols\multicols
        \let\endmulticolmulticols\endmulticols
        \RenewDocumentEnvironment{multicols}{mO{}}
         {%
          \ifnum#1=1
            #2%
          \else % More than 1 column
            \multicolmulticols{#1}[#2]
          \fi
         }
         {%
          \ifnum#1=1
          \else % More than 1 column
            \endmulticolmulticols
          \fi
         }
        \renewcommand{\solutiontitle}{\noindent\textbf{Sol:}\enspace}

        \newcommand{\samedir}{\mathbin{\!/\mkern-5mu/\!}}

        \newcommand{\class}{4º ESO}
        \newcommand{\examdate}{\today}

        \newcommand{\tipo}{A}


        \newcommand{\timelimit}{50 minutos}



        \pagestyle{head}
        \firstpageheader{\includegraphics[width=0.2\columnwidth]{header_left}}{\textbf{Departamento de Matemáticas\linebreak \class}\linebreak \examnum}{\includegraphics[width=0.1\columnwidth]{header_right}}
        \runningheader{\class}{\examnum}{Página \thepage\ of \numpages}
        \runningheadrule

        \newcommand{\examnum}{Autoevaluación - Trimestre 1}
        \begin{document}
        \begin{questions}
        
        \question a02e00 - Racionaliza:
        
        \begin{multicols}{2} 
        \begin{parts} \part[1] $\dfrac{\sqrt{5}-2}{2+\sqrt{5}}$  \begin{solution}   $9 - 4 \sqrt{5}$ \end{solution} 
        \part[] $\dfrac{42+14\sqrt{2}}{2\sqrt{2}-6}$
        \begin{solution}
        $-11 - 6 \sqrt{2}$
        \end{solution}
        \end{parts}
        \end{multicols}
        \question a02e01 -  Aplica la definición de logaritmo para calcular: 
        \begin{multicols}{3} 
        \begin{parts} \part[1]  $\log_{25}{5}$  \begin{solution}   $\frac{1}{2}$  \end{solution} 
        \part[1]  $\log_{0.5}{1/8}$  \begin{solution}  $3$  \end{solution} 
                \part[1]  $\log_{0.5}{32}$  \begin{solution}  $-5$  \end{solution} 
        \end{parts}
        \end{multicols}
        \question a02e02 - Halla el valor de k para que la siguiente división sea exacta:
        \begin{multicols}{2} 
        \begin{parts} \part[1]  $\left( 3x^2+kx-15 \right): \left(x - 3\right)$  \begin{solution}  $3 k + 12=0 \to k = -4 $  \end{solution} 
        \part[1] $\left( k x^{3} - 6 x^{2} + 7 x - 14 \right): \left(x - 2\right)$  \begin{solution}  $8 k - 24=0 \to k = 3 $  \end{solution} 
        \end{parts}
        \end{multicols}
        \question a02e03 - Factoriza:
        \begin{multicols}{2} 
        \begin{parts} \part $4 x^{5} - 12 x^{4} + 13 x^{3} - 6 x^{2} + x$  \begin{solution}$x \left(x - 1\right)^{2} \left(2 x - 1\right)^{2}$\end{solution}  \part $4 x^{4} - 24 x^{3} + 21 x^{2} - 5 x$  \begin{solution}$x \left(x - 5\right) \left(2 x - 1\right)^{2}$\end{solution}  \part $x^{3} - 11 x^{2} + 40 x - 48$  \begin{solution}$\left(x - 4\right)^{2} \left(x - 3\right)$\end{solution}  \part $8 x^{4} - 62 x^{3} + 139 x^{2} - 79 x + 12$  \begin{solution}$\left(x - 4\right) \left(x - 3\right) \left(2 x - 1\right) \left(4 x - 1\right)$\end{solution}
        \end{parts}
        \end{multicols}
        \question a02e04 - Simplifica la siguiente fracción algebraica:
        \begin{multicols}{1} 
        \begin{parts} 
        \part[]$\dfrac{2 x^{3} - x^{2} - 6 x}{x^{3} - 2 x^{2}}$
        \begin{solution}
        $\frac{2 x^{3} - x^{2} - 6 x}{x^{3} - 2 x^{2}}=\frac{x \left(x - 2\right) \left(2 x + 3\right)}{x^{2} \left(x - 2\right)}=\frac{2 x + 3}{x}$
        \end{solution}
        \part[1]  $\dfrac{2 x^{4} - 12 x^{3} + 24 x^{2} - 16 x}{6 x^{4} - 36 x^{3} + 72 x^{2} - 48 x}$  \begin{solution} $\frac{2 x^{4} - 12 x^{3} + 24 x^{2} - 16 x}{6 x^{4} - 36 x^{3} + 72 x^{2} - 48 x}=\frac{2 x \left(x - 2\right)^{3}}{6 x \left(x - 2\right)^{3}}=\frac{1}{3}$
  \end{solution} 
  
        \end{parts}
        \end{multicols}
        \question a01e05 - Resuelve las siguientes ecuaciones
        \begin{multicols}{1} 
        \begin{parts} 
        \part[1]  $4 x^{5} - 12 x^{4} + 13 x^{3} - 6 x^{2} + x = 0$  
        \begin{solution}   
        $4 x^{5} - 12 x^{4} + 13 x^{3} - 6 x^{2} + x = 0 \to x \left(x - 1\right)^{2} \left(2 x - 1\right)^{2} = 0 \to x=0, x=\frac{1}{2}, x=1$
        \end{solution}
        \part[1]  $x^{3} - 11 x^{2} + 40 x - 48= 0$  
        \begin{solution}   
        $x^{3} - 11 x^{2} + 40 x - 48= 0 \to \left(x - 4\right)^{2} \left(x - 3\right) = 0 \to x=3, x=4$
        \end{solution}
        \part[1] $3\sqrt{x-1}+11=2x$  
        \begin{solution}   
        $3\sqrt{x-1}+11=2x \to x=10$
        \end{solution}
        \part[1] $\sqrt{x}+\sqrt{x-4}=2$  
        \begin{solution}   
        $\sqrt{x}+\sqrt{x-4}=2 \to x=4$
        \end{solution}
        \part[1]  $\dfrac{{x + 1}}{{x - 1}} - \dfrac{1}{x}=\dfrac{5}{2}$ 
        \begin{solution}   
        $\frac{{x + 1}}{{x - 1}} - \frac{1}{x}=\frac{5}{2} \to \frac{x^{2} + 1}{x \left(x - 1\right)} = \frac{5}{2}  \to \frac{2}{x \left(x - 1\right)} = \frac{5}{x^{2} + 1}  \to \frac{2}{x \left(x - 1\right)} = \frac{5}{x^{2} + 1}  \to \frac{2}{x^{2} - x} = \frac{5}{x^{2} + 1}  \to 2 \left(x^{2} + 1\right) = 5 \left(x^{2} - x\right)  \to 2 \left(x^{2} + 1\right) - 5 \left(x^{2} - x\right) = 0  \to - 3 x^{2} + 5 x + 2 = 0 \to x=- \frac{1}{3}, x=2$
        \end{solution}
        \part[1]  $2^{x+1}+2^{x-1}=20$  
        \begin{solution}   
        $x=3$
        \end{solution}
        
        \part[1]  $2^{x-1}+2^{x-2}+2^{x-3}=224$  
        \begin{solution}   
        $x=8$
        \end{solution}
        
        \part[1]  $\log 2 + \log x =1$  
        \begin{solution}   
        $x=5$
        \end{solution}
        
        \part[1]  $5\log x -\log 32 = \log \frac{x}{2}$  
        \begin{solution}   
        $x=2$
        \end{solution}
        \end{parts}
        \end{multicols}

    \end{questions}
    \end{document}
    