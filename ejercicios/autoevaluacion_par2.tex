
\documentclass[spanish, 11pt]{exam}

%These tell TeX which packages to use.
\usepackage{array,epsfig}
\usepackage{amsmath, textcomp}
\usepackage{amsfonts}
\usepackage{amssymb}
\usepackage{amsxtra}
\usepackage{amsthm}
\usepackage{mathrsfs}
\usepackage{color}
\usepackage{multicol, xparse}
\usepackage{verbatim}


\usepackage[utf8]{inputenc}
\usepackage[spanish]{babel}
\usepackage{eurosym}

\usepackage{graphicx}
\graphicspath{{../img/}}



\printanswers
\nopointsinmargin
\pointformat{}

%Pagination stuff.
%\setlength{\topmargin}{-.3 in}
%\setlength{\oddsidemargin}{0in}
%\setlength{\evensidemargin}{0in}
%\setlength{\textheight}{9.in}
%\setlength{\textwidth}{6.5in}
%\pagestyle{empty}

\let\multicolmulticols\multicols
\let\endmulticolmulticols\endmulticols
\RenewDocumentEnvironment{multicols}{mO{}}
 {%
  \ifnum#1=1
    #2%
  \else % More than 1 column
    \multicolmulticols{#1}[#2]
  \fi
 }
 {%
  \ifnum#1=1
  \else % More than 1 column
    \endmulticolmulticols
  \fi
 }
\renewcommand{\solutiontitle}{\noindent\textbf{Sol:}\enspace}

\newcommand{\samedir}{\mathbin{\!/\mkern-5mu/\!}}

\newcommand{\class}{4º ESO}
\newcommand{\examdate}{\today}

\newcommand{\tipo}{A}


\newcommand{\timelimit}{50 minutos}



\pagestyle{head}
\firstpageheader{\includegraphics[width=0.2\columnwidth]{header_left}}{\textbf{Departamento de Matemáticas\linebreak \class}\linebreak \examnum}{\includegraphics[width=0.1\columnwidth]{header_right}}
\runningheader{\class}{\examnum}{Página \thepage\ of \numpages}
\runningheadrule

\newcommand{\examnum}{Autoevaluación - Parcial 2}
\begin{document}
\begin{questions}

\question Plantea algebraicametne y resuelve los siguientes problemas

\begin{parts} 
\part[1] Encuentra dos números tales que su suma sea 26 
y la mitad de su diferencia sea 4.
\begin{solution}
$\left\{\begin{matrix}x+y=26 \\ (x-y)/2=4\end{matrix}\right. \to  x = 17, \  y = 9$
\end{solution}
\part[1] Si a un número de dos cifras le sumamos 9 se obtiene un número con 
             las cifras intercambiadas entre sí. Sabiendo que la suma de las cifras de 
             ese número es 11, encuéntralo.
\begin{solution}
$\left\{\begin{matrix}10x+y+9=10y+x \\ x+y=11\end{matrix}\right. \to  x = 5, \  y = 6 \to 56$
\end{solution}
\part[1] Si se aumenta la longitud de un campo rectangular en 5 m y la anchura en 7 m, 
la superficie aumenta en 830 $m^2$; mientras que si se disminuye la longitud en 8 m y la anchura en 4 m, 
la superficie disminuye en 700 $m^2$. Calcular las dimensiones del campo
  \begin{solution}    $\left\{\begin{matrix}(x+5)\cdot(y+7)=xy+830 \\ (x-8)\cdot(y-4)=xy-700\end{matrix}\right. \to  x = 75, \  y = 54$ \end{solution} 
\part[] Busca dos números consecutivos tales que, añadiendo al mayor la mitad del menor, 
el resultado excede en 13 a la suma de la quinta parte del menor 
con la onceava parte del mayor. 
\begin{solution}
$\left\{\begin{matrix}y=x+1 \\ y+\frac{x}{2}=\frac{x}{5}+\frac{y}{11}+13\end{matrix}\right. \to  x = 10, \  y = 11$
\end{solution}
\part[] En un triángulo rectángulo, un cateto mide 24 cm y la hipotenusa supera en 18 cm al otro cateto. 
Busca el perímetro y el área del triángulo.
\begin{solution}
$\left\{\begin{matrix}y=x+18 \\ y^2=24^2+x^2\end{matrix}\right. \to \left[  x = 7, \  y = 25\right] \to  P=56 cm \land A=84 cm^2$
\end{solution}
\part[] En un corral hay conejos y gallinas, en total 50 cabezas y 134 patas. \
            ¿Cuántos animales hay de cada clase?
\begin{solution}
$\left\{\begin{matrix}50=x+y \\ 134=4x+2y\end{matrix}\right. \to  x = 17, \  y = 33$
\end{solution}
\end{parts}

\question Resuelve los siguientes sistemas no lineales:
%\begin{multicols}{2} 
\begin{parts} \part[1]  $\left\{\begin{matrix}3x+y=5 \\ x^2-y^2=3\end{matrix}\right. $  \begin{solution}   $ \to \left[ \left\{ x = \frac{7}{4}, \  y = - \frac{1}{4}\right\}, \  \left\{ x = 2, \  y = -1\right\}\right]$  \end{solution} 
\part[1] $\left\{\begin{matrix}2x^2-3y^2=-6 \\ 4x^2-y^2=8\end{matrix}\right. $  \begin{solution}  $ \to \left[ \left\{ x = - \sqrt{3}, \  y = -2\right\}, \  \left\{ x = - \sqrt{3}, \  y = 2\right\}, \  \left\{ x = \sqrt{3}, \  y = -2\right\}, \  \left\{ x = \sqrt{3}, \  y = 2\right\}\right]$  \end{solution} 
\part[1]  $\left\{\begin{matrix}5^{x+y}=25^3 \\ 5^{x-y}=25\end{matrix}\right. $  \begin{solution}  $ \left\{ x = 4, \  y = 2\right\}$  \end{solution} 
\part[] $\left\{\begin{matrix}x+y=22 \\ \log_{10} x-\log_{10} y =1\end{matrix}\right. $
\begin{solution}
$ \to \left[ \left\{ x = 20, \  y = 2\right\}\right]$
\end{solution}
\end{parts}
%\end{multicols}
\question Resuelve las siguientes inecuaciones:
\begin{multicols}{2} 
\begin{parts} \part[1]  $x\cdot(x+3) - 2x > 4x + 4$  \begin{solution}  $\left(-\infty, -1\right) \cup \left(4, \infty\right)$  \end{solution} 
\part[1] $2x^2 - 4x - 6 > 0$  \begin{solution}  $\left(-\infty, -1\right) \cup \left(3, \infty\right)$  \end{solution}
\part[] $x^4 + 2x^2 - 3x < 0$
\begin{solution}
$\left(0, 1\right)$
\end{solution}
\part[] $- x^{5} + 2x \leq x$
\begin{solution}
$\left[-1, 0\right] \cup \left[1, \infty\right)$
\end{solution}
\end{parts}
\end{multicols}
\question Resuelve los siguientes sistemas de inecuaciones:
\begin{multicols}{2} 
\begin{parts} \part $\left\{\begin{matrix}2x + 8 > 0 \\ x + \frac{1}{2} \geq \frac{x}{3}\end{matrix}\right.$ \begin{solution}$\left[- \frac{3}{4}, \infty\right)$\end{solution}  \part $\left\{\begin{matrix}\frac{x}{3} - \frac{x}{2} \leq 1 \\ {( {x + 1} )^2} - {x^2} \geq 1\end{matrix}\right.$  \begin{solution}$\left[0, \infty\right)$\end{solution}  \part $\left\{\begin{matrix}\frac{{x - 4}}{2} - \frac{{x - 2}}{3} \leq 12 \\  \frac{x}{3} - \frac{x}{2} \geq 6\end{matrix}\right.$  \begin{solution}$\left(-\infty, -36\right]$\end{solution}  \part $\left\{\begin{matrix}2x + 6 > 8 \\ x + \frac{1}{3} \leq \frac{x}{2}\end{matrix}\right.$  \begin{solution}$\emptyset$\end{solution}
\end{parts}
\end{multicols}

\end{questions}
\end{document}
