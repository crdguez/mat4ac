\documentclass[spanish, 12pt]{exam}

%These tell TeX which packages to use.
\usepackage{array,epsfig}
\usepackage{amsmath}
\usepackage{amsfonts}
\usepackage{amssymb}
\usepackage{amsxtra}
\usepackage{amsthm}
\usepackage{mathrsfs}
\usepackage{color}
\usepackage{multicol}

\usepackage[utf8]{inputenc}
\usepackage[spanish]{babel}
\usepackage{eurosym}

\usepackage{graphicx}
\graphicspath{{../img/}}

\printanswers
\nopointsinmargin
\pointformat{}

%Pagination stuff.
%\setlength{\topmargin}{-.3 in}
%\setlength{\oddsidemargin}{0in}
%\setlength{\evensidemargin}{0in}
%\setlength{\textheight}{9.in}
%\setlength{\textwidth}{6.5in}
%\pagestyle{empty}

\renewcommand{\solutiontitle}{\noindent\textbf{Sol:}\enspace}

\newcommand{\class}{4º Académicas}
\newcommand{\examdate}{\today}
\newcommand{\examnum}{Polinomios}
\newcommand{\tipo}{A}


\newcommand{\timelimit}{50 minutos}



\pagestyle{head}
\firstpageheader{\includegraphics[width=0.2\columnwidth]{header_left}}{\textbf{Departamento de Matemáticas\linebreak \class}\linebreak \examnum}{\includegraphics[width=0.1\columnwidth]{header_right}}
\runningheader{\class}{\examnum}{Página \thepage\ of \numpages}
\runningheadrule

\begin{document}



\begin{questions}


\question Realiza las siguientes divisiones:
\begin{parts}
\part[] $\left( {2{x^3} + 15x + 3 - 9{x^2}} \right):\left( { - 2x + {x^2} + 1} \right)$ 
\begin{solution} cociente: $2x-5$ \\ resto: $-3x+8$ \end{solution}
\part[] $\left( { - 5 + x - 2{x^2} - 6{x^3} + 5{x^4}} \right):\left( {{x^2} + x - 1} \right)$ \begin{solution} cociente: $5x^2-11x+14$ \\ resto: $-24x+9$ \end{solution}
\end{parts}

\question Averigua si ${x^2} + 3$ es divisor de $12{x^4} - 26{x^3} + 2{x^2} + 15x$
\begin{solution} NO porque \\
$12x^2-26x-34$ \\ resto: $93x+102$ \end{solution}

\question Halla los valores de $m, n$ y $p$ sabiendo que $\left( {x - 2} \right)\left( {m{x^2} + nx + p} \right) = 2{x^3} - 9{x^2} + 14x - 8$
\begin{solution} $m=2$, $n=-3$, $p=8$ \end{solution}


\question Aplicar Ruffini para realizar las siguientes divisiones:
\begin{multicols}{2}
\begin{parts}
\part[] $\left( {{x^4} - 8{x^2} + 2x - 5} \right):\left( {x - 2} \right)$ \begin{solution} Cociente $x^3+2x^2-4x-6$ \\ Resto $17$ \end{solution}
\part[] $\left( {{x^2} - 9x + 7} \right):\left( {x + 5} \right)$ \begin{solution} Cociente $x-14$ \\ Resto $77$ \end{solution}
\end{parts}
\end{multicols}


\question Halla el valor de k para que:
\begin{multicols}{2}
\begin{parts}
\part[] ${x^2} + kx + 6$ sea divisible por $x - 2$ \begin{solution} $k=-5$ \end{solution}
\part[] $5{x^4} + k{x^3} + 2x - 3$ tenga como factor $x + 1$ \begin{solution} $k=0$ \end{solution}
\part[] $\left( {{x^5} - {x^4} + x + 3k} \right)$ : $\left( {x - 2} \right)$ tenga como resto $5$ \begin{solution} $k=-\frac{13}{3}$ \end{solution}
\end{parts}
\end{multicols}

	
\question Halla el resto de la división del polinomio $P(x) = {x^4} - 2{x^3} + 4x - 5$ entre $x + 2$ aplicando el teorema del resto.
\begin{solution} Resto: $19$ \end{solution}

\question Factoriza los siguientes polinomios, diciendo también sus raíces:
\begin{multicols}{2}
\begin{parts}
\part[] $P\left( x \right) = 8{x^4} - 6{x^3} - 5{x^2} + 3x$ \begin{solution} $ $ \end{solution}
\part[] $P\left( x \right) = {x^4} + {x^3} + 3{x^2} + 5x - 10$ \begin{solution} $ $ \end{solution}
\part[] $P(x) = 2{x^3} - 2{x^2} - 8x + 8$ \begin{solution} $ $ \end{solution}
\part[] $P(x) = 6{x^3} + 11{x^2} - 3x - 2$ \begin{solution} $ $ \end{solution}
\end{parts}
\end{multicols}

\question Simplifica las siguientes fracciones algebraicas (recuerda que antes hay que factorizar, en caso de que no lo esté):
\begin{multicols}{3}
\begin{parts}
\part[] $$\frac{{x{{\left( {x + 2} \right)}^2}{{\left( {x - 3} \right)}^2}\left( {x - 1} \right)}}{{{x^2}{{\left( {x + 2} \right)}^3}\left( {x - 3} \right)\left( {x - 1} \right)}}$$ \begin{solution} $ $ \end{solution}
\part[] $$\frac{{12{x^2} - 12xy}}{{12xy - 12{y^2}}}$$ \begin{solution} $ $ \end{solution}
\part[] $$\frac{{{a^2} - ab}}{{{a^4} - {a^2}{b^2}}}$$ \begin{solution} $ $ \end{solution}
\part[] $$\frac{{2{x^4} + 2{x^3} + 2{x^2} + 2x}}{{4{x^2} + 8x + 4}}$$ \begin{solution} $ $ \end{solution}
\part[] $$\frac{{2{x^4} + 2{x^3} + 2{x^2} + 2x}}{{4{x^2} + 8x + 4}}$$ \begin{solution} $ $ \end{solution}
\end{parts}
\end{multicols}

\question Halla el \textit{m.c.m} y \textit{m.c.d} de los siguientes polinomios:
\begin{parts}
\part[] $P(x) = {x^2} - 4$, $Q(x) = {x^4} + 9{x^3} + 30{x^2}$  y  $R(x) = {x^2} + 4x + 4$ \begin{solution} $ $ \end{solution}
\part[] $P(x) = 2{x^2} + 2x$  y  $L(x) = {x^3} - {x^2} - x + 1$  \begin{solution} $ $ \end{solution}
\end{parts}


\question Opera y simplifica:
\begin{multicols}{3}
\begin{parts}
\part[] $$\frac{x}{{x - 2}} - \frac{x}{{x - 1}} - \frac{x}{{{x^2} - 3x + 2}}$$ \begin{solution} $ $ \end{solution}
\part[] $$\frac{1}{{{x^2} - x}} + \frac{{2x - 1}}{{x - 1}} - \frac{{3x - 1}}{x}$$ \begin{solution} $ $ \end{solution}
\part[] $$\frac{{2{x^3} - 5{x^2} + 3x}}{{2{x^2} + x - 6}}$$ \begin{solution} $ $ \end{solution}
\part[] $$\frac{{3{x^3} - 3x}}{{{x^5} - x}}$$ \begin{solution} $ $ \end{solution}
\part[] $$\left( {\frac{1}{x} + x} \right)\left( {1 - \frac{1}{{x + 1}}} \right)$$ \begin{solution} $ $ \end{solution}
\part[] $$1 + \frac{1}{{2x - 1}} - \frac{{2x}}{{4{x^2} - 1}}$$ \begin{solution} $ $ \end{solution}

\end{parts}
\end{multicols}



\end{questions}
\end{document}


