\documentclass[spanish, 12pt]{exam}

%These tell TeX which packages to use.
\usepackage{array,epsfig}
\usepackage{amsmath}
\usepackage{amsfonts}
\usepackage{amssymb}
\usepackage{amsxtra}
\usepackage{amsthm}
\usepackage{mathrsfs}
\usepackage{color}
\usepackage{multicol}
\usepackage{verbatim}

\usepackage[utf8]{inputenc}
\usepackage[spanish]{babel}
\usepackage{eurosym}

\usepackage{graphicx}
\graphicspath{{../img/}}

\printanswers
\nopointsinmargin
\pointformat{}

%Pagination stuff.
%\setlength{\topmargin}{-.3 in}
%\setlength{\oddsidemargin}{0in}
%\setlength{\evensidemargin}{0in}
%\setlength{\textheight}{9.in}
%\setlength{\textwidth}{6.5in}
%\pagestyle{empty}

\renewcommand{\solutiontitle}{\noindent\textbf{Sol:}\enspace}

\newcommand{\class}{4º Académicas}
\newcommand{\examdate}{\today}
\newcommand{\examnum}{Sistemas de Ecuaciones}
\newcommand{\tipo}{A}


\newcommand{\timelimit}{50 minutos}



\pagestyle{head}
\firstpageheader{\includegraphics[width=0.2\columnwidth]{header_left}}{\textbf{Departamento de Matemáticas\linebreak \class}\linebreak \examnum}{\includegraphics[width=0.1\columnwidth]{header_right}}
\runningheader{\class}{\examnum}{Página \thepage\ of \numpages}
\runningheadrule

\begin{document}



\begin{questions}


\question Resolver por el método de sustitución los siguientes sistemas de ecuaciones:
\begin{multicols}{2}
\begin{parts}
\part[] $\left. \begin{gathered}
	  3x - 2y = 2 \hfill \\
	  x + y = 4 \hfill \\ 
	\end{gathered}  \right\}$ 
\begin{solution} x=2; y=2 \end{solution}
\part[] $\left. \begin{gathered}
	  x - 3y = 4 \hfill \\
	  2x + 9y = 23 \hfill \\ 
	\end{gathered}  \right\}$ \begin{solution} x=7; y=1 \end{solution}

\part[] $\left. \begin{gathered}
	  4\left( {x + 4} \right) - 5\left( {y + 8} \right) =  - 15 \hfill \\
	  3\left( {y + 1} \right) - 2\left( {x - 1} \right) = 0 \hfill \\ 
	\end{gathered}  \right\}$
\begin{solution} x=1; y=-1 \end{solution}

\part[] $\left. \begin{gathered}
	  2x - y - 5 = 0 \hfill \\
	  x + 3y - 13 = 0 \hfill \\ 
	\end{gathered}  \right\}$
\begin{solution} x=4; y=3 \end{solution}

\part[]$\left. \begin{gathered}
	  2\left( {x - 4} \right) - 3\left( {y - 7} \right) + 22 = 0 \hfill \\
	  2\left( {x + 1} \right) + 4\left( {y + 1} \right) - 16 = 0 \hfill \\ 
	\end{gathered}  \right\}$ 
\begin{solution} x=-55/7; y=45/7 \end{solution}




\end{parts}
\end{multicols}



\question Resolver por el método de igualación los sistemas de ecuaciones que siguen:
\begin{multicols}{2}
\begin{parts}
\part[] $\left. \begin{gathered}
	  4x - 2y = 16 \hfill \\
	  3x - 7y = 1 \hfill \\ 
	\end{gathered}  \right\}$
\begin{solution} x=5; y=2 \end{solution}

\part[] $\left. \begin{gathered}
	  x - 7\left( {y + 4} \right) =  - 5 \hfill \\
	  2x - 3y - 19 =  - 6 \hfill \\ 
	\end{gathered}  \right\}$
\begin{solution}  x=2; y=-3 \end{solution}

\part[] $\left. \begin{gathered}
	  \frac{{3x}}{5} - \frac{{2y}}{3} = 7 \hfill \\
	  \frac{{5x}}{3} - 2y = 2 \hfill \\ 
	\end{gathered}  \right\}$ 
\begin{solution} x=285/2; y=471/4 \end{solution}

\part[] $\left. \begin{gathered}
	  3x + 2\left( {x - y} \right) = 3 \hfill \\
	  5x - y = 3 \hfill \\ 
	\end{gathered}  \right\}$
\begin{solution} x=3/5; y=0 \end{solution}

\part[] $\left. \begin{gathered}
	  3x + 2y =  - 7 \hfill \\
	  \frac{1}{3}x - \frac{1}{4}y = \frac{1}{6} \hfill \\ 
	\end{gathered}  \right\}$
\begin{solution} x=-1; y=-2 \end{solution}


\end{parts}
\end{multicols}

\question Resolver por el método de reducción los sistemas de ecuaciones siguientes:
\begin{multicols}{2}
\begin{parts}
\part[] $\left. \begin{gathered}
	  x - 3y = 4 \hfill \\
	  x + 7y = 24 \hfill \\ 
	\end{gathered}  \right\}$
\begin{solution} x=10; y=2 \end{solution}

\part[] $\left. \begin{gathered}
	  \frac{{x + y}}{2} - \frac{{x - y}}{2} = 2 \hfill \\
	  5x - 10y = 40 \hfill \\ 
	\end{gathered}  \right\}$
\begin{solution}  x=12; y=2 \end{solution}

\part[] $\left. \begin{gathered}
	  3y - 2x - 16 = 0 \hfill \\
	  2\left( {x - 5} \right) + 6\left( {y - 2} \right) + 20 = 0 \hfill \\ 
	\end{gathered}  \right\}$ 
\begin{solution} x=-5; y=2 \end{solution}

\part[] $\left. \begin{gathered}
	  2x - y = 7 \hfill \\
	  \frac{4}{3}x - \frac{1}{3}y = \frac{{19}}{3} - 4 \hfill \\ 
	\end{gathered}  \right\}$
\begin{solution} x=0; y=-7 \end{solution}

\part[] $\left. \begin{gathered}
	  \frac{x}{2} - \frac{y}{3} = 2 \hfill \\
	  \frac{{x - 1}}{3} + \frac{{y - 2}}{2} = \frac{{13}}{6} \hfill \\ 
	\end{gathered}  \right\}$
\begin{solution} x=6; y=3 \end{solution}


\end{parts}
\end{multicols}

\question Resuelve, por el método que estimes conveniente, los sistemas de ecuaciones:
\begin{multicols}{2}
\begin{parts}

\part[] $\left. \begin{gathered}
	  5x - 3y = 14 \hfill \\
	  x + 2y = 0 \hfill \\ 
	\end{gathered}  \right\}$
\begin{solution} x=28/13; y=-14/13 \end{solution}

\part[] $\left. \begin{gathered}
	  \frac{{9x}}{{17}} - \frac{{4y}}{3} = 0 \hfill \\
	  3y - 81 = 0 \hfill \\ 
	\end{gathered}  \right\}$
\begin{solution} x=68; y=27 \end{solution}

\part[] $\left. \begin{gathered}
	  \frac{{x - 2}}{3} + \frac{{y - 1}}{4} - 1 = x \hfill \\
	  3y - 8x = 17 \hfill \\ 
	\end{gathered}  \right\}$
\begin{solution} Incompatible \end{solution}

\part[] $\left. \begin{gathered}
	  3\left( {\frac{{x - 2}}{4}} \right) - \frac{{2\left( {x - 1} \right)}}{5} = x + 3 \hfill \\
	  \frac{{2x}}{3} - \frac{{3y}}{4} = \frac{4}{5} \hfill \\ 
	\end{gathered}  \right\}$
\begin{solution} x=-82/13;y=-3904/585 \end{solution}

\part[] $\left. \begin{gathered}
	  2\left( {x - 3} \right) + 5\left( {\frac{y}{2} - 1} \right) + 1 = 0 \hfill \\
	  3\left( {x + 1} \right) - \frac{{5x + 5y - 2}}{9} - 1 = 0 \hfill \\ 
	\end{gathered}  \right\}$
\begin{solution} x=0; y=4 \end{solution}

\part[] $\left. \begin{gathered}
	  2x - y = 1 \hfill \\
	  4x + 3y = 1 \hfill \\ 
	\end{gathered}  \right\}$
\begin{solution} x=2/5; y=-1/5 \end{solution}

\part[] $\left. \begin{gathered}
	  4x - 5y =  - 1 \hfill \\
	  7x + 8y = 15 \hfill \\ 
	\end{gathered}  \right\}$
\begin{solution} x=1; y=1 \end{solution}

\part[] $\left. \begin{gathered}
	  6x + 8y = 6 \hfill \\
	  7x - 5y = 7 \hfill \\ 
	\end{gathered}  \right\}$
\begin{solution} x=1; y=0 \end{solution}

\part[] $\left. \begin{gathered}
	  5x - 7y =  - 4 \hfill \\
	  3x + 5y = 16 \hfill \\ 
	\end{gathered}  \right\}$
\begin{solution} x=2; y=2 \end{solution}

\part[] $\left. \begin{gathered}
	  6x - 3y = \frac{7}{2} \hfill \\
	  5x - 2y = \frac{5}{3} \hfill \\ 
	\end{gathered}  \right\}$
\begin{solution} x=-2/3; y=-5/2 \end{solution}

\part[] $\left. \begin{gathered}
	  10x + 3y = 8 \hfill \\
	  15x + 12y = 22 \hfill \\ 
	\end{gathered}  \right\}$
\begin{solution} x=2/5; y=4/3 \end{solution}

\part[] $\left. \begin{gathered}
	  3x - 2y =  - 4 \hfill \\
	  4x - 5y = 1 \hfill \\ 
	\end{gathered}  \right\}$
\begin{solution} x=-22/7; y=-19/7 \end{solution}

\part[] $\left. \begin{gathered}
	  2x - 3y =  - 1 \hfill \\
	  3y + 2x =  - 9 \hfill \\ 
	\end{gathered}  \right\}$
\begin{solution} x=-5/2; y=-4/3 \end{solution}

\part[] $\left. \begin{gathered}
	  6x - 4y = 8 \hfill \\
	  9x - 6y = 12 \hfill \\ 
	\end{gathered}  \right\}$
\begin{solution} $x=(2y+4)/3 \to 6y+12-6y=12 \to 0y=0 \to \infty \ soluciones$ \end{solution}

\part[] $\left. \begin{gathered}
	  8x + 3y = 7 \hfill \\
	  24x = 2\left( {7 - 3y} \right) \hfill \\ 
	\end{gathered}  \right\}$
\begin{solution} x=0; y=7/3  \end{solution}

\part[] $\left. \begin{gathered}
	  3x - 4y = 8 \hfill \\
	  2x - 3y = 7 \hfill \\ 
	\end{gathered}  \right\}$
\begin{solution} x=-4; y=-5 \end{solution}

\part[] $\left. \begin{gathered}
	  0,3x + 0,2y = 1 \hfill \\
	  0,1x - 0,2y =  - 0,3 \hfill \\ 
	\end{gathered}  \right\}$
\begin{solution} x=7/4; y=19/8 \end{solution}

\part[] 	$\left. \begin{gathered}
	  \frac{{3x}}{4} - \frac{{2y}}{3} = 1 \hfill \\
	  \frac{{5x}}{2} + \frac{{4y}}{3} = 14 \hfill \\ 
	\end{gathered}  \right\}$

\begin{solution} x=4; y=3 \end{solution}

\part[] 	$\left. \begin{gathered}
	  \frac{{x + y}}{{x - y}} = \frac{7}{3} \hfill \\
	  \frac{{x + 4}}{{y + 4}} = \frac{{13}}{6} \hfill \\ 
	\end{gathered}  \right\}$
\begin{solution} x=-10; y=-4 \end{solution}

\part[] 	$\left. \begin{gathered}
	  x + y = 0 \hfill \\
	  x + y = 1 \hfill \\ 
	\end{gathered}  \right\}$
\begin{solution} Incompatible \end{solution}

\part[] $\left. \begin{gathered}
	  x + y = 0 \hfill \\
	  2x + 2y = 0 \hfill \\ 
	\end{gathered}  \right\}$
\begin{solution} $x=-y\to -2y + 2y =0 \to 0y=0 \to \infty\ soluciones$ \end{solution}

\part[] $\left. \begin{gathered}
	  \frac{x}{2} + \frac{y}{3} = 2 \hfill \\
	  \frac{y}{3} - \frac{x}{2} = 0 \hfill \\ 
	\end{gathered}  \right\}$
\begin{solution} x=2; y=3
 \end{solution}


\end{parts}
\end{multicols}

%\begin{multicols}{2}

\question Encuentra dos números tales que la suma de dos de los mismos sea 19 y la diferencia de ambos multiplicada por 6 sea 54.
\begin{solution} $\left\{\begin{matrix}x+y=19 \\ 6(x-y)=54\end{matrix}\right. \to  x = 14, \  y = 5 \to$ Números 14 y 5 \end{solution}

\question La sexta parte de la suma de dos números es 14 y la mitad de su diferencia es 13. Halla esos números.
\begin{solution} $\left\{\begin{matrix}\frac{1}{6}(x+y)=14 \\ \frac{1}{2}(x-y)=13\end{matrix}\right. \to  x = 55, \  y = 29 \to$ Números 55 y 29 \end{solution}

\question Un ganadero vende 7 cerdos y 9 corderos por 660 euros y luego vende 10 cerdos y 5 corderos por el mismo dinero. Calcula el precio de cada animal.
\begin{solution} $\left\{\begin{matrix}7x+9y=660 \\ 10x+5y=660\end{matrix}\right. \to  x = 48, \  y = 36 \to$ 48 cerdos y 36 corderos \end{solution}

\question Las dos cifras de un número suman 10 y la diferencia entre el número y el que resulta de invertir el orden de sus cifras es 36. Averigua de que número se trata.

\begin{solution}  $\left\{\begin{matrix}x+y=10 \\ 10x+y-(10y+x)=36\end{matrix}\right. \to  x = 7, \  y = 3 \to$ Número 73\end{solution}

\question La suma de las cifras de un número de tres cifras es 18. La cifra de las unidades es 8 y las de las centenas es dos unidades mayor que la de las decenas. Halla dicho número. 
\begin{solution} $\left\{\begin{matrix}x+y+8=18 \\ x=y+2\end{matrix}\right. \to  x = 6, \  y = 4 \to$ Número 648 \end{solution}

\question Un número de dos cifras es cuatro veces mayor que la suma de sus cifras. Si al número le sumamos 18 resulta un número que tiene las mismas cifras que el número dado, aunque en orden inverso. Busca el número inicial.
\begin{solution} $\left\{\begin{matrix}10x+y=4(x+y) \\ 10x+y+18=10y+x\end{matrix}\right. \to  x = 2, \  y = 4 \to$ Número 24 \end{solution}

\question Si a un número de dos cifras le quitamos el que resulta de invertir sus cifras da 27 y si le sumamos 9 unidades duplicamos el número que resulta de invertir el orden de sus cifras. Halla el número.
\begin{solution} $\left\{\begin{matrix}10x+y-(10y+x)=27 \\ 10x+y+9=2(10y+x)\end{matrix}\right. \to  x = 6, \  y = 3 \to$ Número 63 \end{solution}

\question Si a un número de dos cifras le sumamos 18 se obtiene un número con las cifras intercambiadas entre sí. Sabiendo que la suma de las cifras de ese número es 16, encuéntralo.
\begin{solution} $\left\{\begin{matrix}10x+y+18=10y+x \\ x+y=16\end{matrix}\right. \to  x = 7, \  y = 9 \to$ Número 79 \end{solution}

\question La suma de las cifras de un capicúa de la forma \emph{aba} es 19 y si a ese número le restamos el número de dos cifras \emph{ab} da por resultado 609. ¿cuál es el número capicúa?
\begin{solution} $\left\{\begin{matrix}x+y+x=19 \\ 100x+10y+x-(10x+y)=609\end{matrix}\right. \to  x = 6, \  y = 7 \to$ Número 676 \end{solution}

\question La cifra de las decenas de un número es triple que la de las unidades y el número disminuye en 36 cuando se invierte el orden de las cifras. Halla el número.
\begin{solution} $\left\{\begin{matrix}x=3y \\ 10x+y=10y+x+36\end{matrix}\right. \to  x = 6, \  y = 2 \to $ Número 62 \end{solution}

\question Un número capicúa es de la forma \emph{abba}. Intercambiamos los valores de sus cifras para obtener \emph{baab} y la diferencia entre ambos es 8019. Si se sabe que la suma de sus cifras es 18, ¿cuál es el número?
\begin{solution} $\left\{\begin{matrix}x+y+y+x=18 \\ 1000x+100y+10y+x-(1000y+100x+10x+y)=8019\end{matrix}\right. \to  x = 9, \  y = 0 \to$ Número 9009 \end{solution}

\question Juan le dice a Luis: Actualmente mi edad es triple que la tuya, pero hace siete años era diez veces mayor que tú. ¿Qué edad tiene cada uno?

\begin{solution} $\left\{\begin{matrix}x=3y \\ x-7=10(y-7)\end{matrix}\right. \to  x = 27, \  y = 9 \to$ Juan 27 años y Luis 9 años \end{solution}

\question Si a cada uno de los términos de una fracción le sumamos 3 resulta una fracción equivalente a 10/11, pero si les restásemos 4 resultaría equivalente a 3/4. Halla la fracción.
\begin{solution} $\left\{\begin{matrix}(x+3)/(y+3)=10/11 \\ (x-4)/(y-4)=3/4\end{matrix}\right. \to  x = 7, \  y = 8 \to$ Fracción 7/8 \end{solution}

\question Una embarcación va a favor de la corriente de un río a 20 km/h y en contra de la corriente a 14 km/h. ¿A qué velocidad descenderá un trozo de madera por el río?

\begin{solution} $\left\{\begin{matrix}x+y=20 \\ x-y=14\end{matrix}\right. \to  x = 17, \  y = 3 \to$ Velocidad del tronco 3 km/h \end{solution}

\question Una persona lleva en el monedero 50 monedas diversas, de uno y de cinco euros, por un valor de 190 euros. ¿Cuántas monedas lleva de un euro?
\begin{solution} $\left\{\begin{matrix}x+y=50 \\ x+5y=190\end{matrix}\right. \to  x = 15, \  y = 35 |to$ 15 monedas de un euro y 35 monedas de cinco euros \end{solution}

\question Determina una fracción tal que si le sumamos una unidad al numerador se transforma en una fracción equivalente a 1/2 y si aumentásemos en dos unidades el denominador se transformaría en otra equivalente a 1/3.

\begin{solution} $\left\{\begin{matrix}(x+1)/y=1/2 \\ x/(y+2)=1/3\end{matrix}\right. \to  x = 4, \  y = 10 \to$ Fracción 4/10 \end{solution}

\question En una hucha hay 55 monedas de cinco y dos euros. Si en total hay 212 euros. ¿Cuántas monedas hay de cada clase?
\begin{solution} $\left\{\begin{matrix}x+y=55 \\ 5x+2y=212\end{matrix}\right. \to  x = 34, \  y = 21 \to$ 34 monedas de cinco euros y 21 monedas de dos euros \end{solution}

\question Halla una fracción equivalente a 3/5 cuya suma de sus términos sea 32.
\begin{solution} $\left\{\begin{matrix}x/y=3/5 \\ x+y=32\end{matrix}\right. \to  x = 12, \  y = 20 \to$ Fracción 12/20 \end{solution}

\question Con dos clases de café de 5,4 euros/kg y 7,2 euros/kg se quiere obtener una mezcla cuyo precio resulte a 6 euros/kg. Calcula la cantidad que hay que poner de cada uno para lograr 600 kg de mezcla.

\begin{solution} \to$\left\{\begin{matrix}5.4x+7.2y=6 \cdot 600 \\ x+y=600\end{matrix}\right. \to  x = 400.0, \  y = 200.0 $ 400 kg de 5,4 euros/kg y 200 kg de 7,2 euros/kg \end{solution}

\question En un corral hay conejos y gallinas en total hay 59 cabezas y 202 patas. ¿cuántos conejos y gallinas hay?
\begin{solution} $\left\{\begin{matrix}x+y=59 \\ 4x+2y=202\end{matrix}\right. \to  x = 42, \  y = 17 \to$ 42 conejos y 17 gallinas \end{solution}

\question En las anotaciones de un camarero se podía leer: \\
	\emph{Mesa 10: 2 cafés y 4 zumos 5,2 euros.\\
	Mesa 15: 3 cafés y 2 zumos 4,2 euros.}\\
¿Cuánto valían el café y el zumo en ese bar?
\begin{solution} $\left\{\begin{matrix}2x+4y=5.2 \\ 3x+2y=4.2\end{matrix}\right. \to  x = 0.8, \  y = 0.9 \to $ Café 0,8 euros y zumo 0,9 euros \end{solution}

\question De acuerdo con las previsiones, entre las dos fábricas de una misma empresa deberían producir 360 máquinas al mes. La primera de ellas cumplió el plan previsto al 112\% y la segunda al 110\% y entre ambas produjeron un total de 400 máquinas. ¿Cuántas máquinas produjo cada una por separado?
\begin{solution} $\left\{\begin{matrix}x+y=360 \\ 1.12x+1.1y=400\end{matrix}\right. \to  x = 200.0, \  y = 160.0 \to $
 224 y 176 máquinas \end{solution}

\question Una persona tiene una bañera de 492 litros. Si quiere llenar a rebosar la bañera, con ella completamente sumergida, debe echar 35 cubos de agua pero si la persona tuviera doble volumen harían falta cinco cubos menos. ¿Cuál es el volumen de la persona y la capacidad del cubo?
\begin{solution} $\left\{\begin{matrix}492-x=35y \\ 492-2x=30y\end{matrix}\right. \to  x = \frac{123}{2}, \  y = \frac{123}{10} \to $ Cubo 12,3 litros y la persona 61,5 litros \end{solution}

\question Hace cinco años Pedro tenía triple edad que Jesús y dentro de un año sólo será el doble. ¿Cuáles son las edades de ambos en la actualidad?
\begin{solution}$\left\{\begin{matrix}x-5=3(y-5) \\ x+1=2(y+1)\end{matrix}\right. \to  x = 23, \  y = 11 \to $ Pedro 23 años y Jesús 11 años \end{solution}

\question Halla una fracción equivalente a 3/8 cuyo numerador más denominador sume 55.

\begin{solution} $\left\{\begin{matrix}x/y=3/8 \\ x+y=55\end{matrix}\right. \to  x = 15, \  y = 40 \to $ Fracción 15/40 \end{solution}

\question El área de un rectángulo no variaría si se aumentase su base en 6 cm y a la vez se disminuyese su altura en 3 cm. Tampoco variaría si la base disminuyese en 4 cm y la altura aumentase en 3 cm. ¿Cuáles son las dimensiones actuales del rectángulo?
\begin{solution} $\left\{\begin{matrix}xy=(x+6)(y-3) \\ xy=(x-4)(y+3)\end{matrix}\right. \to  x = 24, \  y = 15 \to $ Base 24 cm y altura 15 cm \end{solution}

\question Las dos cifras de un número suman 6. Ese número y el que resulta de invertir el orden de sus cifras están en la relación 4:7. Hállalo.
\begin{solution}  $\left\{\begin{matrix}x+y=6 \\ (10x+y)/(10y+x)=4/7\end{matrix}\right. \to  x = 2, \  y = 4 \to $ Número 24 \end{solution}

\question Dos pueblos A y B están situados en lados opuestos de un puerto de montaña. Un ciclista que sube a 12 km/h y desciende a 36 km/h emplea 45 minutos en ir de A a B; en cambio, el viaje de regreso le lleva diez minutos más. ¿Qué distancia, por carretera, separa a y B?
\begin{solution} $\left\{\begin{matrix}x/12+y/36=45/60 \\ y/12+x/36=55/60\end{matrix}\right. \to  x = 6, \  y = 9 \to $ 15 kilómetros \end{solution}

\question En otro puerto de montaña también hay dos pueblos situados a lados distintos y, en este caso, distantes 18 km por carretera. Un ciclista, que sube a 12 km/h y desciende a 30 km/h, emplea una hora en ir de uno al otro. ¿Cuántos kilómetros tiene de subida y bajada?

\begin{solution} $\left\{\begin{matrix}x+y=18 \\ x/12+y/30=1\end{matrix}\right. \to  x = 8, \  y = 10 \to $ Subida 8 km y bajada 10 km \end{solution}

\question Dos capitales son tales que colocados el mayor al 5\% y el menor al 6\% se obtiene una renta anual de 930 euros, pero si se intercambiasen los intereses la renta sería de 940 euros. Halla ambos capitales.
\begin{solution} $\left\{\begin{matrix}0.05x+0.06y=930 \\ 0.06x+0.05y=940\end{matrix}\right. \to  x = 9000.0, \  y = 8000.0 \to $ Mayor 9.000 euros y menor 8.000 euros \end{solution}

\begin{comment}
\question Si se aumenta la base de un rectángulo en 4 cm y se disminuye la altura en 2 cm se tiene la misma área; en cambio, si la base se disminuye en 10 cm y se aumenta la altura en 10 cm, entonces el área es 40 cm2 menor. Averigua las dimensiones del rectángulo.
\begin{solution} $\left\{\begin{matrix}(x+4)(y-2)=xy \\ (x-10)(y+10)=xy-40\end{matrix}\right. \to  x = 16, \  y = 10 \to $ Base 16 cm y altura 10 cm \end{solution}

\question Susana debe pagar dos facturas que importan un total de 1050 euros. Después de mucho regatear consigue en la primera un descuento del 12\% y en la segunda otro del 8\%, resultando que sólo paga 948 euros. ¿Cuál era el importe de cada factura?

\begin{solution} $\left\{\begin{matrix}x+y=1050 \\ 0.88x+0.92y=948\end{matrix}\right. \to  x = 450.0, \  y = 600.0 \to $ Primera 450 euros y segunda 600 euros \end{solution}

\question Se reparten, de forma desigual, 800 litros de vino en dos barricas del mismo tamaño. La primera se llenaría si le echásemos la tercera parte de lo que contiene la segunda. A su vez, la segunda barrica se llenaría si le echáramos la séptima parte de los litros que contiene la primera. Halla la capacidad de las barricas y lo que contiene cada una de ellas.
\begin{solution} $\left\{\begin{matrix}x+y/3=z \\ y+x/7=z \\ x+y=800\end{matrix}\right. \to  x = 350, \  y = 450, \  z = 500 \to $ Primera 350 litros y segunda 450 litros (500 las barricas) \end{solution}

\question En un colegio hay 600 estudiantes y han salido de viaje 280, siendo el 60\% chicos y el 40\% chicas. ¿Cuántas chicas y chicos hay en el colegio?

\begin{solution} $\left\{\begin{matrix}x+y=600 \\ 0.6x+0.4y=280\end{matrix}\right. \to  x = 200.0, \  y = 400.0 \to $ 200 chicos y 400 chicas \end{solution}

\question Un total de 80 vasos están distribuidos entre dos cajas, A y B. Si pasásemos diez vasos de B a A el número de vasos de A sería tres veces  el de los almacenados en B. ¿Cuántos vasos hay en cada caja?
\begin{solution} $\left\{\begin{matrix}x+y=80 \\ x+10=3(y-10)\end{matrix}\right. \to  x = 50, \  y = 30 \to $ 50 vasos en A y 30 vasos en B \end{solution}

\question Halla el precio de coste y venta de un par de zapatillas de deporte sabiendo que si sobre el PVP el comerciante hiciese un descuento del 20\% todavía ganaría 200 céntimos de euro y, en cambio, perdería 250 céntimos de euro si el descuento fuese del 30\% sobre el PVP.
\begin{solution} $\left\{\begin{matrix}0.8y-x=2 \\ 0.7y-x=-2.5\end{matrix}\right. \to  x = 34.0, \  y = 45.0 \to $ Precio de coste 34 euros y precio de venta 45 euros \end{solution}

\question La edad de Luis es actualmente tres veces la edad de Ana. Dentro de 5 años la edad de Luis será solamente doble que la edad de Ana. Halla las edades actuales de ambos.

\begin{solution} $\left\{\begin{matrix}x=3y \\ x+5=2(y+5)\end{matrix}\right. \to  x = 15, \  y = 5 \to $ Luis 15 años y Ana 5 años \end{solution}

\question En una fábrica han mezclado harina de trigo a 1,20 \euro /kg, y harina de maíz a 1,00 \euro /kg y van a obtener por la venta de a mezcla un total de 290 \euro. Halla qué cantidad de harina de cada cereal han usado sabiendo que si el precio de ambas harinas fuese 0,20 \euro /kg mayor el valor de la mezcla sería 344 \euro.

\begin{solution} $\left\{\begin{matrix}1.2x+y=290 \\ 1.4x+1.2y=344\end{matrix}\right. \to  x = 100.0, \  y = 170.0 \to $ 100 kg de harina de trigo y 170 kg de harina de maíz \end{solution}

\question Una peña de amigos va a comer a un restaurante. A la hora de pagar se dan cuenta que si cada uno pone 30 euros faltan 20 euros para el total, mientras que si ponen 35 euros por cabeza sobran un total de 40 euros
	¿Cuántos amigos componen la peña?
	¿Qué cantidad tienen que pagar en total?
\begin{solution} $\left\{\begin{matrix}30x=y-20 \\ 35x=y+40\end{matrix}\right. \to  x = 12, \  y = 380 \to $ 12 amigos y
	380 euros
 \end{solution}
\end{comment}

%\end{multicols}
\end{questions}
\end{document}


