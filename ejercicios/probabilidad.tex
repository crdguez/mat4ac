\documentclass[spanish, 10pt]{exam}

%These tell TeX which packages to use.
\usepackage{array,epsfig}
\usepackage{amsmath, textcomp}
\usepackage{amsfonts}
\usepackage{amssymb}
\usepackage{amsxtra}
\usepackage{amsthm}
\usepackage{mathrsfs}
\usepackage{color}
\usepackage{multicol}
\usepackage{verbatim}
\usepackage{svg}

\usepackage{pgf,tikz}
\usetikzlibrary{shapes, calc, shapes, arrows, math, babel, trees}



\usepackage[utf8]{inputenc}
\usepackage[spanish]{babel}
\usepackage{eurosym}

\usepackage{graphicx}
\graphicspath{{../img/}}



%\printanswers
\nopointsinmargin
\pointformat{}

%Pagination stuff.
%\setlength{\topmargin}{-.3 in}
%\setlength{\oddsidemargin}{0in}
%\setlength{\evensidemargin}{0in}
%\setlength{\textheight}{9.in}
%\setlength{\textwidth}{6.5in}
%\pagestyle{empty}

\renewcommand{\solutiontitle}{\noindent\textbf{Sol:}\enspace}

\newcommand{\samedir}{\mathbin{\!/\mkern-5mu/\!}}

\newcommand{\class}{4º Académicas}
\newcommand{\examdate}{\today}
\newcommand{\examnum}{Probabilidad}
\newcommand{\tipo}{A}


\newcommand{\timelimit}{50 minutos}

\newcommand\xa{3} %tamaño ejes par tikz

\pagestyle{head}
\firstpageheader{\includegraphics[width=0.2\columnwidth]{header_left}}{\textbf{Departamento de Matemáticas\linebreak \class}\linebreak \examnum}{\includegraphics[width=0.1\columnwidth]{header_right}}
\runningheader{\class}{\examnum}{Página \thepage\ of \numpages}
\runningheadrule

\begin{document}
 
\begin{questions}

\question Se lanza un dado en forma de dodecaedro regular, cuyas caras
están numeradas del 1 al 12. Si A es el suceso "salir múltiplo
de 3"; B, "salir un número primo"; y C, "salir un número mayor
que 5":
\begin{parts}
\part calcula A, B y C.
\begin{solution}
$A=\lbrace3,6,9,12\rbrace$, $B=\lbrace2,3,5,7,11\rbrace$, $C=\lbrace6,7,8,9,10,11,12\rbrace$
\end{solution}
\part halla $\overline{A}$ , $\overline{B}$ y $\overline{C}$.
\begin{solution}
$\overline{A}= \lbrace 1,2,4,5,7,8,10,11 \rbrace$ , $\overline{B}=\lbrace 1,4,6,8,9,10,12 \rbrace$ y $\overline{C}= \lbrace 1,2,3,4,5 \rbrace$ 
\end{solution}
\part calcula $A\cup B$, $A\cap B$, $A\cup C$, $A\cap C$, $B\cup C$, $B\cap C$.
\begin{solution}
$A\cup B=\lbrace2,3,5,6,7,9,11,12\rbrace$, $A\cap B=\lbrace3\rbrace$, $A\cup C=\lbrace3,6,7,8,9,10,11,12\rbrace$, $A\cap C=\lbrace6,9,12\rbrace$, $B\cup C=\lbrace2,3,5,6,7,8,9,10,11,12\rbrace$, $B\cap C=\lbrace7,11\rbrace$
\end{solution}
\end{parts}

\question Cuál es el suceso contrario de:
\begin{parts}
\part salir par
\begin{solution} No salir par (=salir impar)
\end{solution}
\part salir un número primo
\begin{solution} no salir un número primo
\end{solution}
\part salir oros (al extraer una carta)
\begin{solution} no salir oros
\end{solution}
\part salir al menos una carta de oros (al extraer cuatro cartas)
\begin{solution}no salir al menos una carta de oros (=no
salir oros)
\end{solution}
\part salir las cuatro cartas de oros (al extraer cuatro cartas)
\begin{solution}no salir las cuatro cartas de oros (=salir al
menos una carta que no sea de oros)
\end{solution}
\end{parts}


\question De una baraja de 40 cartas se extraen dos sin remplazamiento. Halla la probabilidad:
\begin{parts}
\part de que sean el as de oros y el as de copas
\begin{solution}
$\frac{V_2^2}{V_{40}^2}=\frac{2}{40\cdot39}=\frac{1}{780}$
\end{solution}
\part de que sean dos figuras (sota, caballo o rey)
\begin{solution}
$  \frac{V_{12}^2}{V_{40}^2}=\frac{12\cdot11}{40\cdot39}=\frac{11}{130}$
\end{solution}
\part de que al menos una sea de oros
\begin{solution}
$1-\frac{V_{30}^{2}}{V_{40}^{2}}=1-\frac{30\cdot29}{40\cdot39}=\frac{23}{52}$
\end{solution}
\part de que sean dos reyes
\begin{solution}
$\frac{V_{4}^2}{V_{40}^{2}}=\frac{4\cdot3}{40\cdot39}=\frac{1}{130}$
\end{solution}
\part de que sean del mismo palo
\begin{solution}
$\frac{V_{4}^1 \cdot V_{10}^2}{V_{40}^{2}}=\frac{4\cdot10\cdot9}{40\cdot39}=\frac{3}{13}$
\end{solution}
\part de que sean un rey y un caballo. 
\begin{solution}
$\frac{V_{8}^1 \cdot V_{4}^1}{V_{40}^{2}}=\frac{8\cdot4}{40\cdot39}=\frac{4}{195}$
\end{solution}
\end{parts}

\question De una baraja de 40 cartas se extraen dos con remplazamiento. Halla la probabilidad:
\begin{parts}
\part de que sean el as de oros y el as de copas
\begin{solution}
$\frac{V\\
_2^2}{VR_{40}^2}=\frac{2}{40\cdot40}=\frac{1}{800}$
\end{solution}
\part de que sean dos figuras (sota, caballo o rey)
\begin{solution}
$  \frac{VR_{12}^2}{VR_{40}^2}=\frac{12\cdot12}{40\cdot40}=\frac{9}{100}$
\end{solution}
\part de que al menos una sea de oros
\begin{solution}
$1-\frac{VR_{30}^{2}}{VR_{40}^{2}}=1-\frac{30\cdot30}{40\cdot40}=\frac{7}{16}$
\end{solution}
\part de que sean dos reyes
\begin{solution}
$\frac{VR_{4}^2}{VR_{40}^{2}}=\frac{4\cdot4}{40\cdot40}=\frac{1}{100}$
\end{solution}
\part de que sean del mismo palo
\begin{solution}
$\frac{VR_{4}^1 \cdot VR_{10}^2}{VR_{40}^{2}}=\frac{4\cdot10\cdot10}{40\cdot40}=\frac{1}{4}$
\end{solution}
\part de que sean un rey y un caballo. 
\begin{solution}
$\frac{VR_{8}^1 \cdot VR_{4}^1}{VR_{40}^{2}}=\frac{8\cdot4}{40\cdot40}=\frac{1}{50}$
\end{solution}
\end{parts}





\question Al tirar 8 monedas, ¿cuál es la probabilidad de que salgan 6 caras y 2 cruces?

\begin{solution}
$\frac{C_8^2}{VR_8^2}=\frac{8!}{6!\cdot2!\cdot2^8}=\frac{7}{64}$
\end{solution}


\question En una urna hay dos bolas blancas y una negra. Se extraen dos bolas \textbf{con} reemplazamiento. Cuál es la probabilidad de que sean: 

\begin{parts}
\part de distinto color
\begin{solution}
\tikzstyle{bag} = [text width=4em, text centered]
\tikzstyle{end} = [circle, minimum width=3pt,fill, inner sep=0pt]
\tikzstyle{level 1} = [level distance=3.5cm, sibling distance=3.5cm]
\tikzstyle{level 2} = [level distance=3.5cm, sibling distance=2cm]

\begin{tikzpicture}[grow=right, sloped, scale=1]
\node[bag] {$2B, 1N$}
    child {
        node[bag] {$2B, 1N$}        
            child {
                node[end, label=right:
                    {$P(N_1\cap N_2)=\frac{1}{3}\cdot\frac{1}{3}$}] {}
                edge from parent
                node[above] {$N$}
                node[below]  {$\frac{1}{3}$}
            }
            child {
                node[end, label=right:
                    {$P(N_1\cap B_2)=\frac{1}{3}\cdot\frac{2}{3}$}] {}
                edge from parent
                node[above] {$B$}
                node[below] {$\frac{2}{3}$}
            }
            edge from parent 
            node[above] {$N$}
            node[below]  {$\frac{1}{3}$}
    }
    child {
        node[bag] {$2B, 1N$}        
        child {
                node[end, label=right:
                    {$P(B_1\cap N_2)=\frac{2}{3}\cdot\frac{1}{3}$}] {}
                edge from parent
                node[above] {$N$}
                node[below]  {$\frac{1}{3}$}
            }
            child {
                node[end, label=right:
                    {$P(B_1\cap B_2)=\frac{2}{3}\cdot\frac{2}{3}$}] {}
                edge from parent
                node[above] {$B$}
                node[below]  {$\frac{2}{3}$}
            }
        edge from parent         
            node[above] {$B$}
            node[below]  {$\frac{2}{3}$}
    };
\end{tikzpicture} \\
$ P(Distinto\ color) = P(B_1\cap N_2) + P(N_1\cap B_2) = \frac{2}{9}+\frac{2}{9}= \frac{4}{9}$
\end{solution}

\part del mismo color
\begin{solution}
$P(Mismo\ color) = P(B_1\cap B_2) + P(N_1\cap N_2) = \frac{4}{9}+\frac{1}{9}= \frac{5}{9}$ 
\end{solution}

\part Cuál es la probabilidad de que, habiendo sido la segunda bola blanca, la primera haya sido blanca:
\begin{solution}
$P(B_1|B_2)=\dfrac{P(B_1\cap B_2)}{P(B_2)}=\dfrac{\frac{2}{3}\cdot\frac{2}{3}}{\frac{2}{3}\cdot\frac{2}{3}+\frac{1}{3}\cdot\frac{2}{3}}=\dfrac{\frac{4}{9}}{\frac{6}{9}}=\frac{2}{3} $
\end{solution}

\part Cuál es la probabilidad de que, habiendo sido la segunda bola blanca, la primera haya sido negra:
\begin{solution}
$P(N_1|B_2)=\dfrac{P(N_1\cap B_2)}{P(B_2)}=\dfrac{\frac{1}{3}\cdot\frac{2}{3}}{\frac{2}{3}\cdot\frac{2}{3}+\frac{1}{3}\cdot\frac{2}{3}}=\dfrac{\frac{2}{9}}{\frac{6}{9}}=\frac{1}{3} $
\end{solution}


\end{parts}


\question En una urna hay dos bolas blancas y una negra. Se extraen dos bolas \textbf{sin} reemplazamiento. Cuál es la probabilidad de que sean: 

\begin{parts}
\part de distinto color
\begin{solution}
\tikzstyle{bag} = [text width=4em, text centered]
\tikzstyle{end} = [circle, minimum width=3pt,fill, inner sep=0pt]
\tikzstyle{level 1} = [level distance=3.5cm, sibling distance=3.5cm]
\tikzstyle{level 2} = [level distance=3.5cm, sibling distance=2cm]

\begin{tikzpicture}[grow=right, sloped, scale=1]
\node[bag] {$2B, 1N$}
    child {
        node[bag] {$2B, 0N$}        
            child {
                node[end, label=right:
                    {$P(N_1\cap N_2)=\frac{1}{3}\cdot\frac{0}{3}=0$}] {}
                edge from parent
                node[above] {$N$}
                node[below]  {$\frac{0}{2}$}
            }
            child {
                node[end, label=right:
                    {$P(N_1\cap B_2)=\frac{1}{3}\cdot\frac{2}{2}$}] {}
                edge from parent
                node[above] {$B$}
                node[below] {$\frac{2}{2}$}
            }
            edge from parent 
            node[above] {$N$}
            node[below]  {$\frac{1}{3}$}
    }
    child {
        node[bag] {$1B, 1N$}        
        child {
                node[end, label=right:
                    {$P(B_1\cap N_2)=\frac{2}{3}\cdot\frac{1}{2}$}] {}
                edge from parent
                node[above] {$N$}
                node[below]  {$\frac{1}{2}$}
            }
            child {
                node[end, label=right:
                    {$P(B_1\cap B_2)=\frac{2}{3}\cdot\frac{1}{2}$}] {}
                edge from parent
                node[above] {$B$}
                node[below]  {$\frac{1}{2}$}
            }
        edge from parent         
            node[above] {$B$}
            node[below]  {$\frac{2}{3}$}
    };
\end{tikzpicture} \\
$ P(Distinto\ color) = P(B_1\cap N_2) + P(N_1\cap B_2) = \frac{1}{3}+\frac{1}{3}= \frac{2}{3}$
\end{solution}

\part del mismo color
\begin{solution}
$P(Mismo\ color) = P(B_1\cap B_2) + P(N_1\cap N_2) = \frac{1}{3}+\frac{0}{9}= \frac{1}{3}$ 
\end{solution}

\part Cuál es la probabilidad de que, habiendo sido la segunda bola blanca, la primera haya sido blanca:
\begin{solution}
$P(B_1|B_2)=\dfrac{P(B_1\cap B_2)}{P(B_2)}=\dfrac{\frac{2}{3}\cdot\frac{1}{2}}{\frac{2}{3}\cdot\frac{1}{2}+\frac{1}{3}\cdot\frac{2}{2}}=\dfrac{\frac{1}{3}}{\frac{4}{6}}=\frac{1}{2} $
\end{solution}

\part Cuál es la probabilidad de que, habiendo sido la segunda bola blanca, la primera haya sido negra:
\begin{solution}
$P(N_1|B_2)=\dfrac{P(N_1\cap B_2)}{P(B_2)}=\dfrac{\frac{1}{3}\cdot\frac{2}{2}}{\frac{2}{3}\cdot\frac{1}{2}+\frac{1}{3}\cdot\frac{2}{2}}=\dfrac{\frac{1}{3}}{\frac{4}{6}}=\frac{1}{2} $
\end{solution}

\end{parts}




\begin{comment}
\question 
\begin{multicols}{3}
\begin{parts}
\part[]  
\begin{solution} \end{solution}
\end{parts}
\end{multicols}
\end{comment}


\end{questions}
\end{document}

