\documentclass[spanish, 11pt]{exam}

%These tell TeX which packages to use.
\usepackage{array,epsfig}
\usepackage{amsmath}
\usepackage{amsfonts}
\usepackage{amssymb}
\usepackage{amsxtra}
\usepackage{amsthm}
\usepackage{mathrsfs}
\usepackage{color}
\usepackage{multicol}
\usepackage{verbatim}

\usepackage[utf8]{inputenc}
\usepackage[spanish]{babel}
\usepackage{eurosym}

\usepackage{graphicx}
\graphicspath{{../img/}}


%\printanswers
\nopointsinmargin
\pointformat{}

%Pagination stuff.
%\setlength{\topmargin}{-.3 in}
%\setlength{\oddsidemargin}{0in}
%\setlength{\evensidemargin}{0in}
%\setlength{\textheight}{9.in}
%\setlength{\textwidth}{6.5in}
%\pagestyle{empty}

\renewcommand{\solutiontitle}{\noindent\textbf{Sol:}\enspace}

\newcommand{\class}{4º Académicas}
\newcommand{\examdate}{\today}
\newcommand{\examnum}{Trigonometría 2}
\newcommand{\tipo}{A}


\newcommand{\timelimit}{50 minutos}



\pagestyle{head}
\firstpageheader{\includegraphics[width=0.2\columnwidth]{header_left}}{\textbf{Departamento de Matemáticas\linebreak \class}\linebreak \examnum}{\includegraphics[width=0.1\columnwidth]{header_right}}
\runningheader{\class}{\examnum}{Página \thepage\ of \numpages}
\runningheadrule

\begin{document}



\begin{questions}


\question \textbf{Teoría:} Reducción al primer cuadrante
\begin{multicols}{3}
\begin{parts}
\part[]Ángulos complementarios (suman 90º) \\
\[\begin{matrix}
   sen\text{ }\left( 90{}^\text{o}\text{ - }\alpha  \right)\ =\text{ cos }\alpha   \\
   \cos \text{ }\left( \text{90 }\!\!{}^\text{o}\!\!\text{  - }\alpha  \right)\text{ }=\text{ sen }\alpha   \\
   tg\text{ }\left( \text{90 }\!\!{}^\text{o}\!\!\text{  - }\alpha  \right)\text{ }=\text{ cotg }\alpha   \\
\end{matrix}\]

\part[]Ángulos que difieren 90º  \\
\[\begin{matrix}
   sen\text{ }\left( \text{90 }\!\!{}^\text{o}\!\!\text{  }+\text{ }\alpha  \right)\ =\text{ cos }\alpha   \\
   \cos \text{ }\left( \text{90 }\!\!{}^\text{o}\!\!\text{  }+\text{ }\alpha  \right)\text{ }=\text{  -sen }\alpha   \\
   tg\text{ }\left( \text{90 }\!\!{}^\text{o}\!\!\text{  }+\text{ }\alpha  \right)\text{ }=\text{  -cotg }\alpha   \\
\end{matrix}\]

\part[]Ángulos suplementarios (suman 180º)  \\
\[\begin{matrix}
   sen\text{ }\left( \text{180 }\!\!{}^\text{o}\!\!\text{  - }\alpha  \right)\ =\text{ sen }\alpha   \\
   \cos \text{ }\left( \text{180 }\!\!{}^\text{o}\!\!\text{  - }\alpha  \right)\text{ }=\text{ -cos }\alpha   \\
   tg\text{ }\left( \text{180 }\!\!{}^\text{o}\!\!\text{  - }\alpha  \right)\text{ }=\text{ -tg }\alpha   \\
\end{matrix}\]

\part[]Ángulos que difieren 90º  \\
\[\begin{matrix}
   sen\text{ }\left( \text{180 }\!\!{}^\text{o}\!\!\text{  }+\text{ }\alpha  \right)\ =\text{ -sen }\alpha   \\
   \cos \text{ }\left( \text{180 }\!\!{}^\text{o}\!\!\text{  }+\text{ }\alpha  \right)\text{ }=\text{ -cos }\alpha   \\
   tg\text{ }\left( \text{180 }\!\!{}^\text{o}\!\!\text{  }+\text{ }\alpha  \right)\text{ }=\text{ tg }\alpha   \\
\end{matrix}\]

\part[]Ángulos cuya suma es 360º  \\
\[\begin{matrix}
   sen\text{ }\left( 360{}^\text{o}\text{ - }\alpha  \right)\ =\text{ -sen }\alpha   \\
   \cos \text{ }\left( 360{}^\text{o}\text{ - }\alpha  \right)\text{ }=\text{ cos }\alpha   \\
   tg\text{ }\left( 360{}^\text{o}\text{ - }\alpha  \right)\text{ }=\text{ -tg }\alpha   \\
\end{matrix}\]

\end{parts}
\end{multicols}

\question Demostrar que para cualquier ángulo $\alpha $, se verifica: $\sec^2{\alpha}+\cosec^2{\alpha}=\sec^2{\alpha}\cdot\cosec^2{\alpha}$


\question Demostrar si son verdaderas o falsas las siguientes ecuaciones:

\begin{multicols}{2}
\begin{parts}
\part[] $\dfrac{{{\text{tg }}\alpha {\text{ }} + {\text{ tg }}\beta }}{{{\text{cotg }}\alpha {\text{ }} + {\text{ cotg }}\beta }}{\text{ }} = {\text{ tg }}\alpha {\text{ }}{\text{. tg }}\beta   $
\begin{solution} \end{solution}

\part[] $\dfrac{{{\text{sen }}\alpha {\text{ }}{\text{. cos }}\alpha }}{{{\text{co}}{{\text{s}}^{\text{2}}}\alpha {\text{  -  se}}{{\text{n}}^{\text{2}}}\alpha }}{\text{ }} = {\text{ }}\dfrac{{{\text{tg }}\alpha }}{{{\text{1  -  t}}{{\text{g}}^2}\alpha }}   $
\begin{solution} \end{solution}
\part[] $\cotg \alpha {\text{  -  }}\dfrac{{{\text{cot}}{{\text{g}}^{\text{2}}}\alpha {\text{  -  1}}}}{{{\text{cotg }}\alpha }}{\text{ }} = {\text{ tg }}\alpha$
\begin{solution} \end{solution}
\part[] $\dfrac{{{\text{ sen }}\alpha {\text{ }} + {\text{ cotg }}\alpha }}{{{\text{tg }}\alpha {\text{ }} + {\text{ cosec }}\alpha }}{\text{ }} = {\text{ cos }}\alpha$
\begin{solution} \end{solution}
\part[] $\cotg^2\alpha {\text{  -  }}{\cos ^2}\alpha {\text{ }} = {\text{ }}\cotg^2\alpha {\text{ }}{\text{. }}{\cos ^2}\alpha$
\begin{solution} \end{solution}
\part[] $\sen\alpha {\text{ }}{\text{. cos }}\alpha {\text{ }}{\text{. tg }}\alpha {\text{ }}{\text{. cotg}}\alpha {\text{ }}{\text{. sec }}\alpha {\text{ }}{\text{. cosec }}\alpha {\text{ }} = {\text{ 1}}$
\begin{solution} \end{solution}
\part[] $\dfrac{1 + \tg\alpha }{1-\tg\alpha } = \dfrac{\cos\alpha+\sen\alpha}{\cos\alpha-\sen\alpha}$
\begin{solution} \end{solution}
\part[] $\dfrac{{{\text{1 }} + {\text{ t}}{{\text{g}}^{\text{2}}}\alpha }}{{\cot g{\text{ }}\alpha }}{\text{ }} = {\text{ }}\dfrac{{{\text{tg }}\alpha }}{{{\text{co}}{{\text{s}}^{\text{2}}}\alpha }}$
\begin{solution} \end{solution}


\end{parts}
\end{multicols}

\question Simplificar las siguientes expresiones:

\begin{multicols}{2}
\begin{parts}
\part[] $sen{\text{ }}\alpha {\text{ }}{\text{. }}\frac{{\text{1}}}{{{\text{tg }}\alpha }} $
\begin{solution} \end{solution}
\part[] $se{n^3}\alpha  + {\text{ sen }}\alpha {\text{ }}{\text{. co}}{{\text{s}}^{\text{2}}}\alpha$
\begin{solution} \end{solution}
\part[] $\sqrt {1{\text{  -  sen }}\alpha } {\text{ }}{\text{. }}\sqrt {{\text{1 }} + {\text{ sen }}\alpha }$
\begin{solution} \end{solution}
\part[] $ se{n^4}\alpha {\text{  -  co}}{{\text{s}}^{\text{4}}}\alpha$
\begin{solution} \end{solution}
\part[] ${\cos ^3}\alpha {\text{ }} + {\text{ co}}{{\text{s}}^{\text{2}}}\alpha {\text{ }}{\text{. sen }}\alpha {\text{ }} + {\text{ cos }}\alpha {\text{ }}{\text{. se}}{{\text{n}}^{\text{2}}}\alpha {\text{ }} + {\text{ se}}{{\text{n}}^{\text{3}}}\alpha $
\begin{solution} \end{solution}
\part[] $sen{\text{ }}\alpha {\text{ }}{\text{. cos }}\alpha {\text{ }}\left( {{\text{tg }}\alpha {\text{ }} + {\text{ }}\dfrac{{\text{1}}}{{{\text{tg }}\alpha }}} \right) $
\begin{solution} \end{solution}
\part[] $ \dfrac{{{{\cos }^2}\alpha {\text{  -  se}}{{\text{n}}^{\text{2}}}\alpha }}{{{{\cos }^4}\alpha {\text{  -  se}}{{\text{n}}^{\text{4}}}\alpha }}$
\begin{solution} \end{solution}
\part[] $\dfrac{{{{\sec }^2}\alpha {\text{ }} + {\text{ co}}{{\text{s}}^{\text{2}}}\alpha }}{{{{\sec }^2}\alpha {\text{  -  co}}{{\text{s}}^{\text{2}}}\alpha }} $
\begin{solution} \end{solution}
\part[] $ \dfrac{{{{\cos }^2}\alpha }}{{{\text{1  -  sen }}\alpha }}$
\begin{solution} \end{solution}
\part[] $\dfrac{{\cos ec{\text{ }}\alpha }}{{1{\text{ }} + {\text{ }}\cot {g^2}\alpha {\text{ }}}} $
\begin{solution} \end{solution}
\end{parts}
\end{multicols}



\question Calcular las restantes razones trigonométricas de $\alpha$, conocida:
\begin{multicols}{2}
\begin{parts}
\part[]  $\cos {\text{ }}\alpha {\text{ }} = {\text{ }}\frac{{\text{4}}}{{\text{5}}}{\text{   y   }}\alpha {\text{ }} \in {\text{ I}}$
\begin{solution} \end{solution}
\part[] $sen{\text{ }}\alpha {\text{ }} = {\text{ }}\frac{{\text{3}}}{{\text{5}}}{\text{   y   }}\alpha {\text{ }} \in {\text{ II}}$ 
\begin{solution} \end{solution}
\part[]  $tg{\text{ }}\alpha {\text{ }} = {\text{  - }}\frac{{\text{3}}}{{\text{4}}}{\text{   y   }}\alpha {\text{ }} \in {\text{ II}}$
\begin{solution} \end{solution}
\part[]  $\sec {\text{ }}\alpha {\text{ }} = {\text{ 2   y   }}\alpha {\text{ }} \in {\text{ IV}}$
\begin{solution} \end{solution}
\part[]  $\cos ec{\text{ }}\alpha {\text{ }} = {\text{  - 2   y   }}\alpha {\text{ }} \in {\text{ III}}$
\begin{solution} \end{solution}
\part[]  $\cot g{\text{ }}\alpha {\text{ }} = {\text{  - 2   y   }}\alpha {\text{ }} \in {\text{ IV}}$
\begin{solution} \end{solution}

\end{parts}
\end{multicols}

\question Si $\tg\alpha = \dfrac{3}{4}$, halla el valor de las siguientes razones trigonométricas: 
\begin{multicols}{3}
\begin{parts}
\part[]  $\tg{\text{ }}\left( {\frac{\pi }{{\text{2}}}{\text{  -  }}\alpha } \right){\text{ }} = {\text{ }}$
\begin{solution} \end{solution}
\part[] $\sen (\pi-\alpha) =$ 
\begin{solution} \end{solution}
\part[]  $\cos (\pi+\alpha) =$ 
\begin{solution} \end{solution}
\part[]  $\tg (\pi-\alpha) =$ 
\begin{solution} \end{solution}
\part[]  cotg $\left( {\frac{\pi }{2}{\text{ }} + {\text{ }}\alpha } \right)$ =
\begin{solution} \end{solution}
\part[]  $\sec (\pi-\alpha) =$ 
\begin{solution} \end{solution}
\part[]  cosec $\left( {\frac{{3\pi }}{2}{\text{  -  }}\alpha } \right)$
\begin{solution} \end{solution}
\part[]  cos $\left( {\frac{{3\pi }}{2}{\text{ }} + {\text{ }}\alpha } \right)$ = 

\begin{solution} \end{solution}
\part[]  $\sen (\pi+\alpha) =$ 
\begin{solution} \end{solution}
\part[]  tg $\left( {\frac{\pi }{2}{\text{ }} + {\text{ }}\alpha } \right)$ 
\begin{solution} \end{solution}
\part[]  $\cotg (\pi+\alpha) =$ 
\begin{solution} \end{solution}
\part[]  $\sec (\pi-\alpha) =$ 
\begin{solution} \end{solution}

\end{parts}
\end{multicols}

\question Resolver las siguientes ecuaciones:

\begin{multicols}{1}
\begin{parts}
\part[]  sen 2x = $\frac{1}{2}$
\begin{solution} \end{solution}
\part[]  tg $\frac{x}{2}{\text{ }} = {\text{ }}\frac{{\sqrt {\text{3}} }}{3}$    
\begin{solution} \end{solution}
\part[]  ${\text{sen }}\left( {{\text{3x  -  }}\frac{\pi }{{\text{2}}}} \right){\text{ }} = {\text{  -  }}\frac{{\text{1}}}{{\text{2}}}$
\begin{solution} \end{solution}
\end{parts}
\end{multicols}

\question Resolver las siguientes ecuaciones:

\begin{multicols}{1}
\begin{parts}
\part[]  2 sen x + cosec x = 2$\sqrt 2 $
\begin{solution} \end{solution}
\part[]  $\sen x = \cos^2 x + 1$  
\begin{solution} \end{solution}
\part[]  tg x – sen x = 0
\begin{solution} \end{solution}
\part[]  sen x . cos x = 2.sen x
\begin{solution} \end{solution}
\part[]  2 cos x – 3tg x = 0
\begin{solution} \end{solution}
\part[]  tg x + 3.cotg x = 4
\begin{solution} \end{solution}
\end{parts}
\end{multicols}

\begin{comment}
\question 
\begin{multicols}{3}
\begin{parts}
\part[]  
\begin{solution} \end{solution}
\end{parts}
\end{multicols}
\end{comment}



\end{questions}

\end{document}


