% Template created by Karol Kozioł (www.karol-koziol.net) for ShareLaTeX

\documentclass[a4paper,spanish,9pt]{extarticle}
\usepackage[utf8]{inputenc}
\usepackage[T1]{fontenc}
\usepackage{graphicx}
\usepackage{xcolor}
\usepackage{tikz}

\usepackage{amsmath,amssymb,textcomp}
\everymath{\displaystyle}

\usepackage{times}
\renewcommand\familydefault{\sfdefault}
\usepackage{tgheros}
\usepackage[defaultmono,scale=0.85]{droidmono}

\usepackage{multicol}
\setlength{\columnseprule}{0pt}
\setlength{\columnsep}{20.0pt}

\usepackage[utf8]{inputenc}
\usepackage[spanish]{babel}
\usepackage{eurosym}

\usepackage{graphicx}
\graphicspath{{../img/}}
\usepackage{svg}

\usepackage{hyperref}

\usepackage{geometry}
\geometry{
a4paper,
total={210mm,297mm},
left=10mm,right=10mm,top=10mm,bottom=15mm}

\linespread{1.3}


% custom title
\makeatletter
\renewcommand*{\maketitle}{%
\noindent
\begin{minipage}{0.6\textwidth}
\begin{tikzpicture}
\node[rectangle,rounded corners=6pt,inner sep=10pt,fill=blue!50!black,text width= 0.95\textwidth] {\color{white}\Huge \@title};
\end{tikzpicture}
\end{minipage}
\hfill
\begin{minipage}{0.35\textwidth}
\begin{tikzpicture}
\node[rectangle,rounded corners=3pt,inner sep=10pt,draw=blue!50!black,text width= 0.95\textwidth] {\begin{tabular}{cc} \multirow{2}{1cm}{\includegraphics[width=0.15\columnwidth]{header_right}}& \@author \\ & \ies \end{tabular}};
\end{tikzpicture}
\end{minipage}
\bigskip\bigskip
}%
\makeatother

% custom section
\usepackage[explicit]{titlesec}
\newcommand*\sectionlabel{}
\titleformat{\section}
  {\gdef\sectionlabel{}
   \normalfont\sffamily\Large\bfseries\scshape}
  {\gdef\sectionlabel{\thesection\ }}{0pt}
  {
\noindent
\begin{tikzpicture}
\node[rectangle,rounded corners=3pt,inner sep=4pt,fill=blue!50!black,text width= 0.95\columnwidth] {\color{white}\sectionlabel#1};
\end{tikzpicture}
  }
\titlespacing*{\section}{0pt}{15pt}{10pt}


% custom footer
\usepackage{fancyhdr}
\makeatletter
\pagestyle{fancy}
\fancyhead{}
\fancyfoot[C]{\footnotesize \@author \ - \ies}
\renewcommand{\headrulewidth}{0pt}
\renewcommand{\footrulewidth}{0pt}
\makeatother
\usepackage{multirow} % para las tablas


\title{Potencias, radicales y logaritmos}
\author{Departamento de Matemáticas}
\date{2014}
\newcommand{\ies}{IES Pedro Cerrada}



\begin{document}

\maketitle

\begin{multicols*}{2}


\section{Potencias}

Es importante destacar que las propiedades se pueden leer (y por tanto aplicar) de izquierda a derecha o al revés.
$$\forall \ n,m \in \mathbb{N} \ y \ \forall \  n,m \in \mathbb{R}:$$

\begin{tabular}{ll}
\textbf{Definición} de potencia: & $a^n = a \cdot a \stackrel{n}{\cdots} a$ \\
\textbf{Potencia} de exponente \textbf{negativo}: & $a^{-n} = \dfrac{1}{a^n}$ \\
\textbf{Potencia} de exponente \textbf{0} $\left( Si \ a \neq 0\right)$: & $a^{0} = 1$ \\
\textbf{Producto} de potenc. de la \textbf{misma base}: & $a^n a^m = a^{n+m}$ \\
\textbf{Cociente} de potenc. de la \textbf{misma base}: & $\dfrac{a^n}{a^m}  = a^{n-m}$ \\
\textbf{Potencia} de una \textbf{potencia}: & $\left(a^n\right)^m  = a^{n \cdot m}$ \\
\textbf{Potencia} de un \textbf{producto}: & $\left(a \cdot b \right)^n = a^n \cdot b^n$ \\
\textbf{Potencia} de un \textbf{cociente}: & $\left(\dfrac{a}{b} \right)^n = \dfrac{a^n}{a^n}$
\end{tabular}

\subsection{Ejemplos}

\begin{tabular}{lll}
$2^3=2\cdot2\cdot2$ & $3^0=1$ & $2^{-3}=\dfrac{1}{2^3}$\\
$2^3 \cdot 2^4 = 2^{4+3} = 2^{7}$ & $\left(\dfrac{2}{5}\right)^{-3}=\left(\dfrac{5}{2}\right)^{3}$ & $\dfrac{2^4}{2^3} = 2^{4-3} = 2$\\
$2^5:2^3 = 2^{5-3} = 2^2$ & $\left(3^2\right)^4  = 3^{2 \cdot 4}=3^{8}$ & $\left(\dfrac{1}{2} \right)^3 = \dfrac{1^3}{2^3}$\\
$2^3 \cdot 3^3=\left(2 \cdot 3 \right)^3 = 6^3$
\end{tabular}

\section{Radicales}
Recuerda que: $\sqrt[n]{a}=b \longleftrightarrow b^n=a$. De la definición se deducen las siguientes propiedades:

\begin{tabular}{ll}
\textbf{Definición} de potencia: & $a^n = a \cdot a \stackrel{n}{\cdots} a$ \\
\textbf{Potencia} de exponente \textbf{negativo}: & $a^{-n} = \dfrac{1}{a^n}$ \\
\textbf{Potencia} de exponente \textbf{0} $\left( Si \ a \neq 0\right)$: & $a^{0} = 1$ \\
\textbf{Producto} de potenc. de la \textbf{misma base}: & $a^n a^m = a^{n+m}$ \\
\textbf{Cociente} de potenc. de la \textbf{misma base}: & $\dfrac{a^n}{a^m}  = a^{n-m}$ \\
\textbf{Potencia} de una \textbf{potencia}: & $\left(a^n\right)^m  = a^{n \cdot m}$ \\
\textbf{Potencia} de un \textbf{producto}: & $\left(a \cdot b \right)^n = a^n \cdot b^n$ \\
\textbf{Potencia} de un \textbf{cociente}: & $\left(\dfrac{a}{b} \right)^n = \dfrac{a^n}{a^n}$
\end{tabular}

\section{Logaritmos}

\begin{tabular}{lll}
$a^n a^m = a^{n+m}$ & $\frac{a^n}{a^m} = a^{n-m}$ & $(a^n)^m = a^{n \cdot m}$\\
$a^n a^m = a^{n+m}$ & $\frac{a^n}{a^m} = a^{n-m}$ & $(a^n)^m = a^{n \cdot m}$\\
$a^n a^m = a^{n+m}$ & $\frac{a^n}{a^m} = a^{n-m}$ & $(a^n)^m = a^{n \cdot m}$\\
$a^n a^m = a^{n+m}$ & $\frac{a^n}{a^m} = a^{n-m}$ & $(a^n)^m = a^{n \cdot m}$\\
$a^n a^m = a^{n+m}$ & $\frac{a^n}{a^m} = a^{n-m}$ & $(a^n)^m = a^{n \cdot m}$\\
$a^n a^m = a^{n+m}$ & $\frac{a^n}{a^m} = a^{n-m}$ & $(a^n)^m = a^{n \cdot m}$
\end{tabular}




\section{Versión Online}

\url{https://goo.gl/kZNTW4} \includegraphics[width=0.15\columnwidth]{qr_chuletapot}





\end{multicols*}

\end{document}
