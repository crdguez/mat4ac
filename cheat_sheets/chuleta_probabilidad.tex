% Template created by Karol Kozioł (www.karol-koziol.net) for ShareLaTeX

\documentclass[a4paper,spanish,9pt]{extarticle}
\usepackage[utf8]{inputenc}

\usepackage[T1]{fontenc}
\usepackage{verbatim}
\usepackage{graphicx}
\usepackage{xcolor}
\usepackage{pgf,tikz}
\usepackage{mathrsfs}

\usetikzlibrary{shapes, calc, shapes, arrows, math, babel}

\usepackage{amsmath,amssymb,textcomp}
\everymath{\displaystyle}

\usepackage{times}
\renewcommand\familydefault{\sfdefault}
\usepackage{tgheros}
\usepackage[defaultmono,scale=0.85]{droidmono}

\usepackage{multicol}
\setlength{\columnseprule}{0pt}
\setlength{\columnsep}{20.0pt}

\usepackage[utf8]{inputenc}
\usepackage[spanish]{babel}
\usepackage{eurosym}

\usepackage{graphicx}
\graphicspath{{../img/}}
\usepackage{svg}

\usepackage{hyperref}

\usepackage{geometry}
\geometry{
a4paper,
total={210mm,297mm},
left=10mm,right=10mm,top=10mm,bottom=15mm}

\linespread{1.3}

\newcommand{\samedir}{\mathbin{\!/\mkern-5mu/\!}}

% custom title
\makeatletter
\renewcommand*{\maketitle}{%
\noindent
\begin{minipage}{0.6\textwidth}
\begin{tikzpicture}
\node[rectangle,rounded corners=6pt,inner sep=10pt,fill=blue!50!black,text width= 0.95\textwidth] {\color{white}\Huge \@title};
\end{tikzpicture}
\end{minipage}
\hfill
\begin{minipage}{0.35\textwidth}
\begin{tikzpicture}
\node[rectangle,rounded corners=3pt,inner sep=10pt,draw=blue!50!black,text width= 0.95\textwidth] {\begin{tabular}{cc} \multirow{2}{1cm}{\includegraphics[width=0.15\columnwidth]{header_right}}& \@author \\ & \ies \end{tabular}};
\end{tikzpicture}
\end{minipage}
\bigskip\bigskip
}%
\makeatother

% custom section
\usepackage[explicit]{titlesec}
\newcommand*\sectionlabel{}
\titleformat{\section}
  {\gdef\sectionlabel{}
   \normalfont\sffamily\Large\bfseries\scshape}
  {\gdef\sectionlabel{\thesection\ }}{0pt}
  {
\noindent
\begin{tikzpicture}
\node[rectangle,rounded corners=3pt,inner sep=4pt,fill=blue!50!black,text width= 0.95\columnwidth] {\color{white}\sectionlabel#1};
\end{tikzpicture}
  }
\titlespacing*{\section}{0pt}{15pt}{10pt}


% custom footer
\usepackage{fancyhdr}
\makeatletter
\pagestyle{fancy}
\fancyhead{}
\fancyfoot[C]{\footnotesize \@author \ - \ies}
\renewcommand{\headrulewidth}{0pt}
\renewcommand{\footrulewidth}{0pt}
\makeatother
\usepackage{multirow} % para las tablas


\title{Probabilidad}
\author{Departamento de Matemáticas}
\date{2014}
\newcommand{\ies}{IES Pedro Cerrada}



\begin{document}

\maketitle



\begin{multicols*}{2}


\section{Experimento aleatorio}

Un \textbf{experimento aleatorio} o no determinista es aquél que si se repite varias veces no está garantizado obtener siempre el mismo resultado. Es decir, no se puede determinar cuál va a ser el resultado del experimiento hasta que no se realiza. En caso contrario, decimos que el experimento es \textbf{determinista}

Un experimento es aleatorio cuando depende de muchos factores y cualquier pequeña modificación de alguno implica obtener un resultado diferente.

\subsection{Ejemplos}

\begin{itemize}
 \item \textbf{Aleatorio}: Lanzar un dado y ver el resultado
 \item \textbf{Determinista}: Calcular el tiempo que tarda en caer un objeto al suelo desde una distancia determinada
 \end{itemize} 
 
\section{Espacio muestral y sucesos}
\begin{itemize}
\item \textbf{Espacio muestral}: Conjunto de los posibles resultados del experimento. Se denota: $E$
\item \textbf{Sucesos simples o elementales}: Cualquiera de los elementos del espacio muestral
\item \textbf{Sucesos compuestos}: Sucesos formados por varios simples. 
\item \textbf{Suceso seguro}: Suceso compuesto por los elementos del Espacio muestral. Se cumple siempre
\item \textbf{Suceso imposible}: Cualquier suceso que no se cumpla nunca. Se denota con el símbolo: $\varnothing$
\item \textbf{Suceso contrario}: Si $A$ es un suceso, $\overline{A}$ es el suceso contrario. Es aquel que se cumple cuando no se cumple $A$
\end{itemize}

\subsection{Ejemplo:} Lanzamos un dado y comprobamos la cara que sale.
\begin{itemize}
\item \textbf{Espacio muestral}: $E=\lbrace 1,2,3,4,5,6 \rbrace $
\item \textbf{Sucesos simples o elementales}: $1$, $2$, $3$, $4$, $5$ ó $6$
\item \textbf{Sucesos compuestos}: $A=\lbrace que\ salga\ par\rbrace=\lbrace2,4,6\rbrace$
\item \textbf{Suceso seguro}: $E=\lbrace 1,2,3,4,5,6 \rbrace $
\item \textbf{Suceso imposible}: $\varnothing=\lbrace que\ salga \ mayor \ que \ 6\rbrace$
\item \textbf{Suceso contrario}: Si $A=\lbrace que\ salga\ par\rbrace=\lbrace2,4,6\rbrace$, $\overline{A}=\lbrace que\ salga\ impar\rbrace=\lbrace1,3,5\rbrace$ 
\end{itemize}


\section{Operaciones con sucesos y relaciones}

\def\firstcircle{(0,0) circle (1.5cm)}
\def\secondcircle{(0:2cm) circle (1.5cm)}
\def\espacio{(-2,-2) rectangle (4,2)}

\colorlet{circle edge}{blue!50}
\colorlet{circle area}{blue!20}

\tikzset{filled/.style={fill=circle area, draw=circle edge, thick},
    outline/.style={draw=circle edge, thick}}

\setlength{\parskip}{5mm}
\begin{itemize}
\item \textbf{Unión}: la unión de los sucesos $A$ y $B$ es aquel suceso que contiene a todos los elementos de $A$ y a  los de $B$. Se denota: $A\cup B$ 

% Set A or B
\begin{tikzpicture}[scale=0.8]
\draw[outline] \espacio node[above] {$E$};
    \draw[filled] \firstcircle node {$A$}
                  \secondcircle node {$B$};
    \node[anchor=south] at (1,1.3) {$A \cup B$};
\end{tikzpicture}

\item \textbf{Intersección}: la intersección de los sucesos $A$ y $B$ es aquel suceso que contiene a todos los elementos que están tanto en $A$ como en $B$. Se denota: $A\cap B$  
% Set A and B
\begin{tikzpicture}[scale=0.8]
	\draw[outline] \espacio node[above] {$E$};
    \begin{scope}
        \clip \firstcircle;
        \fill[filled] \secondcircle;
    \end{scope}
    \draw[outline] \firstcircle node {$A$};
    \draw[outline] \secondcircle node {$B$};
    %\node[anchor=south] at (current bounding box.north) {$A \cap B$};
    \node[anchor=south] at (1,1.3) {$A \cap B$};
\end{tikzpicture}
\end{itemize}

\subsection{Ejemplo} Tomamos como experimento el resultado de lanzar un dado, y los sucesos: \\
\begin{tabular}{l}
$A=\lbrace que\ salga\ par\rbrace=\lbrace2,4,6\rbrace$ \\
$B=\lbrace que\ sea\ mayor\ que\ 3\rbrace=\lbrace4,5,6\rbrace$ \\
$C=\lbrace que\ salga\ impar\rbrace=\lbrace1,3,5\rbrace$
\end{tabular}
\begin{itemize}	
	\item $A\cup B=\lbrace2,4,5,6\rbrace$ \\
	% Set A or B
\begin{tikzpicture}[scale=0.7]
\draw[outline] \espacio node[above] {$E$};
    \draw[filled] \firstcircle node[left] {$2$}
                  \secondcircle node[right] {$5$};
    \node[anchor=south] at (1,1.3) {$A \cup B$};
    \node[anchor=south] at (1,0.25) {$4$};
    \node[anchor=south] at (1,-0.75) {$6$};
    
\end{tikzpicture}
	\item $A\cap B=\lbrace4,6\rbrace$\\
	% Set A and B
\begin{tikzpicture}[scale=0.7]
	\draw[outline] \espacio node[above] {$E$};
    \begin{scope}
        \clip \firstcircle;
        \fill[filled] \secondcircle;
    \end{scope}
    \draw[outline] \firstcircle node[left] {$2$};
    \draw[outline] \secondcircle node[right] {$5$};
    %\node[anchor=south] at (current bounding box.north) {$A \cap B$};
    \node[anchor=south] at (1,1.3) {$A \cap B$};
    \node[anchor=south] at (1,0.25) {$4$};
    \node[anchor=south] at (1,-0.75) {$6$};
\end{tikzpicture}
\item $A\cup C=\lbrace1,2,3,4,5,6\rbrace=E$
\item $A\cap C=\varnothing$
\end{itemize}

\subsection{Compatibilidad de sucesos} Se dice que dos sucesos son incompatibles cuando su intersección es el conjunto vacío. En caso contrario se dice que son compatibles.

\subsubsection{Ejemplo} En el ejemplo anterior, $A$ y $B$ son compatibles y $A$ y $C$ incompatibles.


\section{Probabilidad en experimentos regulares y Regla de Laplace} Cuando todos los sucesos elementales de un \textbf{espacio muestral finito} están en las mismas condiciones de suceder se dice que son \textbf{equiprobables}, y al experimento se le llama \textbf{regularregular}.
\subsection{Ejemplos de experimentos regulares} Lanzamiento de dados, monedas, extracción de cartas, ...

\subsection{Regla de Laplace} La probabilidad de un suceso de un experimento regular viene determinada por la Regla de Laplace:
$$P(A)=\dfrac{Casos\ favorables}{Casos\ posibles} $$
\subsubsection{Ejemplos}
\begin{tabular}{l}
$a$\\
$b$
\end{tabular}

\section{Propiedades de la probabilidad} La probabilidad de un experimento regular cumple las siguientes propiedades:
\begin{itemize}
\item $0 \leq P(A) \leq 1$ 
\item $P(E) = 1$ y $P(\varnothing) = 0$
\item $P(A) + P(\overline A) = 1$
\item $P(A \cup B) = P(A) + P(B) - P(A \cap B)$
\end{itemize}
Podemos extender el concepto de probabilidad a cualquier función que cumpla las propiedades anteriores.  





\end{multicols*}

\end{document}
