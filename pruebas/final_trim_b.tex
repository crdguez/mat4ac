\documentclass[addpoints,spanish, 12pt,a4paper]{exam}
%\documentclass[answers, spanish, 12pt,a4paper]{exam}
%\printanswers
\pointpoints{punto}{puntos}
\hpword{Puntos:}
\vpword{Puntos:}
\htword{Total}
\vtword{Total}
\hsword{Resultado:}
\hqword{Ejercicio:}
\vqword{Ejercicio:}

\usepackage[utf8]{inputenc}
\usepackage[spanish]{babel}
\usepackage{eurosym}
%\usepackage[spanish,es-lcroman, es-tabla, es-noshorthands]{babel}


\usepackage[margin=1in]{geometry}
\usepackage{amsmath,amssymb}
\usepackage{multicol}
\usepackage{yhmath}

\usepackage{verbatim}
%\usepackage{pstricks}


\usepackage{graphicx}
\graphicspath{{../img/}} 

\newcommand{\class}{4º Académicas}
\newcommand{\examdate}{\today}
\newcommand{\examnum}{Examen final de trimestre 1}
\newcommand{\tipo}{B}


\newcommand{\timelimit}{50 minutos}

\renewcommand{\solutiontitle}{\noindent\textbf{Solución:}\enspace}


\pagestyle{head}
\firstpageheader{\includegraphics[width=0.2\columnwidth]{header_left}}{\textbf{Departamento de Matemáticas\linebreak \class}\linebreak \examnum}{\includegraphics[width=0.1\columnwidth]{header_right}}
\runningheader{\class}{\examnum}{Página \thepage\ of \numpages}
\runningheadrule


\begin{document}

\noindent
\begin{tabular*}{\textwidth}{l @{\extracolsep{\fill}} r @{\extracolsep{6pt}} }
\textbf{Nombre:} \makebox[3.5in]{\hrulefill} & \textbf{Fecha:}\makebox[1in]{\hrulefill} \\
 & \\
\textbf{Tiempo: \timelimit} & Tipo: \tipo 
\end{tabular*}
\rule[2ex]{\textwidth}{2pt}
Esta prueba tiene \numquestions\ ejercicios. La puntuación máxima es de \numpoints. 
La nota final de la prueba será la parte proporcional de la puntuación obtenida sobre la puntuación máxima. Para la recuperación de pendientes de 3º se tendrán en cuenta los apartados: 1.a y 4.a

\begin{center}


\addpoints
 %\gradetable[h][questions]
	\pointtable[h][questions]
\end{center}

\noindent
\rule[2ex]{\textwidth}{2pt}

\begin{questions}

\begin{comment}
\question[1] 
\begin{solution} \end{solution}

\addpoints
\end{comment}

\question Calcula: 
\begin{parts}
\part[1] Racionaliza y simplifica:  $\dfrac{\sqrt{3}}{2\sqrt{3}-\sqrt{2}}$ 
\begin{solution} $=\dfrac{\sqrt{3}\cdot\left(2\sqrt{3}+\sqrt{2}\right)}{\left(2\sqrt{3}-\sqrt{2}\right)\left(2\sqrt{3}+\sqrt{2}\right)}=\dfrac{6\sqrt{6}}{12-2}=\dfrac{6\sqrt{6}}{10}$ \end{solution}
\part[1] Aplica la definición de logaritmo para calcular: $\log_4 \sqrt{0,25}$
\begin{solution} $\to 4^x=\sqrt{\frac{1}{4}}\to4^x=4^{-1/2}\to\log_4 \sqrt{0,25}=-\frac{1}{2}$ \end{solution}

\end{parts}

\addpoints



\question[1] Halla el valor de \emph{k} para que la siguiente división sea exacta: $(3x^2+kx-2):(x+2)$
\begin{solution} $\to 10-2k=0 \to k=5 $ \end{solution}

\addpoints


\question[1] Simplifica la fracción algebraica: $$\dfrac{2x^4-6x^3+6x^2-2x}{6x^3-12x^2+6x} $$
\begin{solution}$=\dfrac{2x\left(x-1\right)^3}{6x\left(x-1\right)^2}=\dfrac{x-1}{3}$ \end{solution}

\addpoints

\question Resuelve las siguientes ecuaciones: 
\begin{parts}

\part[2] $$\dfrac{6x+1}{x^2-4}-\dfrac{x}{x-2}=\dfrac{x+1}{x+2}$$
\begin{solution}$\to \dfrac{6x+1}{\left(x+2\right)\left(x-2\right)}-\dfrac{x\left(x+2\right)}{\left(x+2\right)\left(x-2\right)}=\dfrac{\left(x+1\right)\left(x-2\right)}{\left(x+2\right)\left(x-2\right)} \to 6x+1-x^2-2x=x^2-2x+x-2\to 0=2x^2-5x-3 \to x=\frac{5 \pm \sqrt{25+24}}{4}=\frac{5 \pm 7}{4}=\left\lbrace \begin{gathered}
	  x=-\frac{1}{2}  \hfill\\
	  x=3  
	\end{gathered} \right.$ \end{solution}
\part[2] $$2x^4-6x^3+6x^2-2x=0$$
\begin{solution}$P(x)2x^4-6x^3+6x^2-2x=2x\left(x-1\right)^3$. Soluciones: $x=0$ y $x=1$ triple\end{solution}


\part[2] $$\sqrt{3x-2}+\sqrt{x-1}=3$$
\begin{solution}$\to\sqrt{3x-2}=3-\sqrt{x-1}\to 3x-2=9+x-1-6\sqrt{x-1}\to6\sqrt{x-1}=9+x-1-3x+32\to6\sqrt{x-1}=10-2x\to3\sqrt{x-1}=5-x\to x-1=25+x^2-10x\to x^2-19x+34=0$. Soluciones: $x=2$ (Sí) y $x=17$ No  \end{solution}


\part[2] $$\left(x^2-5x+5\right)\log 5 + \log{20}=\log 4$$

\begin{solution} $\to 5^{\left(x^2-5x+5\right)}\cdot20=4 \to 5^{\left(x^2-5x+5\right)}=\frac{1}{5} \to 5^{\left(x^2-5x+5\right)}=5^{-1}\to x^2-5x+5 = -1 \to x^2-5x+6=0 \to x=\frac{5 \pm \sqrt{25-24}}{2}=\left\lbrace \begin{gathered}
	  x=3  \to \textup{es solución} \hfill\\
	  x=2  \to \textup{es solución}
	\end{gathered} \right. $  \end{solution}

\end{parts}

\addpoints

\end{questions}

\end{document}
\grid
