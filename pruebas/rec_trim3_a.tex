\documentclass[addpoints,spanish, 12pt,a4paper]{exam}
%\documentclass[answers, spanish, 12pt,a4paper]{exam}
\printanswers
\pointpoints{punto}{puntos}
\hpword{Puntos:}
\vpword{Puntos:}
\htword{Total}
\vtword{Total}
\hsword{Resultado:}
\hqword{Ejercicio:}
\vqword{Ejercicio:}

\usepackage[utf8]{inputenc}
\usepackage[spanish]{babel}
\usepackage{eurosym}
%\usepackage[spanish,es-lcroman, es-tabla, es-noshorthands]{babel}

\usepackage{pgf,tikz}
\usetikzlibrary{shapes, calc, shapes, arrows, math, babel}
\usepackage[margin=1in]{geometry}
\usepackage{amsmath,amssymb}
\usepackage{multicol}

\usepackage{yhmath}

\usepackage{verbatim}
\usepackage[thinlines]{easytable}
%\usepackage{pstricks}


\usepackage{graphicx}
\graphicspath{{../img/}} 

\newcommand{\class}{4º Académicas}
\newcommand{\examdate}{\today}
\newcommand{\examnum}{Recuperación de 3ª evaluación}
\newcommand{\tipo}{A}


\newcommand{\timelimit}{50 minutos}

\renewcommand{\solutiontitle}{\noindent\textbf{Solución:}\enspace}


\pagestyle{head}
\firstpageheader{\includegraphics[width=0.2\columnwidth]{header_left}}{\textbf{Departamento de Matemáticas\linebreak \class}\linebreak \examnum}{\includegraphics[width=0.1\columnwidth]{header_right}}
\runningheader{\class}{\examnum}{Página \thepage\ of \numpages}
\runningheadrule
\pointsinrightmargin % Para poner las puntuaciones a la derecha. Se puede cambiar. Si se comenta, sale a la izquierda.
\extrawidth{-2.4cm} %Un poquito más de margen por si ponemos textos largos.
\marginpointname{ \emph{\points}}

\begin{document}

\noindent
\begin{tabular*}{\textwidth}{l @{\extracolsep{\fill}} r @{\extracolsep{6pt}} }
\textbf{Nombre:} \makebox[3.5in]{\hrulefill} & \textbf{Fecha:}\makebox[1in]{\hrulefill} \\
 & \\
\textbf{Tiempo: \timelimit} & Tipo: \tipo 
\end{tabular*}
\rule[2ex]{\textwidth}{2pt}
Esta prueba tiene \numquestions\ ejercicios. La puntuación máxima es de \numpoints. 
La nota final de la prueba será la parte proporcional de la puntuación obtenida sobre la puntuación máxima. 

\begin{center}


\addpoints
 %\gradetable[h][questions]
	\pointtable[h][questions]
\end{center}

\noindent
\textbf{ACLARACIÓN:} Los ejercicios de geometría se han de resolver de manera analítica (no gráfica). Los ejercicios de funciones deberán estar justificados con los cálculos que sean necesarios para su resolución. 

\rule[2ex]{\textwidth}{2pt}

\begin{questions}

\begin{comment}
\question[1] 
\begin{solution} \end{solution}

\addpoints
\end{comment}

\question Resuelve las siguientes cuestiones relacionadas con combinatoria. Indicando previamente \textbf{el tipo de agrupación que calculas} a partir de si importa el orden dentro de la agrupación y si los elementos se pueden repetir:
\begin{parts}
\part[1] ¿De cuántas formas podrán distribuirse 2 premios iguales entre diez aspirantes?  
\begin{solution} $ C_{10}^2=\frac{10!}{8!\cdot2!}=45$  \end{solution}
\part[1] ¿Y si los premios fueran diferentes? 

\begin{solution} $ V_{10}^2=\frac{10!}{8!}=90$  \end{solution}
\part[1] ¿Cuántas palabras se pueden formar con las letras de la palabra AMBROSI de forma que comiencen y terminen por vocal?
\begin{solution}
$V_3^2\cdot P_5= 3*2 \cdot 5!=6\cdot120=720 $
\end{solution}
\begin{comment}
\part[1] ¿Cuántos números naturales se pueden formar con las cifras 1, 3, 5 y 7 sin repetir ninguna de ellas? 
\begin{solution} $ V_4^1+V_4^2+V_4^3+V_4^4 \to ([4, 12, 24, 24], 64)$  \end{solution}
\end{comment}

\end{parts}

\question De una baraja de 40 cartas se extraen dos \textbf{sin} remplazamiento. Halla la probabilidad de cada apartado de dos formas: Sin reducir el experimento compuesto (\textbf{combinatoria}) y reduciéndolo (\textbf{probabilidad condicionada})

\begin{parts}
\begin{comment}
\part[1] de que sean dos ases
\begin{solution}
$\frac{V_{4}^2}{V_{40}^{2}}=\frac{4\cdot3}{40\cdot39}=\frac{1}{130}$ ó $P(A_1 \cap A_2)=P(A1)\cdot P(A_2 | A_1) = \frac{4}{40}\cdot \frac{3}{39}$
\end{solution}
\end{comment}
\begin{comment}
\part[1] de que sean dos figuras (sota, caballo o rey)
\begin{solution}
$  \frac{V_{12}^2}{V_{40}^2}=\frac{12\cdot11}{40\cdot39}=\frac{11}{130}$ ó $P(2F)=P(F_1)\cdot P(F_2 | F_1)=\frac{12}{40}\cdot \frac{11}{39}$
\end{solution}
\end{comment}
\begin{comment}
\part[1] de que al menos haya un as
\begin{solution}
$1-\frac{V_{36}^{2}}{V_{40}^{2}}=1-\frac{36\cdot35}{40\cdot39}=1-\frac{21}{26}=\frac{5}{26}$ ó $1-P(NA_1 \cap NA_2)=1 - P(NA_1)\cdot P(NA_2 | NA_1)=1 - \frac{36}{40}\cdot \frac{35}{39}$
\end{solution}
\end{comment}
\begin{comment}
\part[1] de que sean dos reyes
\begin{solution}
$\frac{V_{4}^2}{V_{40}^{2}}=\frac{4\cdot3}{40\cdot39}=\frac{1}{130}$ ó $P(R_1 \cap R_2)=P(R_1)\cdot P(R_2 | R_1) = \frac{4}{40}\cdot \frac{3}{39}$
\end{solution}
\end{comment}
\begin{comment}
\part[1] de que sean del mismo palo
\begin{solution}

$\frac{V_{4}^1 \cdot V_{10}^2}{V_{40}^{2}}=\frac{4\cdot10\cdot9}{40\cdot39}=\frac{3}{13}$ ó $4\cdot P(P_1 \cap P_2)=4 \cdot P(P_1)\cdot P(P_2 | P_1) = 4\cdot \frac{10}{40}\cdot \frac{9}{39} $
\end{solution}
\end{comment}
%\begin{comment}
\part[1] de que sean un rey y una sota (o al revés). 
\begin{solution}
$\frac{V_{8}^1 \cdot V_{4}^1}{V_{40}^{2}}=\frac{8\cdot4}{40\cdot39}=\frac{4}{195}$ ó $2\cdot P(R_1 \cap S_2)= 2 \cdot P(R_1)\cdot P(S_2 | R_1) = 2 \cdot \frac{4}{40}\cdot \frac{4}{39} $
\end{solution}
%\end{comment}
\end{parts}





\question Dados el triángulo de vértices   $A(3, -1)$ ,   $B(5, 3)$ y   $C(-1, 3)$, determina:
\begin{parts} 
\part[1] si están alineados
\begin{solution} (False, Point2D(2, 4), Point2D(-6, 0)) \end{solution}


\part[1]  La recta que contiene a la altura que pasa por $A$ 
\begin{solution} $x= 3$ \end{solution}

	
\part[1] La recta que contiene a la altura que pasa por $C$
\begin{solution} (-2*x - 4*y + 10 = 0)
\end{solution} 
\part[1] El punto donde se cortan ambas rectas. 
\begin{solution} {x: 3, y: 1} \end{solution}

\end{parts}
\addpoints

\begin{comment}
\question  Dada la siguiente función $f(x) =
\left\{
	\begin{array}{clc}
		-2x  & \mbox{si } & x < -2 \\
		x^2-2x+1 & \mbox{si } & -2 \leq x < 2 \\
		2x-3 & \mbox{si } & x > 2
	\end{array}
\right.$
\begin{parts} 
\part[2] Representa la función gráficamente 
\begin{solution} 

\begin{tikzpicture}[domain=-4.1:5.1 ,>=triangle 45, scale=0.5]

\tikzmath{
			\a = 1; \b = -2; \c = 1; 
			\v = - \b / ( 2 * \a);
			\m1 = -2; \n1 = 0;
			\m3 = 2; \n3 = -3;
			\xmin1 = - 4.1; \xmax1 = -2;
			\xmin2 = - 2; \xmax2 = 2;
			\xmin3 = 2; \xmax3 = 5.1;
          }
          
 
\draw[color=red, domain=\xmin1 -0.5:\xmax1]    plot (\x,{\m1*(\x) + \n1}) node[right] {};

\draw [red] (\xmax1,{\m1*(\xmax1) + \n1}) circle (0.25) node [left] {};

\draw[color=red, domain=\xmin2:\xmax2]    plot (\x,{\a*(\x)^2 + \b *\x + \c})             node[right] {}; 
\draw [red, fill] (\xmin2,{\a*(\xmin2)^2 + \b *\xmin2 + \c}) circle (0.25) node [left] {};
\draw [red] (\xmax2,{\a*(\xmax2)^2 + \b *\xmax2 + \c}) circle (0.25) node [left] {};

\draw[color=red, domain=\xmin3  :\xmax3 + 0.1]    plot (\x,{\m3*(\x) + \n3}) node[right] {};
\draw [red] (\xmin3,{\m3*(\xmin3) + \n3}) circle (0.25) node [left] {};

\draw[very thin,color=lightgray,dash pattern=on 1pt off 1pt] (\xmin1 - 0.5, \a * \v * \v + \b * \v + \c - 0.5) grid (\xmax3 + 0.5 , \a * \xmin2 * \xmin2 + \b * \xmin2 + \c + 0.5);

\draw[<->] (\xmin1 -1,0) -- (\xmax3 + 1,0) node[right] {$x$};
\draw[<->] (0,\a * \v * \v + \b * \v + \c - 0.5) -- (0, \a * \xmin2 * \xmin2 + \b * \xmin2 + \c + 0.5 ) node[above] {$y$};

\end{tikzpicture}  \end{solution}
\part[1] Indica el \emph{dominio} y el \emph{recorrido} de la función utilizando la notación de conjuntos de números reales
\begin{solution} $Dom(f)=\mathbb{R}-\lbrace 2 \rbrace $ \\
$ Im(f)=\left[0, +\infty\right]$
\end{solution}


\end{parts}
\addpoints
\end{comment}

\begin{comment}
\question Dada la siguiente función a trozos:

\begin{tikzpicture}[domain=-4.1:5.1 ,>=triangle 45, scale=0.75]

\tikzmath{
			\a = 1; \b = -2; \c = 1; 
			\v = - \b / ( 2 * \a);
			\m1 = -1; \n1 = -2;
			\m2 = (1/3); \n2 = 1/3;
			\m3 = 1; \n3 = -2;
			\xmin1 = - 4.1; \xmax1 = -1;
			\xmin2 = - 1; \xmax2 = 2;
			\xmin3 = 2; \xmax3 = 5.1;
          }
          
 
\draw[<-,color=red, domain=\xmin1 -0.5:\xmax1]    plot (\x,{\m1*(\x) + \n1}) node[right] {};

\draw [red] (\xmax1,{\m1*(\xmax1) + \n1}) circle (0.25) node [left] {};

\draw[color=red, domain=\xmin2:\xmax2]    plot (\x,{\m2*(\x) + \n2}) node[right] {};

\draw [red, fill] (\xmin2,{\m2*(\xmin2) + \n2}) circle (0.25) node [left] {};
\draw [red] (\xmax2,{\m2*(\xmax2) + \n2}) circle (0.25) node [left] {};

%\draw[color=red, domain=\xmin2:\xmax2]    plot (\x,{\a*(\x)^2 + \b *\x + \c})             node[right] {}; 

%\draw [red, fill] (\xmin2,{\a*(\xmin2)^2 + \b *\xmin2 + \c}) circle (0.25) node [left] {};
%\draw [red] (\xmax2,{\a*(\xmax2)^2 + \b *\xmax2 + \c}) circle (0.25) node [left] {};

\draw[->, color=red, domain=\xmin3  :\xmax3 + 0.1]    plot (\x,{\m3*(\x) + \n3}) node[right] {};
\draw [red] (\xmin3,{\m3*(\xmin3) + \n3}) circle (0.25) node [left] {};

\draw[very thin,color=lightgray,dash pattern=on 1pt off 1pt] (\xmin1 - 0.5, -2) grid (\xmax3 + 0.5 , \a * \xmin2 * \xmin2 + \b * \xmin2 + \c + 0.5);

\draw[<->] (\xmin1 -1,0) -- (\xmax3 + 1,0) node[right] {$x$};
\draw[<->] (0,-2 - 0.5) -- (0, \a * \xmin2 * \xmin2 + \b * \xmin2 + \c + 0.5 ) node[above] {$y$};

\end{tikzpicture}

\begin{parts} 
\part[1] Indica el \emph{dominio} y el \emph{recorrido} de la función utilizando la notación de conjuntos de números reales
\begin{solution} $Dom(f)=\mathbb{R}-\lbrace 2 \rbrace $ \\
$ Im(f)=\left(-1, +\infty\right)$
\end{solution}
\part[1] Calcula las ecuaciones explícitas de las rectas  que contienen a cada trozo de la función. 
\begin{solution} 
$y=-x-2$, $y=\frac{1}{3}x+\frac{1}{3}$, $y=x-2 $
\end{solution}
\part[2] Da la expresión analítica de la función a trozos
\begin{solution} 
$f(x) =
\left\{
	\begin{array}{clc}
		-x -2  & \mbox{si } & x < -1 \\
		\frac{1}{3}x+\frac{1}{3} & \mbox{si } & -1 \leq x < 2 \\
		x-2 & \mbox{si } & x > 2
	\end{array}
\right.$
\end{solution}
\end{parts}
\end{comment}

\begin{comment}
\question Dada la siguiente función a trozos:

\begin{tikzpicture}[domain=-4.1:5.1 ,>=triangle 45, scale=0.75]

\tikzmath{
			\a = 1; \b = -2; \c = 1; 
			\v = - \b / ( 2 * \a);
			\m1 = 2; \n1 = 3;
			\m2 = -(1/3); \n2 = -1/3;
			\m3 = -1; \n3 = 2;
			\xmin1 = - 4.1; \xmax1 = -1;
			\xmin2 = - 1; \xmax2 = 2;
			\xmin3 = 2; \xmax3 = 5.1;
          }
          
 
\draw[<-,color=red, domain=\xmin1 -0.5:\xmax1]    plot (\x,{\m1*(\x) + \n1}) node[right] {};

\draw [red] (\xmax1,{\m1*(\xmax1) + \n1}) circle (0.25) node [left] {};

\draw[color=red, domain=\xmin2:\xmax2]    plot (\x,{\m2*(\x) + \n2}) node[right] {};

\draw [red, fill] (\xmin2,{\m2*(\xmin2) + \n2}) circle (0.25) node [left] {};
\draw [red] (\xmax2,{\m2*(\xmax2) + \n2}) circle (0.25) node [left] {};

%\draw[color=red, domain=\xmin2:\xmax2]    plot (\x,{\a*(\x)^2 + \b *\x + \c})             node[right] {}; 

%\draw [red, fill] (\xmin2,{\a*(\xmin2)^2 + \b *\xmin2 + \c}) circle (0.25) node [left] {};
%\draw [red] (\xmax2,{\a*(\xmax2)^2 + \b *\xmax2 + \c}) circle (0.25) node [left] {};

\draw[->, color=red, domain=\xmin3  :\xmax3 + 0.1]    plot (\x,{\m3*(\x) + \n3}) node[right] {};
\draw [red] (\xmin3,{\m3*(\xmin3) + \n3}) circle (0.25) node [left] {};

\draw[very thin,color=lightgray,dash pattern=on 1pt off 1pt] (\xmin1 - 0.5, -6) grid (\xmax3 + 0.5 , \a * \xmin2 * \xmin2 + \b * \xmin2 + \c + 0.5);

\draw[<->] (\xmin1 -1,0) -- (\xmax3 + 1,0) node[right] {$x$};
\draw[<->] (0,-6 - 0.5) -- (0, \a * \xmin2 * \xmin2 + \b * \xmin2 + \c + 0.5 ) node[above] {$y$};

\end{tikzpicture}

\begin{parts} 
\part[1] Indica el \emph{dominio} y el \emph{recorrido} de la función utilizando la notación de conjuntos de números reales
\begin{solution} $Dom(f)=\mathbb{R}-\lbrace 2 \rbrace $ \\
$ Im(f)=\left(-\infty, 1\right)$
\end{solution}
\part[1] Calcula las ecuaciones explícitas de las rectas  que contienen a cada trozo de la función. 
\begin{solution} 
$y=2x+3$, $y=-\frac{1}{3}x-\frac{1}{3}$, $y=-x+2 $
\end{solution}
\part[2] Da la expresión analítica de la función a trozos
\begin{solution} 
$f(x) =
\left\{
	\begin{array}{clc}
		2x+3  & \mbox{si } & x < -1 \\
		-\frac{1}{3}x-\frac{1}{3} & \mbox{si } & -1 \leq x < 2 \\
		-x+2 & \mbox{si } & x > 2
	\end{array}
\right.$
\end{solution}
\end{parts}

\end{comment}

\question Dada la siguiente función a trozos:

\begin{tikzpicture}[domain=-4.1:5.1 ,>=triangle 45, scale=0.75]

\tikzmath{
			\a = 1; \b = -2; \c = 1; 
			\v = - \b / ( 2 * \a);
			\m1 = 2; \n1 = 3;
			\m2 = -(1/3); \n2 = -1/3;
			\m3 = -1; \n3 = 2;
			\xmin1 = - 4.1; \xmax1 = -1;
			\xmin2 = - 1; \xmax2 = 2;
			\xmin3 = 3; \xmax3 = 5.1;
          }
          
 
\draw[<-,color=red, domain=\xmin1 -0.5:\xmax1]    plot (\x,{\m1*(\x) + \n1}) node[right] {};

\draw [red] (\xmax1,{\m1*(\xmax1) + \n1}) circle (0.25) node [left] {};

\draw[color=red, domain=\xmin2:\xmax2]    plot (\x,{\m2*(\x) + \n2}) node[right] {};

\draw [red, fill] (\xmin2,{\m2*(\xmin2) + \n2}) circle (0.25) node [left] {};
\draw [red] (\xmax2,{\m2*(\xmax2) + \n2}) circle (0.25) node [left] {};

%\draw[color=red, domain=\xmin2:\xmax2]    plot (\x,{\a*(\x)^2 + \b *\x + \c})             node[right] {}; 

%\draw [red, fill] (\xmin2,{\a*(\xmin2)^2 + \b *\xmin2 + \c}) circle (0.25) node [left] {};
%\draw [red] (\xmax2,{\a*(\xmax2)^2 + \b *\xmax2 + \c}) circle (0.25) node [left] {};

\draw[->, color=red, domain=\xmin3  :\xmax3 + 0.1]    plot (\x,{\m3*(\x) + \n3}) node[right] {};
\draw [red] (\xmin3,{\m3*(\xmin3) + \n3}) circle (0.25) node [left] {};

\draw[very thin,color=lightgray,dash pattern=on 1pt off 1pt] (\xmin1 - 0.5, -6) grid (\xmax3 + 0.5 , \a * \xmin2 * \xmin2 + \b * \xmin2 + \c + 0.5);

\draw[<->] (\xmin1 -1,0) -- (\xmax3 + 1,0) node[right] {$x$};
\draw[<->] (0,-6 - 0.5) -- (0, \a * \xmin2 * \xmin2 + \b * \xmin2 + \c + 0.5 ) node[above] {$y$};

\end{tikzpicture}

\begin{parts} 
\part[1] Indica el \emph{dominio} y el \emph{recorrido} de la función utilizando la notación de conjuntos de números reales
\begin{solution} $Dom(f)=(-\infty, 2) \cup (3, +\infty)$ \\
$ Im(f)=\left(-\infty, 1\right)$
\end{solution}
\part[1] Calcula las ecuaciones explícitas de las rectas  que contienen a cada trozo de la función. 
\begin{solution} 
$y=2x+3$, $y=-\frac{1}{3}x-\frac{1}{3}$, $y=-x+2 $
\end{solution}
\part[1] Da la expresión analítica de la función a trozos
\begin{solution} 
$f(x) =
\left\{
	\begin{array}{clc}
		2x+3  & \mbox{si } & x < -1 \\
		-\frac{1}{3}x-\frac{1}{3} & \mbox{si } & -1 \leq x < 2 \\
		-x+2 & \mbox{si } & x > 3
	\end{array}
\right.$
\end{solution}

\begin{comment}
\question  Dada la siguiente función $f(x) =
\left\{
	\begin{array}{clc}
		4  & \mbox{si } & x < -2 \\
		-x^2 & \mbox{si } & -2 \leq x < 4 \\
		2x-3 & \mbox{si } & x > 4
	\end{array}
\right.$
\begin{parts} 
\part[2] Representa la función gráficamente (justificadamente)
\begin{solution} \begin{tikzpicture}[domain=-4.1:6.1 ,>=triangle 45, scale=0.35]


\tikzmath{
			\a = -1; \b = 0; \c = 0; 
			\v = - \b / ( 2 * \a);
			\m1 = 0; \n1 = 4;
			\m3 = 2; \n3 = -3;
			\xmin1 = - 4.1; \xmax1 = -2;
			\xmin2 = - 2; \xmax2 = 4;
			\xmin3 = 4; \xmax3 = 6.1;
          }
          
 
\draw[color=red, domain=\xmin1 -0.5:\xmax1-0.25]    plot (\x,{\m1*(\x) + \n1}) node[right] {};

\draw [red] (\xmax1,{\m1*(\xmax1) + \n1}) circle (0.25) node [left] {};

\draw[color=red, domain=\xmin2:\xmax2]    plot (\x,{\a*(\x)^2 + \b *\x + \c})             node[right] {}; 
\draw [red, fill] (\xmin2,{\a*(\xmin2)^2 + \b *\xmin2 + \c}) circle (0.25) node [left] {};
\draw [red] (\xmax2,{\a*(\xmax2)^2 + \b *\xmax2 + \c}) circle (0.25) node [left] {};

\draw[color=red, domain=\xmin3  :\xmax3 + 0.1]    plot (\x,{\m3*(\x) + \n3}) node[right] {};
\draw [red] (\xmin3,{\m3*(\xmin3) + \n3}) circle (0.25) node [left] {};

\draw[very thin,color=lightgray,dash pattern=on 1pt off 1pt] (\xmin1 - 0.5, \a * \xmax2*\xmax2 + \b * \xmax2 + \c - 0.5) grid (\xmax3 + 0.5 , \m3 * \xmax3 + \n3);

\draw[<->] (\xmin1 -1,0) -- (\xmax3 + 1,0) node[right] {$x$};
\draw[<->] (0,\a * \xmax2*\xmax2 + \b * \xmax2 + \c - 0.5) -- (0, \m3 * \xmax3 + \n3 ) node[above] {$y$};

\end{tikzpicture}  \end{solution}
\part[2] Indica el \emph{dominio} y el \emph{recorrido} de la función utilizando la notación de conjuntos de números reales
\begin{solution} $Dom(f)=\mathbb{R}-\lbrace 4 \rbrace $ \\
$ Im(f)=\left(-\infty, 0\right] \cup \lbrace 4 \rbrace \cup \left(5, \infty\right]$
\end{solution}
\end{comment}

\end{parts}


\addpoints




\end{questions}

\newpage 

\begin{tikzpicture}[line cap=round,line join=round,>=triangle 45,x=1cm,y=1cm, scale=0.78]
\draw [color=lightgray,dash pattern=on 1pt off 1pt, xstep=1cm,ystep=1cm] (-10.6,-10.4) grid (10.1,10.1);
\draw[<->,color=black] (-10.6,0) -- (10.1,0);
\foreach \x in {-10,-9,-8,-7,-6,-5,-4,-3,-2,-1,1,2,3,4,5,6,7,8,9,10}
\draw[shift={(\x,0)},color=black] (0pt,1pt) -- (0pt,-1pt) node[below] {\footnotesize $\x$};
\draw[<->,color=black] (0,-10.43158220601634095) -- (0,10.1);
\foreach \y in {-10,-9,-8,-7,-6,-5,-4,-3,-2,-1,1,2,3,4,5,6,7,8,9,10}
\draw[shift={(0,\y)},color=black] (2pt,0pt) -- (-2pt,0pt) node[left] {\footnotesize $\y$};
%\draw[color=black] (0pt,-10pt) node[right] {\footnotesize $0$};
%\clip(-0.6129302567150502,-0.43158220601634095) rectangle (9.010648940148005,7.8783927087822985);
\end{tikzpicture}


\newpage 

\begin{tikzpicture}[line cap=round,line join=round,>=triangle 45,x=1cm,y=1cm, scale=0.78]
\draw [color=lightgray,dash pattern=on 1pt off 1pt, xstep=1cm,ystep=1cm] (-10.6,-10.4) grid (10.1,10.1);
\draw[<->,color=black] (-10.6,0) -- (10.1,0);
\foreach \x in {-10,-9,-8,-7,-6,-5,-4,-3,-2,-1,1,2,3,4,5,6,7,8,9,10}
\draw[shift={(\x,0)},color=black] (0pt,1pt) -- (0pt,-1pt) node[below] {\footnotesize $\x$};
\draw[<->,color=black] (0,-10.43158220601634095) -- (0,10.1);
\foreach \y in {-10,-9,-8,-7,-6,-5,-4,-3,-2,-1,1,2,3,4,5,6,7,8,9,10}
\draw[shift={(0,\y)},color=black] (2pt,0pt) -- (-2pt,0pt) node[left] {\footnotesize $\y$};
%\draw[color=black] (0pt,-10pt) node[right] {\footnotesize $0$};
%\clip(-0.6129302567150502,-0.43158220601634095) rectangle (9.010648940148005,7.8783927087822985);
\end{tikzpicture}

\end{document}

