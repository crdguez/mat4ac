\documentclass[addpoints,spanish, 12pt,a4paper]{exam}
%\documentclass[answers, spanish, 12pt,a4paper]{exam}
\printanswers
\pointpoints{punto}{puntos}
\hpword{Puntos:}
\vpword{Puntos:}
\htword{Total}
\vtword{Total}
\hsword{Resultado:}
\hqword{Ejercicio:}
\vqword{Ejercicio:}

\usepackage[utf8]{inputenc}
\usepackage[spanish]{babel}
\usepackage{eurosym}
%\usepackage[spanish,es-lcroman, es-tabla, es-noshorthands]{babel}

\usepackage{pgf,tikz}
\usetikzlibrary{shapes, calc, shapes, arrows, math, babel}
\usepackage[margin=1in]{geometry}
\usepackage{amsmath,amssymb}
\usepackage{multicol}

\usepackage{yhmath}

\usepackage{verbatim}
\usepackage[thinlines]{easytable}
%\usepackage{pstricks}


\usepackage{graphicx}
\graphicspath{{../img/}} 

\newcommand{\class}{4º Académicas}
\newcommand{\examdate}{\today}
\newcommand{\examnum}{Examen de geometría analítica y funciones}
\newcommand{\tipo}{C2}


\newcommand{\timelimit}{50 minutos}

\renewcommand{\solutiontitle}{\noindent\textbf{Solución:}\enspace}


\pagestyle{head}
\firstpageheader{\includegraphics[width=0.2\columnwidth]{header_left}}{\textbf{Departamento de Matemáticas\linebreak \class}\linebreak \examnum}{\includegraphics[width=0.1\columnwidth]{header_right}}
\runningheader{\class}{\examnum}{Página \thepage\ of \numpages}
\runningheadrule
\pointsinrightmargin % Para poner las puntuaciones a la derecha. Se puede cambiar. Si se comenta, sale a la izquierda.
\extrawidth{-2.4cm} %Un poquito más de margen por si ponemos textos largos.
\marginpointname{ \emph{\points}}

\begin{document}

\noindent
\begin{tabular*}{\textwidth}{l @{\extracolsep{\fill}} r @{\extracolsep{6pt}} }
\textbf{Nombre:} \makebox[3.5in]{\hrulefill} & \textbf{Fecha:}\makebox[1in]{\hrulefill} \\
 & \\
\textbf{Tiempo: \timelimit} & Tipo: \tipo 
\end{tabular*}
\rule[2ex]{\textwidth}{2pt}
Esta prueba tiene \numquestions\ ejercicios. La puntuación máxima es de \numpoints. 
La nota final de la prueba será la parte proporcional de la puntuación obtenida sobre la puntuación máxima. 

\begin{center}


\addpoints
 %\gradetable[h][questions]
	\pointtable[h][questions]
\end{center}

\textbf{ACLARACIÓN:} Los ejercicios de geometría se han de resolver de manera analítica (no gráfica). Los ejercicios de funciones deberán estar justificados con los cálculos que sean necesarios para su resolución.

\begin{center}
\rule[2ex]{\textwidth}{2pt}
%\noindent
\end{center}

\begin{questions}

\begin{comment}
\question[1] 
\begin{solution} \end{solution}

\addpoints
\end{comment}


\question Resuelve las siguientes cuestiones geométricas:
\begin{parts}

\begin{comment}
\part[1] Averigua el punto simétrico de $A(-1, -4)$ con respecto a $B(5, 0)$ 
  
\begin{solution} $(5,0)=(\frac{-1+x}{2},\frac{-4+y}{2})\to A'(11, 4)$ \end{solution}
\end{comment}

\part[1] Escribe la ecuación vectorial, paramétrica, continua, general y explícita de la recta que pasa por el punto $P(2,0)$ y tiene por vector direccional a $\overrightarrow{v}=[\overrightarrow{CD}]$, siendo $C(2,2)$ y $D(1,0)$  
  
\begin{solution}
$  \overrightarrow{d}(-1, -2) \land P \in r
$ \\
$ (-t + 2, -2t)$ \\
$ r\equiv 2x - y - 4 = 0$

\end{solution}
%\begin{comment}
\part[1] Calcula la distancia que hay entre los puntos  
$A(8,10)$ y $B(3,-2)$  

\begin{solution} $\sqrt{5^2+12^2}=13$\end{solution}
%\end{comment}

\end{parts}

\addpoints


\begin{comment}
\question Sea un paralelogramo $ABCD$ (los vértices van en setido de las agujas del reloj). Si $A(2,3)$, $B(5,1)$ y $C(3,0)$, halla:
\begin{parts}
\part[1] El vértice $D$
\begin{solution} $D(0,2) $ \end{solution}
\part[1] El perímetro del paralelogramo
\begin{solution} $ 2\cdot \left( \sqrt{13} + \sqrt{5}\right) \approx 11.683238505927559$ \end{solution}
\part[2] El área del paralelogramo
\begin{solution} Proyección de A sobre $\overline{BC} = P(12/13, 18/13)$ \\
$\sqrt{13}\cdot7\sqrt{13}/13 = 7 $
\end{solution}
\end{parts}

\addpoints

\end{comment}


	\question En el triángulo de vértices $A(-3,1)$, $B(1,5)$ y $C(4,0)$, halla:
	\begin{parts} 
\begin{comment}
	\part[2] La ecuación de la recta $h$ correspondiente a la altura trazada desde el vértice B.
	\begin{solution} $\overrightarrow{AC}(7,-1) \land A \in  r \to r \equiv x + 7y - 4 = 0 $\\$h\perp r \land B \in h \to \overrightarrow{d'}(1,7) \land B \in h \to h \equiv -7x + y + 2 = 0$ \end{solution}
\end{comment}	

	\part[1] La ecuación de la mediatriz $m$ del lado $\overline{AB}$.
	\begin{solution} $M_{AB}(\frac{-3+1}{2},\frac{1+5}{2})=(-1,3) \in m \land m \perp \overrightarrow{AB}(4,4)\to m \equiv -4x - 4y + 8 = 0$ \end{solution}

		
	\part[2] El perímetro y el área del triángulo.
	\begin{solution} Perímetro: $4 \sqrt{2}(\approx5.65685424949238) + 5 \sqrt{2}(\approx7.07106781186548) + \sqrt{34}(\approx5.8309518948453) = \sqrt{34} + 9 \sqrt{2}\approx18.5588739562032 \ ud$ \\
	Área: \\ Altura que pasa por C: $h\equiv y=4-x $ \\
	recta AB: $r \equiv y = x + 4$ \\
	$r \perp h = Q(0,4)$ \\
	$\dfrac{4\sqrt{2}\cdot 4\sqrt{2}}{2}=16 \ ud^2$
	\end{solution}
\begin{comment} 
	\part[1] El ańgulo del vértice $A$ ($\hat{A})$.
	\begin{solution} $P(9/25, 13/25) = r \cap h$ \\distancias:\\ $ \left|\overrightarrow{BP}\right|=
	\frac{16\sqrt{2}}{5}$ \\
	$\left|\overrightarrow{AP}\right|=
	\frac{12\sqrt{2}}{5} $ \\ $\hat{A}=\arctan(\frac{16\sqrt{2}/5}{12\sqrt{2}/5})= \operatorname{atan}{\left (\frac{4}{3} \right )}\approx53.130102354156
	\end{solution}
\begin{comment} 
	\part[1] El ańgulo del vértice $A$ ($\hat{A})$.
	\begin{solution} $P(9/25, 13/25) = r \cap h$ \\distancias:\\ $ \left|\overrightarrow{BP}\right|=
	\frac{16\sqrt{2}}{5}$ \\
	$\left|\overrightarrow{AP}\right|=
	\frac{12\sqrt{2}}{5} $ \\ $\hat{A}=\arctan(\frac{16\sqrt{2}/5}{12\sqrt{2}/5})= \operatorname{atan}{\left (\frac{4}{3} \right )}\approx53.130102354156$ \end{solution}
\end{comment}		
	\end{parts}
	\addpoints




%\begin{comment}
\question  Dada la siguiente función $f(x) =
\left\{
	\begin{array}{clc}
		-2x  & \mbox{si } & x < -2 \\
		x^2-2x+1 & \mbox{si } & -2 \leq x < 2 \\
		2x-3 & \mbox{si } & x > 2
	\end{array}
\right.$
\begin{parts} 
\part[2] Representa la función gráficamente 
\begin{solution} 

\begin{tikzpicture}[domain=-4.1:5.1 ,>=triangle 45, scale=0.5]

\tikzmath{
			\a = 1; \b = -2; \c = 1; 
			\v = - \b / ( 2 * \a);
			\m1 = -2; \n1 = 0;
			\m3 = 2; \n3 = -3;
			\xmin1 = - 4.1; \xmax1 = -2;
			\xmin2 = - 2; \xmax2 = 2;
			\xmin3 = 2; \xmax3 = 5.1;
          }
          
 
\draw[color=red, domain=\xmin1 -0.5:\xmax1]    plot (\x,{\m1*(\x) + \n1}) node[right] {};

\draw [red] (\xmax1,{\m1*(\xmax1) + \n1}) circle (0.25) node [left] {};

\draw[color=red, domain=\xmin2:\xmax2]    plot (\x,{\a*(\x)^2 + \b *\x + \c})             node[right] {}; 
\draw [red, fill] (\xmin2,{\a*(\xmin2)^2 + \b *\xmin2 + \c}) circle (0.25) node [left] {};
\draw [red] (\xmax2,{\a*(\xmax2)^2 + \b *\xmax2 + \c}) circle (0.25) node [left] {};

\draw[color=red, domain=\xmin3  :\xmax3 + 0.1]    plot (\x,{\m3*(\x) + \n3}) node[right] {};
\draw [red] (\xmin3,{\m3*(\xmin3) + \n3}) circle (0.25) node [left] {};

\draw[very thin,color=lightgray,dash pattern=on 1pt off 1pt] (\xmin1 - 0.5, \a * \v * \v + \b * \v + \c - 0.5) grid (\xmax3 + 0.5 , \a * \xmin2 * \xmin2 + \b * \xmin2 + \c + 0.5);

\draw[<->] (\xmin1 -1,0) -- (\xmax3 + 1,0) node[right] {$x$};
\draw[<->] (0,\a * \v * \v + \b * \v + \c - 0.5) -- (0, \a * \xmin2 * \xmin2 + \b * \xmin2 + \c + 0.5 ) node[above] {$y$};

\end{tikzpicture}  \end{solution}
\part[1] Indica el \emph{dominio} y el \emph{recorrido} de la función utilizando la notación de conjuntos de números reales
\begin{solution} $Dom(f)=\mathbb{R}-\lbrace 2 \rbrace $ \\
$ Im(f)=\left[0, +\infty\right]$
\end{solution}


\end{parts}
\addpoints
%\end{comment}

\begin{comment}
\question  Dada la siguiente función $f(x) =
\left\{
	\begin{array}{clc}
		-x -2  & \mbox{si } & x < -1 \\
		x^2-2x+1 & \mbox{si } & -1 \leq x < 2 \\
		x-2 & \mbox{si } & x > 2
	\end{array}
\right.$
\begin{parts} 
\part[2] Representa la función gráficamente 
\begin{solution} 

\begin{tikzpicture}[domain=-4.1:5.1 ,>=triangle 45, scale=0.5]

\tikzmath{
			\a = 1; \b = -2; \c = 0; 
			\v = - \b / ( 2 * \a);
			\m1 = -1; \n1 = -2;
			\m3 = 1; \n3 = -2;
			\xmin1 = - 4.1; \xmax1 = -1;
			\xmin2 = - 1; \xmax2 = 2;
			\xmin3 = 2; \xmax3 = 5.1;
          }
          
 
\draw[color=red, domain=\xmin1 -0.5:\xmax1]    plot (\x,{\m1*(\x) + \n1}) node[right] {};

\draw [red] (\xmax1,{\m1*(\xmax1) + \n1}) circle (0.25) node [left] {};

\draw[color=red, domain=\xmin2:\xmax2]    plot (\x,{\a*(\x)^2 + \b *\x + \c})             node[right] {}; 
\draw [red, fill] (\xmin2,{\a*(\xmin2)^2 + \b *\xmin2 + \c}) circle (0.25) node [left] {};
\draw [red] (\xmax2,{\a*(\xmax2)^2 + \b *\xmax2 + \c}) circle (0.25) node [left] {};

\draw[color=red, domain=\xmin3  :\xmax3 + 0.1]    plot (\x,{\m3*(\x) + \n3}) node[right] {};
\draw [red] (\xmin3,{\m3*(\xmin3) + \n3}) circle (0.25) node [left] {};

\draw[very thin,color=lightgray,dash pattern=on 1pt off 1pt] (\xmin1 - 0.5, \a * \v * \v + \b * \v + \c - 0.5) grid (\xmax3 + 0.5 , \a * \xmin2 * \xmin2 + \b * \xmin2 + \c + 0.5);

\draw[<->] (\xmin1 -1,0) -- (\xmax3 + 1,0) node[right] {$x$};
\draw[<->] (0,\a * \v * \v + \b * \v + \c - 0.5) -- (0, \a * \xmin2 * \xmin2 + \b * \xmin2 + \c + 0.5 ) node[above] {$y$};

\end{tikzpicture}  \end{solution}
\part[1] Indica el \emph{dominio} y el \emph{recorrido} de la función utilizando la notación de conjuntos de números reales
\begin{solution} $Dom(f)=\mathbb{R}-\lbrace 2 \rbrace $ \\
$ Im(f)=\left[-1, +\infty\right]$
\end{solution}


\end{parts}
\addpoints
\end{comment}

%\begin{comment}
\question Dada la función $f(x)=\left|2x+4\right|$
\begin{parts} 
\part[1] Transforma la función a una función a trozos equivalente
\begin{solution} $f(x) =
\left\{
	\begin{array}{clc}
		-(2x +4)   & \mbox{si } & x < -2 \\
		2x +4   & \mbox{si } & x \geq -2
	\end{array}
\right.$
\end{solution}
\part[1] Representa la función del apartado anterior gráficamente 
\begin{solution} 

\begin{tikzpicture}[domain=-4.1:3.1 ,>=triangle 45, scale=0.35]

\tikzmath{
			\a = 1; \b = -2; \c = 1; 
			\v = - \b / ( 2 * \a);
			\m1 = -2; \n1 = -4;
			\m3 = 2; \n3 = 4;
			\xmin1 = - 4.1; \xmax1 = -2;
			\xmin2 = - 2; \xmax2 = -2;
			\xmin3 = -2; \xmax3 = 3.1;
          }
          
 
\draw[color=red, domain=\xmin1 -0.5:\xmax1]    plot (\x,{\m1*(\x) + \n1}) node[right] {};

%\draw [red] (\xmax1,{\m1*(\xmax1) + \n1}) circle (0.25) node [left] {};

%\draw[color=red, domain=\xmin2:\xmax2]    plot (\x,{\a*(\x)^2 + \b *\x + \c})             node[right] {}; 
%\draw [red, fill] (\xmin2,{\a*(\xmin2)^2 + \b *\xmin2 + \c}) circle (0.25) node [left] {};
%\draw [red] (\xmax2,{\a*(\xmax2)^2 + \b *\xmax2 + \c}) circle (0.25) node [left] {};

\draw[color=red, domain=\xmin3  :\xmax3 + 0.1]    plot (\x,{\m3*(\x) + \n3}) node[right] {};
%\draw [red] (\xmin3,{\m3*(\xmin3) + \n3}) circle (0.25) node [left] {};

\draw[very thin,color=lightgray,dash pattern=on 1pt off 1pt] (\xmin1 - 0.5, \a * \v * \v + \b * \v + \c - 0.5) grid (\xmax3 + 0.5 , \a * \xmin2 * \xmin2 + \b * \xmin2 + \c + 0.5);

\draw[<->] (\xmin1 -1,0) -- (\xmax3 + 1,0) node[right] {$x$};
\draw[<->] (0,\a * \v * \v + \b * \v + \c - 0.5) -- (0, \a * \xmin2 * \xmin2 + \b * \xmin2 + \c + 0.5 ) node[above] {$y$};

\end{tikzpicture}  \end{solution}
\part[1] Indica el \emph{dominio} y el \emph{recorrido} de la función utilizando la notación de conjuntos de números reales
\begin{solution} $Dom(f)=\mathbb{R} $ \\
$ Im(f)=\left[0, +\infty\right]$
\end{solution}
\end{parts}
%\end{comment}


\begin{comment}
\question Dada la función $f(x)=\left|2x+2\right|$
\begin{parts} 
\part[1] Transforma la función a una función a trozos equivalente
\begin{solution} $f(x) =
\left\{
	\begin{array}{clc}
		-(2x +2)   & \mbox{si } & x < -1 \\
		2x +2   & \mbox{si } & x \geq -1
	\end{array}
\right.$
\end{solution}
\part[1] Representa la función del apartado anterior gráficamente 
\begin{solution} 

\begin{tikzpicture}[domain=-4.1:3.1 ,>=triangle 45, scale=0.35]

\tikzmath{
			\a = 1; \b = -2; \c = 1; 
			\v = - \b / ( 2 * \a);
			\m1 = -2; \n1 = -2;
			\m3 = 2; \n3 = 2;
			\xmin1 = - 4.1; \xmax1 = -1;
			\xmin2 = - 1; \xmax2 = -1;
			\xmin3 = -1; \xmax3 = 3.1;
          }
          
 
\draw[color=red, domain=\xmin1 -0.5:\xmax1]    plot (\x,{\m1*(\x) + \n1}) node[right] {};

%\draw [red] (\xmax1,{\m1*(\xmax1) + \n1}) circle (0.25) node [left] {};

%\draw[color=red, domain=\xmin2:\xmax2]    plot (\x,{\a*(\x)^2 + \b *\x + \c})             node[right] {}; 
%\draw [red, fill] (\xmin2,{\a*(\xmin2)^2 + \b *\xmin2 + \c}) circle (0.25) node [left] {};
%\draw [red] (\xmax2,{\a*(\xmax2)^2 + \b *\xmax2 + \c}) circle (0.25) node [left] {};

\draw[color=red, domain=\xmin3  :\xmax3 + 0.1]    plot (\x,{\m3*(\x) + \n3}) node[right] {};
%\draw [red] (\xmin3,{\m3*(\xmin3) + \n3}) circle (0.25) node [left] {};

\draw[very thin,color=lightgray,dash pattern=on 1pt off 1pt] (\xmin1 - 0.5, \a * \v * \v + \b * \v + \c - 0.5) grid (\xmax3 + 0.5 , \a * \xmin2 * \xmin2 + \b * \xmin2 + \c + 0.5);

\draw[<->] (\xmin1 -1,0) -- (\xmax3 + 1,0) node[right] {$x$};
\draw[<->] (0,\a * \v * \v + \b * \v + \c - 0.5) -- (0, \a * \xmin2 * \xmin2 + \b * \xmin2 + \c + 0.5 ) node[above] {$y$};

\end{tikzpicture}  \end{solution}
\part[1] Indica el \emph{dominio} y el \emph{recorrido} de la función utilizando la notación de conjuntos de números reales
\begin{solution} $Dom(f)=\mathbb{R} $ \\
$ Im(f)=\left[0, +\infty\right]$
\end{solution}
\end{parts}
\end{comment}

\end{questions}

\newpage 

\begin{tikzpicture}[line cap=round,line join=round,>=triangle 45,x=1cm,y=1cm, scale=0.78]
\draw [color=lightgray,dash pattern=on 1pt off 1pt, xstep=1cm,ystep=1cm] (-10.6,-10.4) grid (10.1,10.1);
\draw[<->,color=black] (-10.6,0) -- (10.1,0);
\foreach \x in {-10,-9,-8,-7,-6,-5,-4,-3,-2,-1,1,2,3,4,5,6,7,8,9,10}
\draw[shift={(\x,0)},color=black] (0pt,1pt) -- (0pt,-1pt) node[below] {\footnotesize $\x$};
\draw[<->,color=black] (0,-10.43158220601634095) -- (0,10.1);
\foreach \y in {-10,-9,-8,-7,-6,-5,-4,-3,-2,-1,1,2,3,4,5,6,7,8,9,10}
\draw[shift={(0,\y)},color=black] (2pt,0pt) -- (-2pt,0pt) node[left] {\footnotesize $\y$};
%\draw[color=black] (0pt,-10pt) node[right] {\footnotesize $0$};
%\clip(-0.6129302567150502,-0.43158220601634095) rectangle (9.010648940148005,7.8783927087822985);
\end{tikzpicture}

\newpage 

\begin{tikzpicture}[line cap=round,line join=round,>=triangle 45,x=1cm,y=1cm, scale=0.78]
\draw [color=lightgray,dash pattern=on 1pt off 1pt, xstep=1cm,ystep=1cm] (-10.6,-10.4) grid (10.1,10.1);
\draw[<->,color=black] (-10.6,0) -- (10.1,0);
\foreach \x in {-10,-9,-8,-7,-6,-5,-4,-3,-2,-1,1,2,3,4,5,6,7,8,9,10}
\draw[shift={(\x,0)},color=black] (0pt,1pt) -- (0pt,-1pt) node[below] {\footnotesize $\x$};
\draw[<->,color=black] (0,-10.43158220601634095) -- (0,10.1);
\foreach \y in {-10,-9,-8,-7,-6,-5,-4,-3,-2,-1,1,2,3,4,5,6,7,8,9,10}
\draw[shift={(0,\y)},color=black] (2pt,0pt) -- (-2pt,0pt) node[left] {\footnotesize $\y$};
%\draw[color=black] (0pt,-10pt) node[right] {\footnotesize $0$};
%\clip(-0.6129302567150502,-0.43158220601634095) rectangle (9.010648940148005,7.8783927087822985);
\end{tikzpicture}

\newpage 

\begin{tikzpicture}[line cap=round,line join=round,>=triangle 45,x=1cm,y=1cm, scale=0.78]
\draw [color=lightgray,dash pattern=on 1pt off 1pt, xstep=1cm,ystep=1cm] (-10.6,-10.4) grid (10.1,10.1);
\draw[<->,color=black] (-10.6,0) -- (10.1,0);
\foreach \x in {-10,-9,-8,-7,-6,-5,-4,-3,-2,-1,1,2,3,4,5,6,7,8,9,10}
\draw[shift={(\x,0)},color=black] (0pt,1pt) -- (0pt,-1pt) node[below] {\footnotesize $\x$};
\draw[<->,color=black] (0,-10.43158220601634095) -- (0,10.1);
\foreach \y in {-10,-9,-8,-7,-6,-5,-4,-3,-2,-1,1,2,3,4,5,6,7,8,9,10}
\draw[shift={(0,\y)},color=black] (2pt,0pt) -- (-2pt,0pt) node[left] {\footnotesize $\y$};
%\draw[color=black] (0pt,-10pt) node[right] {\footnotesize $0$};
%\clip(-0.6129302567150502,-0.43158220601634095) rectangle (9.010648940148005,7.8783927087822985);
\end{tikzpicture}


\end{document}

