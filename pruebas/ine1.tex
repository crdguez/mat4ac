\documentclass[addpoints,spanish, 12pt,a4paper]{exam}
%\documentclass[answers, spanish, 12pt,a4paper]{exam}
\printanswers
\pointpoints{punto}{puntos}
\hpword{Puntos:}
\vpword{Puntos:}
\htword{Total}
\vtword{Total}
\hsword{Resultado:}
\hqword{Ejercicio:}
\vqword{Ejercicio:}

\usepackage[utf8]{inputenc}
\usepackage[spanish]{babel}
\usepackage{eurosym}
%\usepackage[spanish,es-lcroman, es-tabla, es-noshorthands]{babel}


\usepackage[margin=1in]{geometry}
\usepackage{amsmath,amssymb}
\usepackage{multicol}
\usepackage{yhmath}

\usepackage{verbatim}
%\usepackage{pstricks}


\usepackage{graphicx}
\graphicspath{{../img/}} 

\newcommand{\class}{4º Académicas}
\newcommand{\examdate}{\today}
\newcommand{\examnum}{Sistemas de ecuacione e inecuaciones}
\newcommand{\tipo}{A}


\newcommand{\timelimit}{50 minutos}

\renewcommand{\solutiontitle}{\noindent\textbf{Solución:}\enspace}


\pagestyle{head}
\firstpageheader{\includegraphics[width=0.2\columnwidth]{header_left}}{\textbf{Departamento de Matemáticas\linebreak \class}\linebreak \examnum}{\includegraphics[width=0.1\columnwidth]{header_right}}
\runningheader{\class}{\examnum}{Página \thepage\ of \numpages}
\runningheadrule
\pointsinrightmargin % Para poner las puntuaciones a la derecha. Se puede cambiar. Si se comenta, sale a la izquierda.
\extrawidth{-2.4cm} %Un poquito más de margen por si ponemos textos largos.
\marginpointname{ \emph{\points}}

\begin{document}

\noindent
\begin{tabular*}{\textwidth}{l @{\extracolsep{\fill}} r @{\extracolsep{6pt}} }
\textbf{Nombre:} \makebox[3.5in]{\hrulefill} & \textbf{Fecha:}\makebox[1in]{\hrulefill} \\
 & \\
\textbf{Tiempo: \timelimit} & Tipo: \tipo 
\end{tabular*}
\rule[2ex]{\textwidth}{2pt}
Esta prueba tiene \numquestions\ ejercicios. La puntuación máxima es de \numpoints. 
La nota final de la prueba será la parte proporcional de la puntuación obtenida sobre la puntuación máxima. Para la recuperación de pendientes de 3º se tendrán en cuenta los apartados: 1. 2.a y 4.a ?

\begin{center}


\addpoints
 %\gradetable[h][questions]
	\pointtable[h][questions]
\end{center}

\noindent
\rule[2ex]{\textwidth}{2pt}

\begin{questions}

\begin{comment}
\question[1] 
\begin{solution} \end{solution}

\addpoints
\end{comment}

\question[1] Resuelve por el método que quieras:
$$\left. \begin{gathered}
	  \frac{{x + y}}{2} - \frac{{x - y}}{2} = 2 \hfill \\
	  5x - 10y = 40 \hfill \\ 
	\end{gathered}  \right\rbrace$$
	\begin{solution}  x=12; y=2 \end{solution}

\question Cuánto vale el área de un rectángulo sabiendo que su diagonal mide 13 m y su perímetro es 34 m.
\begin{parts}
\part[1] Traduce a lenguaje algebraico el enunciado anterior
\begin{solution}  $\left. \begin{gathered}
	  2x+2y=34 \hfill \\
	  x^2+y^2=169 \hfill \\ 
	\end{gathered}  \right\rbrace$ \end{solution}
\part[1] Resuelve la expresión del apartado anterior indicando cuántas soluciones hay
\begin{solution}  
%solve([2*x+2*y-34,x**2+y**2-169],[x,y])
$\left ( 5, \quad 12\right ), \quad \left ( 12, \quad 5\right)$ \end{solution}
\end{parts}

\question Resuelve las siguientes inecuaciones:
\begin{parts}

\part[2]$ 2x^2 - 4x - 6 \geqslant 0  $  
\begin{solution} $ \left(-\infty, -1\right] \cup \left[3, \infty\right)$ \end{solution}

\part[2]$x^{3} - 5 x^{2} + 6 x < 0$  
% solve_univariate_inequality (expand(x*(x-2)*(x-3))<0,x,relational =false )
\begin{solution} $\left(-\infty, 0\right) \cup \left(2, 3\right)$ \end{solution}

\part[2]$\dfrac{x^{2} - x}{x + 1}\geqslant 0$  
%from sympy.solvers.inequalities import reduce_rational_inequalities
%reduce_rational_inequalities([[(2*x-2)/(1-3*x) < -2/3]], x,relational=0)
%reduce_rational_inequalities([[expand(x*(x-1))/expand((x+1)**1) >= 0]], x,relational=0)
\begin{solution} $\left(-1, 0\right] \cup \left[1, \infty\right)$ \end{solution}

\part[2]  $\left| {2x - 12} \right| > 2$ 
%reduce_abs_inequality(Abs(2*x - 12) - 2, '>', x)
\begin{solution} $\left(-\infty, 5\right) \cup \left(7, \infty\right) $ \end{solution}

\end{parts}

\addpoints

\question En  un  examen  de  40  preguntas  te  dan  dos puntos  por  cada  acierto  y  te  restan  0,5  puntos  por  cada fallo. ¿Cuántas preguntas hay que contestar bien para obtener como mínimo 40 puntos, si es obligatorio responder a todas?
\begin{parts} 
\part[2] Traduce a lenguaje algebraico el enunciado anterior
\begin{solution} $\left. \begin{gathered}
	  2x - \frac{y}{2} \geqslant 40 \hfill \\
	  x + y = 40 \hfill
	 \end{gathered}  \right\rbrace \to 2x-\left(40-x\right)\cdot 0,5 \geqslant 40$ \end{solution}
\part[2] Resuelve la expresión del apartado anterior e indica cuáles son las soluciones
\begin{solution} 24 o más \end{solution}

\end{parts}
\addpoints

\end{questions}

\end{document}
\grid
