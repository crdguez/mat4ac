\documentclass[addpoints,spanish, 12pt,a4paper]{exam}
%\documentclass[answers, spanish, 12pt,a4paper]{exam}
%\printanswers
\pointpoints{punto}{puntos}
\hpword{Puntos:}
\vpword{Puntos:}
\htword{Total}
\vtword{Total}
\hsword{Resultado:}
\hqword{Ejercicio:}
\vqword{Ejercicio:}

\usepackage[utf8]{inputenc}
\usepackage[spanish]{babel}
\usepackage{eurosym}
%\usepackage[spanish,es-lcroman, es-tabla, es-noshorthands]{babel}


\usepackage[margin=1in]{geometry}
\usepackage{amsmath,amssymb}
\usepackage{multicol}

\usepackage{yhmath}

\usepackage{verbatim}
\usepackage[thinlines]{easytable}
%\usepackage{pstricks}


\usepackage{graphicx}
\graphicspath{{../img/}} 

\newcommand{\class}{4º Académicas}
\newcommand{\examdate}{\today}
\newcommand{\examnum}{Examen de Recuperación}
\newcommand{\tipo}{B}


\newcommand{\timelimit}{50 minutos}

\renewcommand{\solutiontitle}{\noindent\textbf{Solución:}\enspace}


\pagestyle{head}
\firstpageheader{\includegraphics[width=0.2\columnwidth]{header_left}}{\textbf{Departamento de Matemáticas\linebreak \class}\linebreak \examnum}{\includegraphics[width=0.1\columnwidth]{header_right}}
\runningheader{\class}{\examnum}{Página \thepage\ of \numpages}
\runningheadrule
\pointsinrightmargin % Para poner las puntuaciones a la derecha. Se puede cambiar. Si se comenta, sale a la izquierda.
\extrawidth{-2.4cm} %Un poquito más de margen por si ponemos textos largos.
\marginpointname{ \emph{\points}}

\begin{document}

\noindent
\begin{tabular*}{\textwidth}{l @{\extracolsep{\fill}} r @{\extracolsep{6pt}} }
\textbf{Nombre:} \makebox[3.5in]{\hrulefill} & \textbf{Fecha:}\makebox[1in]{\hrulefill} \\
 & \\
\textbf{Tiempo: \timelimit} & Tipo: \tipo 
\end{tabular*}
\rule[2ex]{\textwidth}{2pt}
Esta prueba tiene \numquestions\ ejercicios. La puntuación máxima es de \numpoints. 
La nota final de la prueba será la parte proporcional de la puntuación obtenida sobre la puntuación máxima. 

\begin{center}


\addpoints
 %\gradetable[h][questions]
	\pointtable[h][questions]
\end{center}

\noindent
\rule[2ex]{\textwidth}{2pt}

\begin{questions}

\begin{comment}
\question[1] 
\begin{solution} \end{solution}

\addpoints
\end{comment}


\question Resuelve las siguientes inecuaciones de manera justificada:
\begin{parts}

\part[1]$ x^{3}  + x <  2 x^{2} $
%solve_univariate_inequality (expand(x*(x-1)*(x-1))<0,x,relational =false )  
\begin{solution} $\left(-\infty, 0\right)$ \end{solution}

\part[1]$\dfrac{2x-2}{1-3x}<-\dfrac{2}{3} $  
%from sympy.solvers.inequalities import reduce_rational_inequalities
%reduce_rational_inequalities([[(2*x-2)/(1-3*x) < -2/3]], x,relational=0)
\begin{solution} $\left(-\infty, \frac{1}{3}\right)$\end{solution}


\end{parts}

\addpoints

\question[2] Halla sin calculadora y con radicales el valor de las siguientes expresiones:
\begin{multicols}{2}
\begin{parts}
\part $\dfrac{\sen 60 - \sen 30}{\sen 60 + \sen 30}$   
%(sin(rad(60))-sin(rad(30)))/(sin(rad(60))+sin(rad(30)))
\begin{solution} $- \sqrt{3} + 2\approx0.267949192431123$\end{solution}
\part $\dfrac{\cos 45 - \sen 30}{\sen 45 + \cos 60}$     
\begin{solution} $- 2 \sqrt{2} + 3\approx0.17157287525381$\end{solution}

\end{parts}
\end{multicols}

\question[2] Para  hallar  el  ancho  de  un  rio  procedemos  así:  Nos  situamos  en  un  punto  A,  en  una  orilla 
del  rio,  y 
medimos  el  ángulo  (60º)  bajo
el  cual  se  ve  un  árbol  que  está  frente  a  nosotros,  en  la  otra  orilla.  Nos 
alejamos 20 m de la orilla en dirección perpendicular a ella y volvemos a medir el ángulo
bajo el cual se ve 
el árbol, 30
º. ¿Cuánto mide el ancho del rio? 
 
\begin{solution} $\sen \left(\frac{\pi}{2}+ \alpha\right)= \cos \alpha=\frac{12}{13}\approx0.923076923076923 \\
\cos \left(\frac{\pi}{2}+ \alpha\right)= -\sen \alpha=-\frac{5}{13}\\
\tg \left(\frac{\pi}{2}+ \alpha\right)= -\cotg \alpha=-\frac{5}{12}-\approx0.416666666666667$ \end{solution}	 


\addpoints


\question[2] Calcula el área de un decágono regular de 5 cm de lado.
\begin{solution}$ ap = \frac{2.5}{\tg18}\approx7.69420884293813 cm \\ A=\dfrac{10\cdot5\cdot\frac{2.5}{\tg18}}{2}\approx192.355221073453 cm^2 $ \end{solution}


\end{questions}

\end{document}
\grid
