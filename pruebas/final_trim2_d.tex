\documentclass[addpoints,spanish, 12pt,a4paper]{exam}
%\documentclass[answers, spanish, 12pt,a4paper]{exam}
%\printanswers
\pointpoints{punto}{puntos}
\hpword{Puntos:}
\vpword{Puntos:}
\htword{Total}
\vtword{Total}
\hsword{Resultado:}
\hqword{Ejercicio:}
\vqword{Ejercicio:}

\usepackage[utf8]{inputenc}
\usepackage[spanish]{babel}
\usepackage{eurosym}
%\usepackage[spanish,es-lcroman, es-tabla, es-noshorthands]{babel}


\usepackage[margin=1in]{geometry}
\usepackage{amsmath,amssymb}
\usepackage{multicol}

\usepackage{yhmath}

\usepackage{verbatim}
\usepackage[thinlines]{easytable}
%\usepackage{pstricks}


\usepackage{graphicx}
\graphicspath{{../img/}} 

\newcommand{\class}{4º Académicas}
\newcommand{\examdate}{\today}
\newcommand{\examnum}{Examen de final de trimestre}
\newcommand{\tipo}{D}


\newcommand{\timelimit}{50 minutos}

\renewcommand{\solutiontitle}{\noindent\textbf{Solución:}\enspace}


\pagestyle{head}
\firstpageheader{\includegraphics[width=0.2\columnwidth]{header_left}}{\textbf{Departamento de Matemáticas\linebreak \class}\linebreak \examnum}{\includegraphics[width=0.1\columnwidth]{header_right}}
\runningheader{\class}{\examnum}{Página \thepage\ of \numpages}
\runningheadrule
\pointsinrightmargin % Para poner las puntuaciones a la derecha. Se puede cambiar. Si se comenta, sale a la izquierda.
\extrawidth{-2.4cm} %Un poquito más de margen por si ponemos textos largos.
\marginpointname{ \emph{\points}}

\begin{document}

\noindent
\begin{tabular*}{\textwidth}{l @{\extracolsep{\fill}} r @{\extracolsep{6pt}} }
\textbf{Nombre:} \makebox[3.5in]{\hrulefill} & \textbf{Fecha:}\makebox[1in]{\hrulefill} \\
 & \\
\textbf{Tiempo: \timelimit} & Tipo: \tipo 
\end{tabular*}
\rule[2ex]{\textwidth}{2pt}
Esta prueba tiene \numquestions\ ejercicios. La puntuación máxima es de \numpoints. 
La nota final de la prueba será la parte proporcional de la puntuación obtenida sobre la puntuación máxima. 

\begin{center}


\addpoints
 %\gradetable[h][questions]
	\pointtable[h][questions]
\end{center}

\noindent
\rule[2ex]{\textwidth}{2pt}

\begin{questions}

\begin{comment}
\question[1] 
\begin{solution} \end{solution}

\addpoints
\end{comment}


\question Resuelve las siguientes inecuaciones de manera justificada:
\begin{parts}

\part[1]$ x^{3}  + x <  2 x^{2} $
%solve_univariate_inequality (expand(x*(x-1)*(x-1))<0,x,relational =false )  
\begin{solution} $\left(-\infty, 0\right)$ \end{solution}


\part[2]$\dfrac{x - 1}{x^{2} + x}\geqslant 0$  
%from sympy.solvers.inequalities import reduce_rational_inequalities
%reduce_rational_inequalities([[expand((x-1))/expand(x*(x+1)) >= 0]], x,relational=0)
\begin{solution} $\left(-1, 0\right) \cup \left[1, \infty\right)$ \end{solution}


\end{parts}

\addpoints

\question[1] Comprueba, usando el teorema de Pitágoras, que el triángulo de lados 6 cm, 8 cm y 10 cm es rectángulo y calcula las razones trigonométricas de sus dos ángulos agudos.

\begin{solution}
$10^2= 8^2+6^2$ \\
$\sen \alpha = \frac{8}{10} \ \cos \alpha = \frac{6}{10} \ \tg \alpha = \frac{8}{6}$\\
$\cos \beta = \frac{6}{10} \ \sen \beta = \frac{8}{10} \ \tg \beta = \frac{6}{8}$
\end{solution}


\question Si $\cos \alpha = \frac{5}{13}$:
\begin{parts} 
\part[2] Calcula el resto de las razones trigonométricas (seno y tangente) usando las relaciones trigonométricas fundamenteles y sabiendo que $\alpha \in I$ (primer cuadrante)
\begin{solution} $\sen \alpha = \sqrt{1-\left(\frac{5}{13}\right)^2}=\sqrt{\frac{144}{169}}=\frac{12}{13} \to \tg \alpha = \frac{12}{5}$ \end{solution}
\part[1] Utilizando el apartado anterior calcula las razones trigonométricas (seno, coseno y tangente)  del ángulo $\left(\frac{\pi}{2}+ \alpha\right)$ 
\begin{solution} $\sen \left(\frac{\pi}{2}+ \alpha\right)= \cos \alpha=\frac{12}{13} \\
\cos \left(\frac{\pi}{2}+ \alpha\right)= -\sen \alpha=-\frac{5}{13}\\
\tg \left(\frac{\pi}{2}+ \alpha\right)= -\cotg \alpha=-\frac{12}{5}$ \end{solution}	 


\end{parts}
\addpoints

\question[1]  Calcula la altura de una torre sabiendo que su sombra mide 13 m cuando los rayos del
sol forman un ángulo de 50º con el suelo.
\begin{solution} $\tg 50 = \dfrac{x}{13} \to x=13\cdot\tg 50\approx15.4927967037247m$ \end{solution}

\question[2]   Una antena de radio está sujeta al suelo con dos cables, que forman con la antena ángulos de 30º y 45º. Los puntos de sujeción de los cables están alineados  con el pie de la antena  y distan entre sí 98 m.
Calcula la altura de la antena y la longitud de los cables.
\begin{solution} $\left. \begin{gathered}
	  \tg 30 = \frac{y}{x} \\
	  \tg 45 = \frac{y}{98-x} \hfill
	 \end{gathered}  \right\rbrace \to \\
	 x=\frac{98\tg45}{\tg45+tg30}\approx62.129510429125 m\\
	 y=\frac{98\tg45\tg30}{\tg45+tg30}\approx35.870489570875 m\\
	 x_1=\frac{y}{\sen30}\approx71.74097914175 m \\
	 x_2=\frac{y}{\sen45}\approx50.7285328400941m$\end{solution}


\end{questions}

\end{document}
\grid
