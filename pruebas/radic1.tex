\documentclass[addpoints,spanish, 12pt,a4paper]{exam}
%\documentclass[answers, spanish, 12pt,a4paper]{exam}
%\printanswers
\pointpoints{punto}{puntos}
\hpword{Puntos:}
\vpword{Puntos:}
\htword{Total}
\vtword{Total}
\hsword{Resultado:}
\hqword{Ejercicio:}
\vqword{Ejercicio:}

\usepackage[utf8]{inputenc}
\usepackage[spanish]{babel}
\usepackage{eurosym}
%\usepackage[spanish,es-lcroman, es-tabla, es-noshorthands]{babel}


\usepackage[margin=1in]{geometry}
\usepackage{amsmath,amssymb}
\usepackage{multicol}
\usepackage{yhmath}



\usepackage{graphicx}
\graphicspath{{../img/}} 

\newcommand{\class}{Matemáticas 4º Académicas}
\newcommand{\examdate}{\today}
\newcommand{\examnum}{Examen de potencias, radicales y logaritmos}
\newcommand{\tipo}{A}


\newcommand{\timelimit}{50 minutos}

\renewcommand{\solutiontitle}{\noindent\textbf{Solución:}\enspace}


\pagestyle{head}
\firstpageheader{\includegraphics[width=0.2\columnwidth]{header_left}}{\textbf{Departamento de Matemáticas\linebreak \class}\linebreak \examnum}{\includegraphics[width=0.1\columnwidth]{header_right}}
\runningheader{\class}{\examnum}{Página \thepage\ of \numpages}
\runningheadrule


\begin{document}

\noindent
\begin{tabular*}{\textwidth}{l @{\extracolsep{\fill}} r @{\extracolsep{6pt}} }
\textbf{Nombre:} \makebox[3.5in]{\hrulefill} & \textbf{Fecha:}\makebox[1in]{\hrulefill} \\
 & \\
\textbf{Tiempo: \timelimit} & Tipo: \tipo 
\end{tabular*}
\rule[2ex]{\textwidth}{2pt}
Esta prueba tiene \numquestions\ ejercicios. La puntuación máxima es de \numpoints. 
La nota final de la prueba será la parte proporcional de la puntuación obtenida sobre la puntuación máxima.

\begin{center}


\addpoints
 %\gradetable[h][questions]
	\pointtable[h][questions]
\end{center}

\noindent
\rule[2ex]{\textwidth}{2pt}

\begin{questions}

\question[2] Indica a cuáles de los conjuntos
$\mathbb{N}$, $\mathbb{Z}$, $\mathbb{Q}$, $\mathbb{R}$ pertenecen cada uno de los siguientes números:
\begin{center}
\begin{tabular}{|c |c |c |c |c|}\hline
&$\mathbb{N}$& $\mathbb{Z}$& $\mathbb{Q}$&$\mathbb{R}$\\ 
\hline
$\frac{8}{16}$&&&&\\
\hline
$\sqrt[3]{-27}$&&&&\\
\hline
$3.0\wideparen{1}$&&&&\\
\hline
$-\frac{12}{4}$&&&&\\
\hline
$-\sqrt{25}$&&&&\\
\hline
$\sqrt{8}$&&&&\\
\hline
$4$&&&&\\
\hline
$\pi$&&&&\\
\hline
$\sqrt{-4}$&&&&\\
\hline
$\frac{39}{13}$&&&&\\
\hline
\end{tabular}

\end{center}

\addpoints

\question[1] Representa en la recta real y en forma de intervalo el siguiente conjunto numérico:
\addpoints % to omit double points count
$$\left\{ x \in \mathbb{R} \left| -2 \leqslant x < 4 \right. \right\}$$

\begin{solution}
$ $ 
\end{solution}

\question[4] Opera:
\noaddpoints % to omit double points count

\begin{parts}
\part[2] \[\dfrac{\left(2^3\cdot3^2\cdot5\right)^{-4}}{\left(2^{-2}\cdot3^{-3}\right)^{3}}\]
\begin{solution}
el 27.m $ $ 
\end{solution}
\part[2] \[\dfrac{9xy^3z^2}{14x^0yz^3}:\dfrac{18x^2yz^2}{21xy^3z}\]
\begin{solution}
el 21.x $ $ 
\end{solution}

\part[2] \[\left(\dfrac{6p^3d^2}{5q}\right)^4\cdot\left(\dfrac{20p^2q^3}{24d}\right)^4\]
\begin{solution}
el 22 j $ $
\end{solution}
\end{parts}

\addpoints

\question Expresa en notación científica, opera y simplifica:
\begin{multicols}{2}
\begin{parts}
\part[] $\dfrac{0'0001\cdot0'01\cdot10000}{0'1\cdot100\cdot0'01}$
\begin{solution} $10^{-1}$ \end{solution}
\part[] $\dfrac{0'2\cdot100\cdot1000}{8000\cdot0'1\cdot10000}$
\begin{solution} $2'5\cdot 10^{-3}$ \end{solution}
\part[] $\dfrac{1000\cdot12000\cdot0'02\cdot0'01}{400\cdot0'00003}$
\begin{solution} $2\cdot10^6$ \end{solution}
\part[] $\dfrac{0'0012\cdot0'002\cdot100000}{8000\cdot0'0003\cdot0'01}$
\begin{solution} $10$ \end{solution}
\end{parts}
\end{multicols} 
 
\question[4] Opera y simplifica
\noaddpoints % to omit double points count

\begin{parts}
\part[2] \[\dfrac{\left(3\sqrt{2}+\sqrt{3}\right)^2}{3}\]
\begin{solution}
$ $
\end{solution}

\part[2] \[\dfrac{10}{2\sqrt{3}-\sqrt{2}}\]
\begin{solution}
$ $
\end{solution}
\end{parts}

\addpoints

\question[4] Opera y simplifica cada una de estas expresiones:
\noaddpoints % to omit double points count

\begin{parts}
\part[1] \[4\sqrt{20}-3\sqrt{45}+11\sqrt{125}-20\sqrt{5}\]
\begin{solution}
$ $
\end{solution}

\part[1] \[\sqrt{72}\cdot3\sqrt{8}\]
\begin{solution}
$ $
\end{solution}

\part[1] \[\sqrt{2\sqrt{2\sqrt{2}}}\]
\begin{solution}
$ $
\end{solution}

\part[1] \[\sqrt[4]{\frac{25}{9}\sqrt[3]{\frac{9}{25}}}\]
\begin{solution}
$ $
\end{solution}

\part[1] \[\dfrac{\sqrt{45}+\sqrt{180}}{\sqrt{176}+4\sqrt{44}}\]
\begin{solution}
el 55.k $ $
\end{solution}

\part[1] \[\dfrac{\sqrt{45}+\sqrt{180}}{\sqrt{176}+4\sqrt{44}}\]
\begin{solution}
el 55.k $ $
\end{solution}


\end{parts}
\addpoints


\question[4] Calcula el valor de la $x$:
\noaddpoints % to omit double points count

\begin{parts}
\part[2] \[\log x=4\log a+3\log b-2\log c\]
\begin{solution}
$ $
\end{solution}

\part[2] \[3^{x}+3^{1-x}=4\]
\begin{solution}
$ $
\end{solution}
\end{parts}

\addpoints






\end{questions}

\end{document}
\grid
