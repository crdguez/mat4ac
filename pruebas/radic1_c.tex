\documentclass[addpoints,spanish, 12pt,a4paper]{exam}
%\documentclass[answers, spanish, 12pt,a4paper]{exam}
\printanswers
\pointpoints{punto}{puntos}
\hpword{Puntos:}
\vpword{Puntos:}
\htword{Total}
\vtword{Total}
\hsword{Resultado:}
\hqword{Ejercicio:}
\vqword{Ejercicio:}

\usepackage[utf8]{inputenc}
\usepackage[spanish]{babel}
\usepackage{eurosym}
%\usepackage[spanish,es-lcroman, es-tabla, es-noshorthands]{babel}


\usepackage[margin=1in]{geometry}
\usepackage{amsmath,amssymb}
\usepackage{multicol}
\usepackage{yhmath}



\usepackage{graphicx}
\graphicspath{{../img/}} 

\newcommand{\class}{4º Académicas}
\newcommand{\examdate}{\today}
\newcommand{\examnum}{Examen de potencias, radicales y logaritmos}
\newcommand{\tipo}{C}


\newcommand{\timelimit}{50 minutos}

\renewcommand{\solutiontitle}{\noindent\textbf{Solución:}\enspace}


\pagestyle{head}
\firstpageheader{\includegraphics[width=0.2\columnwidth]{header_left}}{\textbf{Departamento de Matemáticas\linebreak \class}\linebreak \examnum}{\includegraphics[width=0.1\columnwidth]{header_right}}
\runningheader{\class}{\examnum}{Página \thepage\ of \numpages}
\runningheadrule


\begin{document}

\noindent
\begin{tabular*}{\textwidth}{l @{\extracolsep{\fill}} r @{\extracolsep{6pt}} }
\textbf{Nombre:} \makebox[3.5in]{\hrulefill} & \textbf{Fecha:}\makebox[1in]{\hrulefill} \\
 & \\
\textbf{Tiempo: \timelimit} & Tipo: \tipo 
\end{tabular*}
\rule[2ex]{\textwidth}{2pt}
Esta prueba tiene \numquestions\ ejercicios. La puntuación máxima es de \numpoints. 
La nota final de la prueba será la parte proporcional de la puntuación obtenida sobre la puntuación máxima. Para la recuperación de pendientes de 3º se tendrán en cuenta los apartados: 3.a, 3.b, 4, 5.a, 5.b, 5.c

\begin{center}


\addpoints
 %\gradetable[h][questions]
	\pointtable[h][questions]
\end{center}

\noindent
\rule[2ex]{\textwidth}{2pt}

\begin{questions}

\question[2] Indica a cuáles de los conjuntos
$\mathbb{N}$, $\mathbb{Z}$, $\mathbb{Q}$, $\mathbb{R}$ pertenecen cada uno de los siguientes números:
\begin{center}
\begin{tabular}{|c |c |c |c |c|}\hline
&$\mathbb{N}$& $\mathbb{Z}$& $\mathbb{Q}$&$\mathbb{R}$\\ 
\hline
$\frac{8}{16}$&&&&\\
\hline
$\sqrt[3]{-27}$&&&&\\
\hline
$3.0\wideparen{1}$&&&&\\
\hline
$-\frac{12}{4}$&&&&\\
\hline
$-\sqrt{25}$&&&&\\
\hline
$\sqrt{8}$&&&&\\
\hline
$4$&&&&\\
\hline
$\pi$&&&&\\
\hline
$\sqrt{-4}$&&&&\\
\hline
$\frac{39}{13}$&&&&\\
\hline
\end{tabular}

\end{center}

\begin{solution}
\begin{tabular}{|c |c |c |c |c|}\hline
&$\mathbb{N}$& $\mathbb{Z}$& $\mathbb{Q}$&$\mathbb{R}$\\ 
\hline
$\frac{8}{16}$&&&X&X\\
\hline
$\sqrt[3]{-27}$&&X&X&X\\
\hline
$3.0\wideparen{1}$&&&X&X\\
\hline
$-\frac{12}{4}$&&X&X&X\\
\hline
$-\sqrt{25}$&&X&X&X\\
\hline
$\sqrt{8}$&&&&X\\
\hline
$4$&X&X&X&X\\
\hline
$\pi$&&&&X\\
\hline
$\sqrt{-4}$&&&&\\
\hline
$\frac{39}{13}$&X&X&X&X\\
\hline
\end{tabular}
\end{solution}
\addpoints

\question[1] Representa en la recta real y en forma de intervalo el siguiente conjunto numérico:
\addpoints % to omit double points count
$$\left\{ x \in \mathbb{R} \left| -2 \leqslant x < 4 \right. \right\}$$

\begin{solution}
$\left[-2 \ , 4\right)$ 
\end{solution}

\question Opera:
%\noaddpoints % to omit double points count

\begin{parts}
\part[1] \[\dfrac{\left(2^3\cdot3^2\cdot5\right)^{-4}}{\left(2^{-2}\cdot3^{-3}\right)^{3}}\]
\begin{solution}
$\dfrac{3}{2^6\cdot5^4} $ 
\end{solution}
\part[1] \[\dfrac{9xy^3z^2}{14x^0yz^3}:\dfrac{18x^2yz^2}{21xy^3z}\]
\begin{solution}
 $\frac{3y^4}{4z^2}$ 
\end{solution}
\end{parts}

\addpoints

\question[2] Expresa en notación científica, opera y simplifica:
\addpoints 

$$\dfrac{0'0001\cdot0'01\cdot10000}{0'1\cdot100\cdot0'01}$$
\begin{solution} $10^{-1}$ \end{solution}
 

\question Opera y simplifica cada una de estas expresiones:
%\noaddpoints % to omit double points count

\begin{parts}
\part[1] \[4\sqrt{20}-3\sqrt{45}+11\sqrt{125}-20\sqrt{5}\]
\begin{solution}
$34\sqrt{5}$
\end{solution}

\part[1] \[\sqrt{72}\cdot3\sqrt{8}\]
\begin{solution}
$72$
\end{solution}

\part[1] \[\sqrt{2\sqrt{2\sqrt{2}}}\]
\begin{solution}
$\sqrt[8]{2^7}$
\end{solution}

\part[2] \[\dfrac{\left(3\sqrt{2}+\sqrt{3}\right)^2}{3}\]
\begin{solution}
$ 7+2\sqrt{6}$
\end{solution}


\end{parts}
\addpoints

\question Racionaliza y simplifica:
%\noaddpoints % to omit double points count
\begin{parts}
\part[2] \[\dfrac{10}{2\sqrt{3}-\sqrt{2}}\]
\begin{solution}
$2\sqrt{3}+\sqrt{2}$
\end{solution}


\part[2] \[\dfrac{\sqrt{45}+\sqrt{180}}{\sqrt{176}+4\sqrt{44}}\]
\begin{solution}
$\frac{3\sqrt{55}}{44}$
\end{solution}



\end{parts}
\addpoints


\question Sin utilizar la calculadora, resuelve los siguientes logaritmos:
\begin{multicols}{2}

\begin{parts}

\part[1] $\log_8 \frac{1}{8}$
\begin{solution}
$-1$
\end{solution}

\part[1] $\log_9 3$
\begin{solution}
$\frac{1}{2}$
\end{solution}
\part[1] ${\log _3}27$ \begin{solution} ${\log _3}27 = {\log _3}{3^3} = 3$ \end{solution}
\part[1] ${\log _5}\sqrt {125} $ \begin{solution} ${\log _5}\sqrt {125}  = {\log _5}{5^{{\raise0.7ex\hbox{$3$} \!\mathord{\left/
 {\vphantom {3 2}}\right.\kern-\nulldelimiterspace}
\!\lower0.7ex\hbox{$2$}}}} = \frac{3}{2}$ \end{solution}
\end{parts}
\end{multicols}


\addpoints


\question Calcula:
%\noaddpoints % to omit double points count

\begin{parts}


\part[1] \[\log_2 8 +  \log_3 27 + \log_5 125\]
\begin{solution}
$9$
\end{solution}




\end{parts}

\addpoints


\question Calcula aplicando la propiedades de los logaritmos:
%\noaddpoints % to omit double points count

\begin{parts}
\part[2] \[\log 4+ 	\log 8 + 3\log 5 - 2\log 2\]
\begin{solution}
$1$
\end{solution}

\end{parts}

\addpoints

\end{questions}

\end{document}
\grid
