\documentclass[addpoints,spanish, 12pt,a4paper]{exam}
%\documentclass[answers, spanish, 12pt,a4paper]{exam}
\printanswers
\pointpoints{punto}{puntos}
\hpword{Puntos:}
\vpword{Puntos:}
\htword{Total}
\vtword{Total}
\hsword{Resultado:}
\hqword{Ejercicio:}
\vqword{Ejercicio:}

\usepackage[utf8]{inputenc}
\usepackage[spanish]{babel}
\usepackage{eurosym}
%\usepackage[spanish,es-lcroman, es-tabla, es-noshorthands]{babel}

\usepackage{pgf,tikz}
\usetikzlibrary{shapes, calc, shapes, arrows, math, babel}
\usepackage[margin=1in]{geometry}
\usepackage{amsmath,amssymb}
\usepackage{multicol}

\usepackage{yhmath}

\usepackage{verbatim}
\usepackage[thinlines]{easytable}
%\usepackage{pstricks}


\usepackage{graphicx}
\graphicspath{{../img/}} 

\newcommand{\class}{4º Académicas}
\newcommand{\examdate}{\today}
\newcommand{\examnum}{Global}
\newcommand{\tipo}{B}


\newcommand{\timelimit}{50 minutos}

\renewcommand{\solutiontitle}{\noindent\textbf{Solución:}\enspace}


\pagestyle{head}
\firstpageheader{\includegraphics[width=0.2\columnwidth]{header_left}}{\textbf{Departamento de Matemáticas\linebreak \class}\linebreak \examnum}{\includegraphics[width=0.1\columnwidth]{header_right}}
\runningheader{\class}{\examnum}{Página \thepage\ of \numpages}
\runningheadrule
\pointsinrightmargin % Para poner las puntuaciones a la derecha. Se puede cambiar. Si se comenta, sale a la izquierda.
\extrawidth{-2.4cm} %Un poquito más de margen por si ponemos textos largos.
\marginpointname{ \emph{\points}}

\begin{document}

\noindent
\begin{tabular*}{\textwidth}{l @{\extracolsep{\fill}} r @{\extracolsep{6pt}} }
\textbf{Nombre:} \makebox[3.5in]{\hrulefill} & \textbf{Fecha:}\makebox[1in]{\hrulefill} \\
 & \\
\textbf{Tiempo: \timelimit} & Tipo: \tipo 
\end{tabular*}
\rule[2ex]{\textwidth}{2pt}


\textbf{Instrucciones:} \begin{itemize}
\item \textbf{Si tienes alguna/s evaluación pendiente:} Tienes que hacer \textbf{todos} los ejercicios salvo el último
\item \textbf{Si tienes todas las evaluaciones aprobadas:} Tienes que hacer el \textbf{último ejercicio}, y luego del resto cuatro ejercicios

\end{itemize}


\begin{comment}
Esta prueba tiene \numquestions\ ejercicios. La puntuación máxima es de \numpoints. 
La nota final de la prueba será la parte proporcional de la puntuación obtenida sobre la puntuación máxima. 

\begin{center}


\addpoints
 %\gradetable[h][questions]
	\pointtable[h][questions]
\end{center}

\noindent
\textbf{ACLARACIÓN:} Los ejercicios de geometría se han de resolver de manera analítica (no gráfica). Los ejercicios de funciones deberán estar justificados con los cálculos que sean necesarios para su resolución. 
\end{comment}\\

\rule[2ex]{\textwidth}{2pt}

\begin{questions}

\begin{comment}
\question[1] 
\begin{solution} \end{solution}

\addpoints
\end{comment}

\question Calcula: 
\begin{parts}
\part[1] Racionaliza y simplifica:  $\dfrac{\sqrt{3}}{2\sqrt{3}-\sqrt{2}}$ 
\begin{solution} $=\dfrac{\sqrt{3}\cdot\left(2\sqrt{3}+\sqrt{2}\right)}{\left(2\sqrt{3}-\sqrt{2}\right)\left(2\sqrt{3}+\sqrt{2}\right)}=\dfrac{6+\sqrt{6}}{12-2}=\dfrac{6+\sqrt{6}}{10}$ \end{solution}
\part[1] Aplica la definición de logaritmo para calcular: $\log_4 \sqrt{0,25}$
\begin{solution} $\to 4^x=\sqrt{\frac{1}{4}}\to4^x=4^{-1/2}\to\log_4 \sqrt{0,25}=-\frac{1}{2}$ \end{solution}
\part[1]Resuelve la siguiente ecuación: 
$\sqrt{3x-2}+\sqrt{x-1}=3$
\begin{solution}$\to\sqrt{3x-2}=3-\sqrt{x-1}\to 3x-2=9+x-1-6\sqrt{x-1}\to6\sqrt{x-1}=9+x-1-3x+32\to6\sqrt{x-1}=10-2x\to3\sqrt{x-1}=5-x\to x-1=25+x^2-10x\to x^2-19x+34=0$. Soluciones: $x=2$ (Sí) y $x=17$ No  \end{solution}


\end{parts}



\question Resuelve las siguientes inecuaciones de manera justificada:
\begin{parts}

\part[1]$ x < x^3  $
%solve_univariate_inequality (x-x**3<0,x,relational =false )  
\begin{solution} $\left(-1, 0\right) \cup \left(1, \infty\right)$ \end{solution}

\part[]$\dfrac{2x-2}{1-3x}<-\dfrac{2}{3} $  
%from sympy.solvers.inequalities import reduce_rational_inequalities
%reduce_rational_inequalities([[(2*x-2)/(1-3*x) < -2/3]], x,relational=0)
\begin{solution} $\left(-\infty, \frac{1}{3}\right)$\end{solution}
\begin{comment}
\part[1]$\dfrac{x - 1}{x^{2} + x}\geqslant 0$  
%from sympy.solvers.inequalities import reduce_rational_inequalities
%reduce_rational_inequalities([[expand((x-1))/expand(x*(x+1)) >= 0]], x,relational=0)
\begin{solution} $\left(-1, 0\right) \cup \left[1, \infty\right)$ \end{solution}
\end{comment}

\end{parts}

\question[2] Un triángulo isósceles mide 32 cm de perímetro y la altura correspondiente al lado
desigual mide 8 cm. Calcula los lados del triángulo y su área.
\begin{solution}
Los lados iguales miden 10 cm, y el lado desigual, 12 cm.
\end{solution}

\begin{comment}
\question[2] El  diámetro  de  la  base  de  un  cilindro  es  igual  a  su  altura.  El  área  total  es  169,56 metros cuadrados. Calcula sus dimensiones
\begin{solution}
d=h=6m
\end{solution}
\end{comment}

\question[2]   En lo alto de un edificio en construcción hay una grúa de 4 m. Desde un punto del
suelo se ve el punto más alto de la grúa bajo un ángulo de 45º con respecto a la horizontal y el punto más alto del edificio bajo un ángulo de 40º con la horizontal. Calcula la
altura del edificio.
\begin{solution} $\left. \begin{gathered}
	  \tg 40 = \frac{h}{x} \\
	  \tg 45 = \frac{h+4}{x} \hfill
	 \end{gathered}  \right\rbrace \\
	 \to 
	 h= \tg 40 \cdot x \to h = 20.86    $ \end{solution}

\question Resuelve las siguientes cuestiones relacionadas con combinatoria:
\begin{parts}
\part[1] ¿De cuántas formas podrán distribuirse dos premios iguales entre diez aspirantes?  
\begin{solution} $ C_{10}^2=\frac{10!}{8!\cdot2!}=45$  \end{solution}

\part[1] ¿Cuántas palabras se pueden formar con las letras de la palabra AMBROSI de forma que comiencen y terminen por vocal?
\begin{solution}
$V_3^2\cdot P_5= 3*2 \cdot 5!=6\cdot120=720 $
\end{solution}
\begin{comment}
\part[1] ¿Cuántos números naturales se pueden formar con las cifras 1, 3, 5 y 7 sin repetir ninguna de ellas? 
\begin{solution} $ V_4^1+V_4^2+V_4^3+V_4^4 \to ([4, 12, 24, 24], 64)$  \end{solution}
\end{comment}

\end{parts}



\question Dados el triángulo de vértices   $A(3, -1)$ ,   $B(5, 3)$ y   $C(-1, 3)$, determina:
\begin{parts} 
\begin {comment}
\part[1] si están alineados
\begin{solution} (False, Point2D(2, 4), Point2D(-6, 0)) \end{solution}
\end {comment}


\part[1]  La recta que contiene a la altura que pasa por $A$ y la recta que contiene a la altura $C$
\begin{solution} $x= 3$ (-2*x - 4*y + 10 = 0)\end{solution}


\part[1] El punto donde se cortan ambas rectas. 
\begin{solution} {x: 3, y: 1} \end{solution}

\end{parts}
\addpoints

\begin{comment}
\question  Dada la siguiente función $f(x) =
\left\{
	\begin{array}{clc}
		-2x  & \mbox{si } & x < -2 \\
		x^2-2x+1 & \mbox{si } & -2 \leq x < 2 \\
		2x-3 & \mbox{si } & x > 2
	\end{array}
\right.$
\begin{parts} 
\part[2] Representa la función gráficamente 
\begin{solution} 

\begin{tikzpicture}[domain=-4.1:5.1 ,>=triangle 45, scale=0.5]

\tikzmath{
			\a = 1; \b = -2; \c = 1; 
			\v = - \b / ( 2 * \a);
			\m1 = -2; \n1 = 0;
			\m3 = 2; \n3 = -3;
			\xmin1 = - 4.1; \xmax1 = -2;
			\xmin2 = - 2; \xmax2 = 2;
			\xmin3 = 2; \xmax3 = 5.1;
          }
          
 
\draw[color=red, domain=\xmin1 -0.5:\xmax1]    plot (\x,{\m1*(\x) + \n1}) node[right] {};

\draw [red] (\xmax1,{\m1*(\xmax1) + \n1}) circle (0.25) node [left] {};

\draw[color=red, domain=\xmin2:\xmax2]    plot (\x,{\a*(\x)^2 + \b *\x + \c})             node[right] {}; 
\draw [red, fill] (\xmin2,{\a*(\xmin2)^2 + \b *\xmin2 + \c}) circle (0.25) node [left] {};
\draw [red] (\xmax2,{\a*(\xmax2)^2 + \b *\xmax2 + \c}) circle (0.25) node [left] {};

\draw[color=red, domain=\xmin3  :\xmax3 + 0.1]    plot (\x,{\m3*(\x) + \n3}) node[right] {};
\draw [red] (\xmin3,{\m3*(\xmin3) + \n3}) circle (0.25) node [left] {};

\draw[very thin,color=lightgray,dash pattern=on 1pt off 1pt] (\xmin1 - 0.5, \a * \v * \v + \b * \v + \c - 0.5) grid (\xmax3 + 0.5 , \a * \xmin2 * \xmin2 + \b * \xmin2 + \c + 0.5);

\draw[<->] (\xmin1 -1,0) -- (\xmax3 + 1,0) node[right] {$x$};
\draw[<->] (0,\a * \v * \v + \b * \v + \c - 0.5) -- (0, \a * \xmin2 * \xmin2 + \b * \xmin2 + \c + 0.5 ) node[above] {$y$};

\end{tikzpicture}  \end{solution}
\part[1] Indica el \emph{dominio} y el \emph{recorrido} de la función utilizando la notación de conjuntos de números reales
\begin{solution} $Dom(f)=\mathbb{R}-\lbrace 2 \rbrace $ \\
$ Im(f)=\left[0, +\infty\right]$
\end{solution}


\end{parts}
\addpoints
\end{comment}

\begin{comment}
\question Dada la siguiente función a trozos:

\begin{tikzpicture}[domain=-4.1:5.1 ,>=triangle 45, scale=0.75]

\tikzmath{
			\a = 1; \b = -2; \c = 1; 
			\v = - \b / ( 2 * \a);
			\m1 = -1; \n1 = -2;
			\m2 = (1/3); \n2 = 1/3;
			\m3 = 1; \n3 = -2;
			\xmin1 = - 4.1; \xmax1 = -1;
			\xmin2 = - 1; \xmax2 = 2;
			\xmin3 = 2; \xmax3 = 5.1;
          }
          
 
\draw[<-,color=red, domain=\xmin1 -0.5:\xmax1]    plot (\x,{\m1*(\x) + \n1}) node[right] {};

\draw [red] (\xmax1,{\m1*(\xmax1) + \n1}) circle (0.25) node [left] {};

\draw[color=red, domain=\xmin2:\xmax2]    plot (\x,{\m2*(\x) + \n2}) node[right] {};

\draw [red, fill] (\xmin2,{\m2*(\xmin2) + \n2}) circle (0.25) node [left] {};
\draw [red] (\xmax2,{\m2*(\xmax2) + \n2}) circle (0.25) node [left] {};

%\draw[color=red, domain=\xmin2:\xmax2]    plot (\x,{\a*(\x)^2 + \b *\x + \c})             node[right] {}; 

%\draw [red, fill] (\xmin2,{\a*(\xmin2)^2 + \b *\xmin2 + \c}) circle (0.25) node [left] {};
%\draw [red] (\xmax2,{\a*(\xmax2)^2 + \b *\xmax2 + \c}) circle (0.25) node [left] {};

\draw[->, color=red, domain=\xmin3  :\xmax3 + 0.1]    plot (\x,{\m3*(\x) + \n3}) node[right] {};
\draw [red] (\xmin3,{\m3*(\xmin3) + \n3}) circle (0.25) node [left] {};

\draw[very thin,color=lightgray,dash pattern=on 1pt off 1pt] (\xmin1 - 0.5, -2) grid (\xmax3 + 0.5 , \a * \xmin2 * \xmin2 + \b * \xmin2 + \c + 0.5);

\draw[<->] (\xmin1 -1,0) -- (\xmax3 + 1,0) node[right] {$x$};
\draw[<->] (0,-2 - 0.5) -- (0, \a * \xmin2 * \xmin2 + \b * \xmin2 + \c + 0.5 ) node[above] {$y$};

\end{tikzpicture}

\begin{parts} 
\part[1] Indica el \emph{dominio} y el \emph{recorrido} de la función utilizando la notación de conjuntos de números reales
\begin{solution} $Dom(f)=\mathbb{R}-\lbrace 2 \rbrace $ \\
$ Im(f)=\left(-1, +\infty\right)$
\end{solution}
\part[1] Calcula las ecuaciones explícitas de las rectas  que contienen a cada trozo de la función. 
\begin{solution} 
$y=-x-2$, $y=\frac{1}{3}x+\frac{1}{3}$, $y=x-2 $
\end{solution}
\part[2] Da la expresión analítica de la función a trozos
\begin{solution} 
$f(x) =
\left\{
	\begin{array}{clc}
		-x -2  & \mbox{si } & x < -1 \\
		\frac{1}{3}x+\frac{1}{3} & \mbox{si } & -1 \leq x < 2 \\
		x-2 & \mbox{si } & x > 2
	\end{array}
\right.$
\end{solution}
\end{parts}
\end{comment}

\begin{comment}
\question Dada la siguiente función a trozos:

\begin{tikzpicture}[domain=-4.1:5.1 ,>=triangle 45, scale=0.75]

\tikzmath{
			\a = 1; \b = -2; \c = 1; 
			\v = - \b / ( 2 * \a);
			\m1 = 2; \n1 = 3;
			\m2 = -(1/3); \n2 = -1/3;
			\m3 = -1; \n3 = 2;
			\xmin1 = - 4.1; \xmax1 = -1;
			\xmin2 = - 1; \xmax2 = 2;
			\xmin3 = 2; \xmax3 = 5.1;
          }
          
 
\draw[<-,color=red, domain=\xmin1 -0.5:\xmax1]    plot (\x,{\m1*(\x) + \n1}) node[right] {};

\draw [red] (\xmax1,{\m1*(\xmax1) + \n1}) circle (0.25) node [left] {};

\draw[color=red, domain=\xmin2:\xmax2]    plot (\x,{\m2*(\x) + \n2}) node[right] {};

\draw [red, fill] (\xmin2,{\m2*(\xmin2) + \n2}) circle (0.25) node [left] {};
\draw [red] (\xmax2,{\m2*(\xmax2) + \n2}) circle (0.25) node [left] {};

%\draw[color=red, domain=\xmin2:\xmax2]    plot (\x,{\a*(\x)^2 + \b *\x + \c})             node[right] {}; 

%\draw [red, fill] (\xmin2,{\a*(\xmin2)^2 + \b *\xmin2 + \c}) circle (0.25) node [left] {};
%\draw [red] (\xmax2,{\a*(\xmax2)^2 + \b *\xmax2 + \c}) circle (0.25) node [left] {};

\draw[->, color=red, domain=\xmin3  :\xmax3 + 0.1]    plot (\x,{\m3*(\x) + \n3}) node[right] {};
\draw [red] (\xmin3,{\m3*(\xmin3) + \n3}) circle (0.25) node [left] {};

\draw[very thin,color=lightgray,dash pattern=on 1pt off 1pt] (\xmin1 - 0.5, -6) grid (\xmax3 + 0.5 , \a * \xmin2 * \xmin2 + \b * \xmin2 + \c + 0.5);

\draw[<->] (\xmin1 -1,0) -- (\xmax3 + 1,0) node[right] {$x$};
\draw[<->] (0,-6 - 0.5) -- (0, \a * \xmin2 * \xmin2 + \b * \xmin2 + \c + 0.5 ) node[above] {$y$};

\end{tikzpicture}

\begin{parts} 
\part[1] Indica el \emph{dominio} y el \emph{recorrido} de la función utilizando la notación de conjuntos de números reales
\begin{solution} $Dom(f)=\mathbb{R}-\lbrace 2 \rbrace $ \\
$ Im(f)=\left(-\infty, 1\right)$
\end{solution}
\part[1] Calcula las ecuaciones explícitas de las rectas  que contienen a cada trozo de la función. 
\begin{solution} 
$y=2x+3$, $y=-\frac{1}{3}x-\frac{1}{3}$, $y=-x+2 $
\end{solution}
\part[2] Da la expresión analítica de la función a trozos
\begin{solution} 
$f(x) =
\left\{
	\begin{array}{clc}
		2x+3  & \mbox{si } & x < -1 \\
		-\frac{1}{3}x-\frac{1}{3} & \mbox{si } & -1 \leq x < 2 \\
		-x+2 & \mbox{si } & x > 2
	\end{array}
\right.$
\end{solution}
\end{parts}

\end{comment}

\question[1] Halla el área de un paralelogramo cuyos lados miden 16 cm y 24 cm y forman un
ángulo de 40º.
\begin{solution}
A=$246.72 cm^2$
\end{solution}
\begin{comment}
\question En una urna hay cinco bolas blancas y cuatro negras. Se extraen dos bolas \textbf{sin} reemplazamiento. Cuál es la probabilidad de que sean: 

\begin{parts}
\part de distinto color
\begin{solution}
\tikzstyle{bag} = [text width=4em, text centered]
\tikzstyle{end} = [circle, minimum width=3pt,fill, inner sep=0pt]
\tikzstyle{level 1} = [level distance=3.5cm, sibling distance=3.5cm]
\tikzstyle{level 2} = [level distance=3.5cm, sibling distance=2cm]

\begin{tikzpicture}[grow=right, sloped, scale=1]
\node[bag] {$2B, 1N$}
    child {
        node[bag] {$2B, 0N$}        
            child {
                node[end, label=right:
                    {$P(N_1\cap N_2)=\frac{1}{3}\cdot\frac{0}{3}=0$}] {}
                edge from parent
                node[above] {$N$}
                node[below]  {$\frac{0}{2}$}
            }
            child {
                node[end, label=right:
                    {$P(N_1\cap B_2)=\frac{1}{3}\cdot\frac{2}{2}$}] {}
                edge from parent
                node[above] {$B$}
                node[below] {$\frac{2}{2}$}
            }
            edge from parent 
            node[above] {$N$}
            node[below]  {$\frac{1}{3}$}
    }
    child {
        node[bag] {$1B, 1N$}        
        child {
                node[end, label=right:
                    {$P(B_1\cap N_2)=\frac{2}{3}\cdot\frac{1}{2}$}] {}
                edge from parent
                node[above] {$N$}
                node[below]  {$\frac{1}{2}$}
            }
            child {
                node[end, label=right:
                    {$P(B_1\cap B_2)=\frac{2}{3}\cdot\frac{1}{2}$}] {}
                edge from parent
                node[above] {$B$}
                node[below]  {$\frac{1}{2}$}
            }
        edge from parent         
            node[above] {$B$}
            node[below]  {$\frac{2}{3}$}
    };
\end{tikzpicture} \\
$ P(Distinto\ color) = P(B_1\cap N_2) + P(N_1\cap B_2) = \frac{1}{3}+\frac{1}{3}= \frac{2}{3}$
\end{solution}

\part del mismo color
\begin{solution}
$P(Mismo\ color) = P(B_1\cap B_2) + P(N_1\cap N_2) = \frac{1}{3}+\frac{0}{9}= \frac{1}{3}$ 
\end{solution}

\part Cuál es la probabilidad de que, habiendo sido la segunda bola blanca, la primera haya sido blanca:
\begin{solution}
$P(B_1|B_2)=\dfrac{P(B_1\cap B_2)}{P(B_2)}=\dfrac{\frac{2}{3}\cdot\frac{1}{2}}{\frac{2}{3}\cdot\frac{1}{2}+\frac{1}{3}\cdot\frac{2}{2}}=\dfrac{\frac{1}{3}}{\frac{4}{6}}=\frac{1}{2} $
\end{solution}

\part Cuál es la probabilidad de que, habiendo sido la segunda bola blanca, la primera haya sido negra:
\begin{solution}
$P(N_1|B_2)=\dfrac{P(N_1\cap B_2)}{P(B_2)}=\dfrac{\frac{1}{3}\cdot\frac{2}{2}}{\frac{2}{3}\cdot\frac{1}{2}+\frac{1}{3}\cdot\frac{2}{2}}=\dfrac{\frac{1}{3}}{\frac{4}{6}}=\frac{1}{2} $
\end{solution}

\end{parts}
\end{comment}


\end{questions}


\end{document}

