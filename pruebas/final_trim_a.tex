\documentclass[addpoints,spanish, 12pt,a4paper]{exam}
%\documentclass[answers, spanish, 12pt,a4paper]{exam}
%\printanswers
\pointpoints{punto}{puntos}
\hpword{Puntos:}
\vpword{Puntos:}
\htword{Total}
\vtword{Total}
\hsword{Resultado:}
\hqword{Ejercicio:}
\vqword{Ejercicio:}

\usepackage[utf8]{inputenc}
\usepackage[spanish]{babel}
\usepackage{eurosym}
%\usepackage[spanish,es-lcroman, es-tabla, es-noshorthands]{babel}


\usepackage[margin=1in]{geometry}
\usepackage{amsmath,amssymb}
\usepackage{multicol}
\usepackage{yhmath}

\usepackage{verbatim}
%\usepackage{pstricks}


\usepackage{graphicx}
\graphicspath{{../img/}} 

\newcommand{\class}{4º Académicas}
\newcommand{\examdate}{\today}
\newcommand{\examnum}{Examen final de trimestre 1}
\newcommand{\tipo}{A}


\newcommand{\timelimit}{50 minutos}

\renewcommand{\solutiontitle}{\noindent\textbf{Solución:}\enspace}


\pagestyle{head}
\firstpageheader{\includegraphics[width=0.2\columnwidth]{header_left}}{\textbf{Departamento de Matemáticas\linebreak \class}\linebreak \examnum}{\includegraphics[width=0.1\columnwidth]{header_right}}
\runningheader{\class}{\examnum}{Página \thepage\ of \numpages}
\runningheadrule


\begin{document}

\noindent
\begin{tabular*}{\textwidth}{l @{\extracolsep{\fill}} r @{\extracolsep{6pt}} }
\textbf{Nombre:} \makebox[3.5in]{\hrulefill} & \textbf{Fecha:}\makebox[1in]{\hrulefill} \\
 & \\
\textbf{Tiempo: \timelimit} & Tipo: \tipo 
\end{tabular*}
\rule[2ex]{\textwidth}{2pt}
Esta prueba tiene \numquestions\ ejercicios. La puntuación máxima es de \numpoints. 
La nota final de la prueba será la parte proporcional de la puntuación obtenida sobre la puntuación máxima. Para la recuperación de pendientes de 3º se tendrán en cuenta los apartados: 1.a y 4.a

\begin{center}


\addpoints
 %\gradetable[h][questions]
	\pointtable[h][questions]
\end{center}

\noindent
\rule[2ex]{\textwidth}{2pt}

\begin{questions}

\begin{comment}
\question[1] 
\begin{solution} \end{solution}

\addpoints
\end{comment}

\question Calcula: 
\begin{parts}
\part[1] Racionaliza y simplifica:  $\dfrac{\sqrt{3}}{2\sqrt{3}-\sqrt{2}}$ 
\begin{solution} $=\dfrac{\sqrt{3}\cdot\left(2\sqrt{3}+\sqrt{2}\right)}{\left(2\sqrt{3}-\sqrt{2}\right)\left(2\sqrt{3}+\sqrt{2}\right)}=\dfrac{6+\sqrt{6}}{12-2}=\dfrac{6+\sqrt{6}}{10}$ \end{solution}
\part[1] Aplica la definición de logaritmo para calcular: $\log_4 \sqrt{0,25}$
\begin{solution} $\to 4^x=\sqrt{\frac{1}{4}}\to4^x=4^{-1/2}\to\log_4 \sqrt{0,25}=-\frac{1}{2}$ \end{solution}

\end{parts}

\addpoints

\question[1] Utilizando el teorema del resto para el polinomio $P(x)=-2x^3 + x^2 - 3x - 6$, resuelve: 
\begin{parts}
\part Valor numérico para $x=-1$ 
\begin{solution} $0$ \end{solution}
\part ¿Es divisible $P(x)$ por $x+1$? Justifica tu respuesta 
\begin{solution} Sí. Por el teorema del resto \end{solution}
\end{parts}

\addpoints




\question[1] Simplifica la fracción algebraica: $$\dfrac{2x^3-5x^2+3x}{2x^2+x-6} $$
\begin{solution}$=\dfrac{2x\left(x-1\right)\left(x-\dfrac{3}{2}\right)}{2\left(x+2\right)\left(x-\dfrac{3}{2}\right)}=\dfrac{x(x-1)}{x+2}$  \end{solution}

\addpoints



\question Resuelve las siguientes ecuaciones: 
\begin{parts}
\part[2] $$\dfrac{2x}{x+1}-\dfrac{1}{x}=\dfrac{5}{6}$$
\begin{solution}$\to \dfrac{12x^2}{6x\left(x+1\right)}-\dfrac{6\left(x+1\right)}{5x\left(x+1\right)\left(x+1\right)}=\dfrac{5}{6x\left(x+1\right)}\to 12x^2-6x-6=5x^2+5x\to 7x^2-11x-6=0\to x=2 \ x=-\frac{3}{7}$ \end{solution}

\part[2] $$2x^4-6x^3+6x^2-2x=0$$
\begin{solution}$P(x)2x^4-6x^3+6x^2-2x=2x\left(x-1\right)^3$. Soluciones: $x=0$ y $x=1$ triple\end{solution}

\part[2] $$\sqrt{3x-2}+\sqrt{x-1}=3$$
\begin{solution}$\to\sqrt{3x-2}=3-\sqrt{x-1}\to 3x-2=9+x-1-6\sqrt{x-1}\to6\sqrt{x-1}=9+x-1-3x+32\to6\sqrt{x-1}=10-2x\to3\sqrt{x-1}=5-x\to x-1=25+x^2-10x\to x^2-19x+34=0$. Soluciones: $x=2$ (Sí) y $x=17$ No  \end{solution}

\part[2] $$2\log x- \log{\left(3x-5\right)}=\log{5x} - 1$$
\begin{solution}$\to 2\log x- \log{\left(3x-5\right)}=\log{5x} - \log {10} \to \log {\frac{x^2}{3x-5}}=\log {\frac{5x}{10}}\to$ \\
$\to \frac{x^2}{3x-5}=\frac{x}{2}\to2x^2=3x^2-5x\to 0=x^2-5 \to 0=x\left(x-5\right)\to x=0 \ y \ x=5 \to$ \\ $\to$ de las dos soluciones, la única válida es $x=5$ ya que $\log 0$ no existe \end{solution}

\end{parts}

\addpoints

\end{questions}

\end{document}
\grid
