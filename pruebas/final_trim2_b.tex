\documentclass[addpoints,spanish, 12pt,a4paper]{exam}
%\documentclass[answers, spanish, 12pt,a4paper]{exam}
%\printanswers
\pointpoints{punto}{puntos}
\hpword{Puntos:}
\vpword{Puntos:}
\htword{Total}
\vtword{Total}
\hsword{Resultado:}
\hqword{Ejercicio:}
\vqword{Ejercicio:}

\usepackage[utf8]{inputenc}
\usepackage[spanish]{babel}
\usepackage{eurosym}
%\usepackage[spanish,es-lcroman, es-tabla, es-noshorthands]{babel}


\usepackage[margin=1in]{geometry}
\usepackage{amsmath,amssymb}
\usepackage{multicol}

\usepackage{yhmath}

\usepackage{verbatim}
\usepackage[thinlines]{easytable}
%\usepackage{pstricks}


\usepackage{graphicx}
\graphicspath{{../img/}} 

\newcommand{\class}{4º Académicas}
\newcommand{\examdate}{\today}
\newcommand{\examnum}{Examen de final de trimestre}
\newcommand{\tipo}{B}


\newcommand{\timelimit}{50 minutos}

\renewcommand{\solutiontitle}{\noindent\textbf{Solución:}\enspace}


\pagestyle{head}
\firstpageheader{\includegraphics[width=0.2\columnwidth]{header_left}}{\textbf{Departamento de Matemáticas\linebreak \class}\linebreak \examnum}{\includegraphics[width=0.1\columnwidth]{header_right}}
\runningheader{\class}{\examnum}{Página \thepage\ of \numpages}
\runningheadrule
\pointsinrightmargin % Para poner las puntuaciones a la derecha. Se puede cambiar. Si se comenta, sale a la izquierda.
\extrawidth{-2.4cm} %Un poquito más de margen por si ponemos textos largos.
\marginpointname{ \emph{\points}}

\begin{document}

\noindent
\begin{tabular*}{\textwidth}{l @{\extracolsep{\fill}} r @{\extracolsep{6pt}} }
\textbf{Nombre:} \makebox[3.5in]{\hrulefill} & \textbf{Fecha:}\makebox[1in]{\hrulefill} \\
 & \\
\textbf{Tiempo: \timelimit} & Tipo: \tipo 
\end{tabular*}
\rule[2ex]{\textwidth}{2pt}
Esta prueba tiene \numquestions\ ejercicios. La puntuación máxima es de \numpoints. 
La nota final de la prueba será la parte proporcional de la puntuación obtenida sobre la puntuación máxima. 

\begin{center}


\addpoints
 %\gradetable[h][questions]
	\pointtable[h][questions]
\end{center}

\noindent
\rule[2ex]{\textwidth}{2pt}

\begin{questions}

\begin{comment}
\question[1] 
\begin{solution} \end{solution}

\addpoints
\end{comment}


\question Resuelve las siguientes inecuaciones de manera justificada:
\begin{parts}

\part[1]$ x^{3}  + x <  2 x^{2} $
%solve_univariate_inequality (expand(x*(x-1)*(x-1))<0,x,relational =false )  
\begin{solution} $\left(-\infty, 0\right)$ \end{solution}


\part[2]$\dfrac{x - 1}{x^{2} + x}\geqslant 0$  
%from sympy.solvers.inequalities import reduce_rational_inequalities
%reduce_rational_inequalities([[expand((x-1))/expand(x*(x+1)) >= 0]], x,relational=0)
\begin{solution} $\left(-1, 0\right) \cup \left[1, \infty\right)$ \end{solution}


\end{parts}

\addpoints

\question[2] 
Calcula el perímetro y el área de un triángulo rectángulo sabiendo que la altura y la proyección de un cateto sobre la hipotenusa son de 2 cm y 2,5 cm, respectivamente.

\question Si $\tg \alpha = \sqrt{3}$, calcula usando radicales:
\begin{parts} 
\part[2] El resto de las razones trigonométricas principales usando las relaciones trigonométricas fundamenteles y sabiendo que $\alpha \in I$ (primer cuadrante)
\begin{solution} $\left. \begin{gathered}
	  2x - \frac{y}{2} \geqslant 40 \hfill \\
	  x + y = 40 \hfill
	 \end{gathered}  \right\rbrace \to 2x-\left(40-x\right)\cdot 0,5 \geqslant 40$ \end{solution}
\part[1] El resto de las razones trigonométricas principales usando el apartado anterior y sabiendo que $\alpha \in IV$ (cuarto cuadrante)
\begin{solution}  \end{solution}	 


\end{parts}
\addpoints


\question[1]  Calcula la altura de una torre sabiendo que su sombra mide 13 m cuando los rayos del
sol forman un ángulo de 50º con el suelo.
\begin{solution}  \end{solution}

\question[2]   Desde el lugar donde me encuentro la visual de la torre forma un ángulo de 32º con la
horizontal. Si me acerco 15 m, el ángulo es de 50º. ¿Cuál es la altura de la torre?
\begin{solution}  \end{solution}


\end{questions}

\end{document}
\grid
