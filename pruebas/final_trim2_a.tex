\documentclass[addpoints,spanish, 12pt,a4paper]{exam}
%\documentclass[answers, spanish, 12pt,a4paper]{exam}
\printanswers
\pointpoints{punto}{puntos}
\hpword{Puntos:}
\vpword{Puntos:}
\htword{Total}
\vtword{Total}
\hsword{Resultado:}
\hqword{Ejercicio:}
\vqword{Ejercicio:}

\usepackage[utf8]{inputenc}
\usepackage[spanish]{babel}
\usepackage{eurosym}
%\usepackage[spanish,es-lcroman, es-tabla, es-noshorthands]{babel}


\usepackage[margin=1in]{geometry}
\usepackage{amsmath,amssymb}
\usepackage{multicol}
\usepackage{yhmath}

\usepackage{verbatim}
%\usepackage{pstricks}


\usepackage{graphicx}
\graphicspath{{../img/}} 

\newcommand{\class}{4º Académicas}
\newcommand{\examdate}{\today}
\newcommand{\examnum}{Examen de final de trimestre}
\newcommand{\tipo}{A}


\newcommand{\timelimit}{50 minutos}

\renewcommand{\solutiontitle}{\noindent\textbf{Solución:}\enspace}


\pagestyle{head}
\firstpageheader{\includegraphics[width=0.2\columnwidth]{header_left}}{\textbf{Departamento de Matemáticas\linebreak \class}\linebreak \examnum}{\includegraphics[width=0.1\columnwidth]{header_right}}
\runningheader{\class}{\examnum}{Página \thepage\ of \numpages}
\runningheadrule
\pointsinrightmargin % Para poner las puntuaciones a la derecha. Se puede cambiar. Si se comenta, sale a la izquierda.
\extrawidth{-2.4cm} %Un poquito más de margen por si ponemos textos largos.
\marginpointname{ \emph{\points}}

\begin{document}

\noindent
\begin{tabular*}{\textwidth}{l @{\extracolsep{\fill}} r @{\extracolsep{6pt}} }
\textbf{Nombre:} \makebox[3.5in]{\hrulefill} & \textbf{Fecha:}\makebox[1in]{\hrulefill} \\
 & \\
\textbf{Tiempo: \timelimit} & Tipo: \tipo 
\end{tabular*}
\rule[2ex]{\textwidth}{2pt}
Esta prueba tiene \numquestions\ ejercicios. La puntuación máxima es de \numpoints. 
La nota final de la prueba será la parte proporcional de la puntuación obtenida sobre la puntuación máxima. Para la recuperación de pendientes de 3º se tendrán en cuenta los apartados: 1. 2.a y 4.a

\begin{center}


\addpoints
 %\gradetable[h][questions]
	\pointtable[h][questions]
\end{center}

\noindent
\rule[2ex]{\textwidth}{2pt}

\begin{questions}

\begin{comment}
\question[1] 
\begin{solution} \end{solution}

\addpoints
\end{comment}


\question Resuelve las siguientes inecuaciones:
\begin{parts}

\part[2]$ x^3 < x  $
%solve_univariate_inequality (expand(x*(x-1)*(x+1))<0,x,relational =false )  
\begin{solution} $\left(-\infty, -1\right) \cup \left(0, 1\right)$ \end{solution}



\part[2]$\dfrac{x^{2} - x}{x^{2} + x}\geqslant 0$  
%from sympy.solvers.inequalities import reduce_rational_inequalities
%reduce_rational_inequalities([[expand(x*(x-1))/expand(x*(x+1)) >= 0]], x,relational=0)
\begin{solution} $\left(-\infty, -1\right) \cup \left[1, \infty\right)$ \end{solution}


\end{parts}

\addpoints

\question Si la $\tg \alpha = 1$, calcula:
\begin{parts} 
\part[2] El resto de las razones trigonométricas principales usando las relaciones trigonométricas fundamenteles y sabiendo que $\alpha \in I$ (primer cuadrante)
\begin{solution} $\left. \begin{gathered}
	  2x - \frac{y}{2} \geqslant 40 \hfill \\
	  x + y = 40 \hfill
	 \end{gathered}  \right\rbrace \to 2x-\left(40-x\right)\cdot 0,5 \geqslant 40$ \end{solution}
\part[2] El resto de las razones trigonométricas principales usando las relaciones trigonométricas fundamenteles y sabiendo que $\alpha \in III$ (tercer cuadrante)
\begin{solution} $\left. \begin{gathered}
	  2x - \frac{y}{2} \geqslant 40 \hfill \\
	  x + y = 40 \hfill
	 \end{gathered}  \right\rbrace \to 2x-\left(40-x\right)\cdot 0,5 \geqslant 40$ \end{solution}	 
\part[2] Si $\alpha \in I$ 
\begin{solution} 24 o más \end{solution}

\end{parts}
\addpoints

\end{questions}

\end{document}
\grid
